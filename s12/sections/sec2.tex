\section{PS12, Ex. 2 (A): Farrell \& Rabin (1996): "Cheap talk"}

\begin{frame}{PS12, Ex. 2 (A): Farrell \& Rabin (1996): "Cheap talk"}
    \textit{(A)} In their paper “Cheap Talk” published in the \href{https://www.aeaweb.org/articles?id=10.1257/jep.10.3.103}{\textit{Journal of Economic Perspectives} (1996)}, Joseph Farrell and Matthew Rabin describe the following situation: “Sally knows which one of two tasks is efficient to perform. Rayco [the firm] could hire Sally specifically to perform Job 1, specifically to perform Job 2, or as a highly paid manager who will choose which job to perform. If Rayco knew which task is efficient, it would still hire her to perform the task, but at a lower salary, because she has lost her informational advantage. Sally wants to be hired as manager, but prefers to be hired to do the right task and be more productive rather than to do the wrong task and be less productive.” Payoffs in this situation are\vspace{-6pt}
    \begin{table}
      \begin{tabular}{ll|c|c|c|}
          & \multicolumn{1}{c}{} & \multicolumn{1}{c}{Job 1} & \multicolumn{1}{c}{Job 2} & \multicolumn{1}{c}{Manager} \\\cline{3-5}
          \parbox[t]{20mm}{\multirow{2}{*}{Sally's knowledge}}
           & Task 1 efficient & 2, 5 & 1, -2 & 3, 3 \\\cline{3-5}
           & Task 2 efficient & 1, -2 & 2, 5 & 3, 3 \\\cline{3-5}
      \end{tabular}
    \end{table}\vspace{-2pt}
    where the left number in each cell is Sally’s payoff and the right number is the firm’s payoff (note that this matrix does not describe the normal form of the game!)\vspace{-4pt}
    \begin{itemize}
      \item[(a)] Formulate this strategic situation as a cheap talk game, assuming that the type space is equal to the message space $(T = M)$.
      \item[(b)] Show that a separating equilibrium exists where Sally truthfully reveals which job is efficient, and the firm then places Sally in that specific job.
      \item[(c)] Discuss whether or not this separating equilibrium seems reasonable. Why isn’t Sally able to convince the firm to give her the manager position?
    \end{itemize}\vspace{-6pt}
    \vfill\null
\end{frame}



\subsection{PS12, Ex. 2.a (A): Formulate a cheap talk game}

\begin{frame}{PS12, Ex. 2.a (A): Farrell \& Rabin (1996). Formulate a cheap talk game}
    \begin{table}
      \begin{tabular}{l|c|c|c|}
          \multicolumn{1}{c}{} & \multicolumn{1}{c}{Job 1 $(a=T_1)$} & \multicolumn{1}{c}{Job 2 $(a=T_2)$} & \multicolumn{1}{c}{Manager $(a=Manager)$} \\\cline{2-4}
           Task 1 efficient $(t=T_1)$ & 2, 5 & 1, -2 & 3, 3 \\\cline{2-4}
           Task 2 efficient $(t=T_2)$ & 1, -2 & 2, 5 & 3, 3 \\\cline{2-4}
      \end{tabular}
    \end{table}\vspace{-12pt}
    \begin{itemize}
      \item[(a)] Formulate this strategic situation as a cheap talk game, assuming that the type space is equal to the message space $(T = M)$.
    \end{itemize}\vspace{-6pt}
    \vfill\null
\end{frame}

\begin{frame}{PS12, Ex. 2.a (A): Farrell \& Rabin (1996). Formulate a cheap talk game}
  \begin{table}
    \begin{tabular}{l|c|c|c|}
        \multicolumn{1}{c}{} & \multicolumn{1}{c}{Job 1 $(a=T_1)$} & \multicolumn{1}{c}{Job 2 $(a=T_2)$} & \multicolumn{1}{c}{Manager $(a=Manager)$} \\\cline{2-4}
         Task 1 efficient $(t=T_1)$ & 2, 5 & 1, -2 & 3, 3 \\\cline{2-4}
         Task 2 efficient $(t=T_2)$ & 1, -2 & 2, 5 & 3, 3 \\\cline{2-4}
    \end{tabular}
  \end{table}\vspace{-12pt}
  \begin{itemize}
      \item[(a)] Formulate this strategic situation as a cheap talk game, assuming that the type space is equal to the message space $(T = M)$.
    \end{itemize}\vspace{-6pt}
    The game is as follows:\vspace{-6pt}
    \begin{enumerate}
      \item Sally's type is realized: $t\in\{T_1,T_2\}$, where $t=T_1,T_2$ corresponds to efficiency with Task 1 and Task 2 respectively.
      \item Sally observes her type and sends a cheap talk message $m(t)\in\{T_1,T_2\}$.
      \item Rayco observes the message and chooses a job for Sally: $a(m)\in\{T_1,T_2,M\}$ where $a=T_1,T_2,M$ corresponds to giving Sally one of three jobs:
      \begin{itemize}\normalsize
        \item Job 1 (compatible with task 1 efficiency)
        \item Job 2 (compatible with task 2 efficiency)
        \item Manager (compatible with both but more expensive).
      \end{itemize}
    \end{enumerate}
    \vfill\null
\end{frame}



\subsection{PS12, Ex. 2.b (A): Find a separating PBE}

\begin{frame}{PS12, Ex. 2.b (A): Farrell \& Rabin (1996). Find a separating PBE}
    \begin{table}
      \begin{tabular}{l|c|c|c|}
          \multicolumn{1}{c}{} & \multicolumn{1}{c}{Job 1 $(a=T_1)$} & \multicolumn{1}{c}{Job 2 $(a=T_2)$} & \multicolumn{1}{c}{Manager $(a=Manager)$} \\\cline{2-4}
           Task 1 efficient $(t=T_1)$ & 2, 5 & 1, -2 & 3, 3 \\\cline{2-4}
           Task 2 efficient $(t=T_2)$ & 1, -2 & 2, 5 & 3, 3 \\\cline{2-4}
      \end{tabular}
    \end{table}\vspace{-12pt}
    \begin{itemize}
      \item[(b)] Show that a separating equilibrium exists where Sally truthfully reveals which job is efficient, and the firm then places Sally in that specific job.
    \end{itemize}\vspace{-6pt}
    \vfill\null
\end{frame}
\begin{frame}{PS12, Ex. 2.b (A): Farrell \& Rabin (1996). Find a separating PBE}
    \begin{table}
      \begin{tabular}{l|c|c|c|}
          \multicolumn{1}{c}{} & \multicolumn{1}{c}{Job 1 $(a=T_1)$} & \multicolumn{1}{c}{Job 2 $(a=T_2)$} & \multicolumn{1}{c}{Manager $(a=Manager)$} \\\cline{2-4}
           Task 1 efficient $(t=T_1)$ & 2, 5 & 1, -2 & 3, 3 \\\cline{2-4}
           Task 2 efficient $(t=T_2)$ & 1, -2 & 2, 5 & 3, 3 \\\cline{2-4}
      \end{tabular}
    \end{table}\vspace{-12pt}
    \begin{itemize}
      \item[(b)] Show that a separating equilibrium exists where Sally truthfully reveals which job is efficient, and the firm then places Sally in that specific job.
      \item[Step 1:] \textbf{Go over the beliefs and actions in such a separating PBE. Does either type want to deviate?}
    \end{itemize}
    \vfill\null
\end{frame}
\begin{frame}{PS12, Ex. 2.b (A): Farrell \& Rabin (1996). Find a separating PBE}
    \begin{table}
      \begin{tabular}{l|c|c|c|}
          \multicolumn{1}{c}{} & \multicolumn{1}{c}{Job 1 $(a=T_1)$} & \multicolumn{1}{c}{Job 2 $(a=T_2)$} & \multicolumn{1}{c}{Manager $(a=Manager)$} \\\cline{2-4}
           Task 1 efficient $(t=T_1)$ & 2, 5 & 1, -2 & 3, 3 \\\cline{2-4}
           Task 2 efficient $(t=T_2)$ & 1, -2 & 2, 5 & 3, 3 \\\cline{2-4}
      \end{tabular}
    \end{table}\vspace{-12pt}
    \begin{itemize}
      \item[(b)] Show that a separating equilibrium exists where Sally truthfully reveals which job is efficient, and the firm then places Sally in that specific job.
      \item[Step 1:] Go over the beliefs and actions in such a separating PBE. Does either type want to deviate?
    \end{itemize}\vspace{-6pt}
    In a PBE, the beliefs must correspond to the action of the senders.\\
    Thus in a separating PBE where $m(t=T_1)=T_1$ and $m(t=T_2)=T_2$, beliefs are\vspace{-2pt}
    \begin{align*}
      \mu(t=T_1|m=T_1)=1\text{ and }\mu(t=T_2|m=T_2)=1
    \end{align*}
    This gives R's best responses.
    \vspace{-2pt}
    \begin{align*}
      a(m=T_1)=T_1\text{ and }a(m=T_2)=T_2
    \end{align*}
    Since no message yields the position \textit{Manager}, neither type can imitate the other to get this position, thus, no type has an incentive to deviate.\vspace{-6pt}
    \begin{itemize}
      \item[Step 2:] \textbf{Write up the separating PBE.}
    \end{itemize}
    \vfill\null
\end{frame}
\begin{frame}{PS12, Ex. 2.b (A): Farrell \& Rabin (1996). Find a separating PBE}
    \begin{table}
      \begin{tabular}{l|c|c|c|}
          \multicolumn{1}{c}{} & \multicolumn{1}{c}{Job 1 $(a=T_1)$} & \multicolumn{1}{c}{Job 2 $(a=T_2)$} & \multicolumn{1}{c}{Manager $(a=Manager)$} \\\cline{2-4}
           Task 1 efficient $(t=T_1)$ & 2, 5 & 1, -2 & 3, 3 \\\cline{2-4}
           Task 2 efficient $(t=T_2)$ & 1, -2 & 2, 5 & 3, 3 \\\cline{2-4}
      \end{tabular}
    \end{table}\vspace{-6pt}
    \begin{itemize}
      \item[(b)] Show that a separating equilibrium exists where Sally truthfully reveals which job is efficient, and the firm then places Sally in that specific job.
      \item[Step 1:] Go over the beliefs and actions in such a separating PBE. Does either type want to deviate?
    \end{itemize}\vspace{-6pt}
    In a PBE, the beliefs must correspond to the action of the senders.\\
    Thus in a separating PBE where $m(t=T_1)=T_1$ and $m(t=T_2)=T_2$, beliefs are\vspace{-2pt}
    \begin{align*}
      \mu(t=T_1|m=T_1)=1\text{ and }\mu(t=T_2|m=T_2)=1
    \end{align*}
    This gives R's best responses.
    \vspace{-2pt}
    \begin{align*}
      a(m=T_1)=T_1\text{ and }a(m=T_2)=T_2
    \end{align*}
    Since no message yields the position \textit{Manager}, neither type can imitate the other to get this position, thus, no type has an incentive to deviate.\vspace{-6pt}
    \begin{itemize}
      \item[Step 2:] Write up the separating PBE:
    \end{itemize}\vspace{-6pt}
    \begin{align*}
      \{\underbrace{(T_1,T_2)}_{m(t=T_1),m(t=T_2)},\underbrace{(T_1,T_2)}_{a(m=T_1),a(m=T_2)},\underbrace{\mu(T_1|T_1)=1}_{\mu(t=T_1|m=T_1)},\underbrace{\mu(T_2|T_2)=1}_{\mu(t=T_2|m=T_2)}\}
    \end{align*}
    \vfill\null
\end{frame}


\subsection{PS12, Ex. 2.c (A): Discuss if PBE is reasonable?}

\begin{frame}{PS12, Ex. 2.c (A): Farrell \& Rabin (1996). Discuss if PBE is reasonable?}
    \begin{table}
      \begin{tabular}{l|c|c|c|}
          \multicolumn{1}{c}{} & \multicolumn{1}{c}{Job 1 $(a=T_1)$} & \multicolumn{1}{c}{Job 2 $(a=T_2)$} & \multicolumn{1}{c}{Manager $(a=Manager)$} \\\cline{2-4}
           Task 1 efficient $(t=T_1)$ & 2, 5 & 1, -2 & 3, 3 \\\cline{2-4}
           Task 2 efficient $(t=T_2)$ & 1, -2 & 2, 5 & 3, 3 \\\cline{2-4}
      \end{tabular}
    \end{table}\vspace{-12pt}
    \begin{itemize}
      \item[(c)] Discuss whether or not this separating equilibrium seems reasonable. Why isn’t Sally able to convince the firm to give her the manager position?
    \end{itemize}\vspace{-6pt}
    \begin{multicols}{2}
      \vfill\null\columnbreak
      \vfill\null
    \end{multicols}
\end{frame}
\begin{frame}{PS12, Ex. 2.c (A): Farrell \& Rabin (1996). Discuss if PBE is reasonable?}
    \begin{table}
      \begin{tabular}{l|c|c|c|}
          \multicolumn{1}{c}{} & \multicolumn{1}{c}{Job 1 $(a=T_1)$} & \multicolumn{1}{c}{Job 2 $(a=T_2)$} & \multicolumn{1}{c}{Manager $(a=Manager)$} \\\cline{2-4}
           Task 1 efficient $(t=T_1)$ & 2, 5 & 1, -2 & 3, 3 \\\cline{2-4}
           Task 2 efficient $(t=T_2)$ & 1, -2 & 2, 5 & 3, 3 \\\cline{2-4}
      \end{tabular}
    \end{table}\vspace{-12pt}
    \begin{itemize}
      \item[(c)] Discuss whether or not this separating equilibrium seems reasonable. Why isn’t Sally able to convince the firm to give her the manager position?
      \item[Step 1:] \textbf{Why does Sally not have an incentive to deviate?}
    \end{itemize}
    \vfill\null
\end{frame}
\begin{frame}{PS12, Ex. 2.c (A): Farrell \& Rabin (1996). Discuss if PBE is reasonable?}
    \begin{table}
      \begin{tabular}{l|c|c|c|}
          \multicolumn{1}{c}{} & \multicolumn{1}{c}{Job 1 $(a=T_1)$} & \multicolumn{1}{c}{Job 2 $(a=T_2)$} & \multicolumn{1}{c}{Manager $(a=Manager)$} \\\cline{2-4}
           Task 1 efficient $(t=T_1)$ & 2, 5 & 1, -2 & 3, 3 \\\cline{2-4}
           Task 2 efficient $(t=T_2)$ & 1, -2 & 2, 5 & 3, 3 \\\cline{2-4}
      \end{tabular}
    \end{table}\vspace{-12pt}
    \begin{itemize}
      \item[(c)] Discuss whether or not this separating equilibrium seems reasonable. Why isn’t Sally able to convince the firm to give her the manager position?
      \item[Step 1:] Why does Sally not have an incentive to deviate?
    \end{itemize}\vspace{-6pt}
    Sally has to say that she is efficient at one of the jobs. Since Rayco believe she is telling the truth, she has no choice but to tell the truth, as she would otherwise end up in a worse job for her.
    \vfill\null
\end{frame}
\begin{frame}{PS12, Ex. 2.c (A): Farrell \& Rabin (1996). Discuss if PBE is reasonable?}
    \begin{table}
      \begin{tabular}{l|c|c|c|}
          \multicolumn{1}{c}{} & \multicolumn{1}{c}{Job 1 $(a=T_1)$} & \multicolumn{1}{c}{Job 2 $(a=T_2)$} & \multicolumn{1}{c}{Manager $(a=Manager)$} \\\cline{2-4}
           Task 1 efficient $(t=T_1)$ & 2, 5 & 1, -2 & 3, 3 \\\cline{2-4}
           Task 2 efficient $(t=T_2)$ & 1, -2 & 2, 5 & 3, 3 \\\cline{2-4}
      \end{tabular}
    \end{table}\vspace{-12pt}
    \begin{itemize}
      \item[(c)] Discuss whether or not this separating equilibrium seems reasonable. Why isn’t Sally able to convince the firm to give her the manager position?
      \item[Step 1:] Why does Sally not have an incentive to deviate?
    \end{itemize}\vspace{-6pt}
    Sally has to say that she is efficient at one of the jobs. Since Rayco believe she is telling the truth, she has no choice but to tell the truth, as she would otherwise end up in a worse job for her.\vspace{-6pt}
    \begin{itemize}
      \item[Step 2:] \textbf{Under which circumstances would a pooling PBE exist where both types would get hired as \textit{Manager}?}
    \end{itemize}
    \vfill\null
\end{frame}
\begin{frame}{PS12, Ex. 2.c (A): Farrell \& Rabin (1996). Discuss if PBE is reasonable?}
    \begin{table}
      \begin{tabular}{l|c|c|c|}
          \multicolumn{1}{c}{} & \multicolumn{1}{c}{Job 1 $(a=T_1)$} & \multicolumn{1}{c}{Job 2 $(a=T_2)$} & \multicolumn{1}{c}{Manager $(a=Manager)$} \\\cline{2-4}
           Task 1 efficient $(t=T_1)$ & 2, 5 & 1, -2 & 3, 3 \\\cline{2-4}
           Task 2 efficient $(t=T_2)$ & 1, -2 & 2, 5 & 3, 3 \\\cline{2-4}
      \end{tabular}
    \end{table}\vspace{-6pt}
    \begin{itemize}
      \item[(c)] Discuss whether or not this separating equilibrium seems reasonable. Why isn’t Sally able to convince the firm to give her the manager position?
      \item[Step 1:] Why does Sally not have an incentive to deviate?
    \end{itemize}\vspace{-6pt}
    Sally has to say that she is efficient at one of the jobs. Since Rayco believe she is telling the truth, she has no choice but to tell the truth, as she would otherwise end up in a worse job for her.\vspace{-6pt}
    \begin{itemize}
      \item[Step 2:] Under which circumstances would a pooling PBE exist where both types would get hired as \textit{Manager}?
    \end{itemize}\vspace{-6pt}
    There is a pooling equilibrium where all types of Sally choose the same message ($m=T_1$ or $m=T_2$), thus, don't send any informative signal on what type they are.\\
    In this case, if nature distributes types somewhat equally, i.e. $\mathbb{P}(t=T_1)=p\in\left[\frac{2}{7};\frac{5}{7}\right]$, then the best response for Rayco would be $a(m=T_1)=a(m=T_2)=Manager$ and no type of Sally would want to deviate.
    If nature is very likely to draw type $T_1$ $\left(p>\frac{5}{7}\right)$ then Rayco always offers $T_1$ as $\mathbb{E}[u_\text{R}(a=T_1)]>3$ no matter the signals.
    Likewise, if nature is unlikely to draw $T_1$ $\left(p<\frac{2}{7}\right)$ then $\mathbb{E}[u_\text{R}(a=T_2)]>3$.\\
    I.e. the separating PBE would be realistic if each type made a strategy in isolation. However, as Sally constitutes all types, she can decide on a tactic for all type of senders. No type would want to deviate from the pooling PBE $\left(\text{for }p\in\left[\frac{2}{7};\frac{5}{7}\right]\right)$.
    \vfill\null
\end{frame}
