\section{PS12, Ex. 4: Three-type job-applicant cheap talk game}

\begin{frame}{PS12, Ex. 4: Three-type job-applicant cheap talk game}
    Consider the following job-applicant cheap talk game based on \href{https://www.aeaweb.org/articles?id=10.1257/jep.10.3.103}{Farrell-Rabin (1996)}. Suppose that there are three types of potential applicants (high ability, medium ability, and low ability) and the firm can place the applicant in one of three possible positions (highly qualified, medium qualified, low qualified). The applicant is equally likely to be each of the three types (probability 1/3). Payoff are represented below, where for each cell, the left entry gives the payoff of the applicant, and the right entry gives the payoff of the firm, conditional on the firm’s action and the applicant’s type. Notice: this matrix does not show the normal form game! It merely gives you the payoffs for each type-job combination, but does not incorporate the cheap talk message.\\
    The game is as follows: first, the applicant’s type is realized: $t\in\{L,M,H\}$, where $t=L$ corresponds to low ability etc. The applicant observes his type and sends a cheap talk message $m\in\{L,M,H\}$. The firm observes the message and chooses a job for the applicant: $a\in\{L,M,H\}$, where $a=L$ corresponds to giving the applicant the low qualified job etc.\vspace{-14pt}
    \begin{table}
      \begin{tabular}{l|c|c|c|}
          \multicolumn{1}{c}{} & \multicolumn{1}{c}{Highly qualified} & \multicolumn{1}{c}{Medium qualified} & \multicolumn{1}{c}{Low qualified} \\\cline{2-4}
          High ability   & 3, 3 & 0, 0 & 0, 0 \\\cline{2-4}
          Medium ability & 1, 0 & 2, 2 & 0, 0 \\\cline{2-4}
          Low ability    & 1, 0 & 2, 0 & 1, 1 \\\cline{2-4}
      \end{tabular}
    \end{table}\vspace{-8pt}
    \begin{itemize}
      \item[(a)] Show that no fully separating PBE exist, where each type of applicant sends a different message. What is the intuition behind this result?
      \item[(b)] Show that a partial pooling PBE does exist, where $m(H)=H$ and $m(M)=m(L)=M$. What are the firm's beliefs? Solve for each case.
      \item[(c)] (If time permits) Does a fully pooling PBE exist, $m(H)=m(M)=m(L)=M$?
    \end{itemize}\vspace{-6pt}
    \vfill\null
\end{frame}



\subsection{PS12, Ex. 4.a: Fully separating PBE}

\begin{frame}{PS12, Ex. 4.a: Three-type: Fully separating PBE}
    \begin{table}
      \begin{tabular}{l|c|c|c|}
          \multicolumn{1}{c}{} & \multicolumn{1}{c}{Highly qualified} & \multicolumn{1}{c}{Medium qualified} & \multicolumn{1}{c}{Low qualified} \\\cline{2-4}
          High ability   & 3, 3 & 0, 0 & 0, 0 \\\cline{2-4}
          Medium ability & 1, 0 & 2, 2 & 0, 0 \\\cline{2-4}
          Low ability    & 1, 0 & 2, 0 & 1, 1 \\\cline{2-4}
      \end{tabular}
    \end{table}\vspace{-8pt}
    \begin{itemize}
      \item[(a)] Show that no fully separating PBE exist, where each type of applicant sends a different message. What is the intuition behind this result?
    \end{itemize}\vspace{-6pt}
    \begin{multicols}{2}
      \vfill\null\columnbreak
      \vfill\null
    \end{multicols}
\end{frame}

\begin{frame}{PS12, Ex. 4.a: Three-type: Fully separating PBE}
    \begin{itemize}
      \item[(a)] Show that no fully separating PBE exist, where each type of applicant sends a different message. What is the intuition behind this result?
    \end{itemize}\vspace{-6pt}
    The firm prefers to give each candidate the job corresponding to their type, and thus want a separating PBE. So in order to show that no fully separating PBE exist, look for a type who would like to get a job that does not correspond to one's type, and go from there.\\
    Type $H$ and $M$ prefer their corresponding jobs and will send the corresponding message to let themselves be identified. However, type $L$ would like to get a type $M$ job. Since messaging has no cost, in a separating equilibrium he would have an incentive to deviate and use the message $M$.\\
    To show this formally, we look at the PBE where the applicant sending the message $H$ will be given the job $H$ and so forth. (Each message could be paired with any job, as long as each applicant type sends a different message and are correctly identified, but the argument stays the same).\\
    Since $u_s(m=L)<max[u_s(m=M),u_s(m=H)]$ for all types, no separating PBE can exist as no type will use message $L$.
\end{frame}


\subsection{PS12, Ex. 4.b: Partial pooling PBE}

\begin{frame}{PS12, Ex. 4.b: Three-type: Partial pooling PBE}
    \begin{table}
      \begin{tabular}{l|c|c|c|}
          \multicolumn{1}{c}{} & \multicolumn{1}{c}{Highly qualified} & \multicolumn{1}{c}{Medium qualified} & \multicolumn{1}{c}{Low qualified} \\\cline{2-4}
          High ability   & 3, 3 & 0, 0 & 0, 0 \\\cline{2-4}
          Medium ability & 1, 0 & 2, 2 & 0, 0 \\\cline{2-4}
          Low ability    & 1, 0 & 2, 0 & 1, 1 \\\cline{2-4}
      \end{tabular}
    \end{table}\vspace{-8pt}
    \begin{itemize}
      \item[(b)] Show that a partial pooling PBE does exist, where the high-ability applicant sends the message $m = H$, and the other two types send the message $m = M$. What are the firm’s beliefs about the applicant if he receives the message $m = H$ or $m = M$ (on the equilibrium path), or if he receives the message $m = L$ (off the equilibrium path)? In each case, solve for the firm’s optimal action given its beliefs.
    \end{itemize}\vspace{-6pt}
    \begin{multicols}{2}
      \vfill\null\columnbreak
      \vfill\null
    \end{multicols}
\end{frame}

\begin{frame}{PS12, Ex. 4.b: Three-type: Partial pooling PBE}
    \begin{itemize}
      \item[(b)] Show that a partial pooling PBE does exist, where the high-ability applicant sends the message $m = H$, and the other two types send the message $m = M$.
    \end{itemize}\vspace{-6pt}
    \begin{itemize}
      \item[Step 1:] On the equilibrium path, find R's beliefs when receiving the messages $m=H,M$ and the corresponding best responses for the receiver.
    \end{itemize}\vspace{-4pt}
    In a PBE, the receiver needs to have beliefs corresponding to what happens in the game. This yields the following beliefs and best responses on the equilibrium path:
    \begin{align*}
        \mu(t=H|m=H)=1;\ 
        \mu(t=M|m=M)=\frac{1}{2};\ 
        \mu(t=L|m=M)=\frac{1}{2}
    \end{align*}\vspace{-18pt}
    \begin{align*}
        a^*(m=H)=H;\ a^*(m=M)=M
    \end{align*}\vspace{-18pt}
    \begin{itemize}
      \item[Step 2:] Off the equilibrium path, find beliefs for $m=L$ and the corresponding best responses for R, which will uphold the PBE, i.e. such that S will not deviate.
    \end{itemize}\vspace{-6pt}
    We covered in (a) that no sender wants to get $a=L$, so find the beliefs which allows for $a(m=L)=L$.\\
    Consider $\mu(t=L|m=L)=1$ where R believes that if someone is to deviate, he believes it's a low type and the best response is $a(m=L)=L$.\vspace{-2pt}
    \begin{itemize}
      \item[Step 3:] Write up this partially separating PBE:
    \end{itemize}\vspace{-10pt}
    \begin{align*}
    \{(H,M,M),(H,M,L),\underbrace{\mu(t=H)=1}_{m=H},\underbrace{\mu(t=M)=\mu(t=L)=\frac{1}{2}}_{m=M},\underbrace{\mu(t=L|L)=1\}}_{m=L}
    \end{align*}
      \vfill\null
\end{frame}


\subsection{PS12, Ex. 4.c: Fully pooling PBE}

\begin{frame}{PS12, Ex. 4.c: Three-type: Fully pooling PBE}
    \begin{table}
      \begin{tabular}{l|c|c|c|}
          \multicolumn{1}{c}{} & \multicolumn{1}{c}{Highly qualified} & \multicolumn{1}{c}{Medium qualified} & \multicolumn{1}{c}{Low qualified} \\\cline{2-4}
          High ability   & 3, 3 & 0, 0 & 0, 0 \\\cline{2-4}
          Medium ability & 1, 0 & 2, 2 & 0, 0 \\\cline{2-4}
          Low ability    & 1, 0 & 2, 0 & 1, 1 \\\cline{2-4}
      \end{tabular}
    \end{table}\vspace{-8pt}
    \begin{itemize}
      \item[(c)] (If time permits) Does a fully pooling PBE exist where all types send the message $m = H$? If so, describe the players’ equilibrium strategies and beliefs, and discuss whether this pooling PBE looks more or less reasonable than the partial pooling PBE from (b).
    \end{itemize}\vspace{-6pt}
    \begin{multicols}{2}
      \vfill\null\columnbreak
      \vfill\null
    \end{multicols}
\end{frame}

\begin{frame}{PS12, Ex. 4.c: Three-type: Fully pooling PBE}
    \begin{itemize}
      \item[(c)] (If time permits) Does a fully pooling PBE exist where all types send the message $m = H$? If so, describe the players’ equilibrium strategies and beliefs, and discuss whether this pooling PBE looks more or less reasonable than the partial pooling PBE from (b).
      \item[Step 1:] Suggest a fully pooling PBE:
    \end{itemize}\vspace{-10pt}
    \begin{align*}
    \{(M,M,M),(L,M,L),\mu(t=L)=1,\mu(t=H)=\mu(t=M)=\mu(t=L)=\frac{1}{3},\mu(t=L)=1\}
    \end{align*}\vspace{-14pt}
    \begin{itemize}
        \item[Step 2:] For the above PBE, explain the messages being send and the responding actions:
    \end{itemize}\vspace{-2pt}
    In this PBE, every type of applicants sends the signal $m=M$. Off the equilibrium path, the receiver believes that anyone who plays $m=H$ or $m=L$ will be of type $L$.\vspace{-2pt}
    \begin{itemize}
        \item[Step 3:] Does this PBE seem realistic?
    \end{itemize}\vspace{-2pt}
    No, the only type who would want something other than $a=M$ is type $H$, thus, the receiver should believe that anyone who deviates is type $H$, and therefore, offer $a(H)=H$ in response which would be a strict Pareto improvement.
\end{frame}

