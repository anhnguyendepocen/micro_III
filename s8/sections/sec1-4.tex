\section{PS8, Ex. 1 (A): Trigger strategy in infinitely repeated game:}

\begin{frame}{PS8, Ex. 1 (A): Trigger strategy in infinitely repeated game:}
    \vspace{-10pt}
    Consider the following game G:
    \begin{table}
      \begin{tabular}{cl|c|c|c|}
        & \multicolumn{1}{c}{} & \multicolumn{3}{c}{Player 2}\\
        \parbox[t]{1mm}{\multirow{4}{*}{\rotatebox[origin=r]{90}{Player 1}}}
        & \multicolumn{1}{c}{} & \multicolumn{1}{c}{X} & \multicolumn{1}{c}{Y} & \multicolumn{1}{c}{Z}\\\cline{3-5}
        & A   & 6, 6 &  0, 8 &  0, 0  \\\cline{3-5}
        & B & 7, 1  & 2, 2 &  1, 1  \\\cline{3-5}
        & C & 0, 0  & 1, 1 & 4, 5  \\\cline{3-5}
      \end{tabular}
    \end{table}
    \begin{itemize}
        \item[] Suppose that G is repeated infinitely many times, so that we have G(1, $\infty$). Define trigger strategies such that the outcome of all stages is (A,X). Find the smallest value of $\delta$ such that these strategies constitute a SPNE.
    \end{itemize}
    \vfill\null
\end{frame}

\begin{frame}{PS8, Ex. 1 (A): Trigger strategy in infinitely repeated game - exam answer)}
    \begin{itemize}
        %\item The following is how I personally would write my answer at the exam, it cannot be guaranteed that writing your answer in this way would yield full points.
        \item[] Suppose that G is repeated infinitely many times, so that we have G(1, $\infty$). Define trigger strategies such that the outcome of all stages is (A,X). Find the smallest value of $\delta$ such that these strategies constitute a SPNE.
    \end{itemize}
    \vspace{-6pt}
  \begin{multicols}{2}
  Trigger strategies such that the outcome of all stages of the game is (A,X) are possible using respectively B,Y or C,Z as the threats. Since the threats B,Y will make the SPNE possible for the smallest $\delta$, I will use B,Y in the trigger strategies i define: \\
    \begin{enumerate}
    \item Trigger strategy $P1$: In the \nth{1} turn, play A. In every subsequent turn, if outcome from every previous turn was (A,X), play A, otherwise play B.
    \item Trigger strategy $P2$: In the \nth{1} turn, play X. In every subsequent turn, if outcome from every previous turn was (A,X), play X, otherwise play Y.
    \end{enumerate}
    \vfill\null\columnbreak
    Player 2 has the highest incentive to deviate, so I only examine player 2's incentive to deviate. In order to find the lowest $\delta$ to secure cooperation, I set up the inequality for which the payoff for cooperation is higher than the payoff for deviating:
    \begin{align*}
        \frac{6}{1-\delta} &\geq 8 + \frac{2\delta}{1-\delta} \Leftrightarrow\\
        6 &\geq 8 - 8\delta + 2\delta \Leftrightarrow\\
        \delta &\geq \frac{1}{3}
    \end{align*}
    $\delta=\frac{1}{3}$ is the smallest value for which the strategies constitute a SPNE.
    \vfill\null
  \end{multicols}
\end{frame}

\section{PS8, Ex. 2 (A): Hit-and-run cab (Bayes' Rule)}

\begin{frame}{PS8, Ex. 2 (A): Hit-and-run cab (Bayes' Rule)}
    \vspace{-10pt}
    Review the intuition from the ‘Doctor’ example in lecture 7 (slides 6-9), and then use Bayes’ rule to solve the following problem:
    \begin{itemize}
        \item[] \textit{A cab was involved in a hit and run accident at night. $85\%$ of the cabs in the city are Green and $15\%$ are Blue. A witness later recalls that the cab was Blue, and we know that this witness’ memory is reliable $80\%$ of the time. Given the statement from the witness, calculate the probability that the cab involved in the accident was actually Blue.}
    \end{itemize}
    \textit{\textbf{First, try to write up Bayes' Rule on your own} (it is written on the next slide)}
    \vfill\null
\end{frame}
\begin{frame}{PS8, Ex. 2 (A): Hit-and-run cab (Bayes' Rule)}
    \vspace{-10pt}
    Review the intuition from the ‘Doctor’ example in lecture 7 (slides 6-9), and then use Bayes’ rule to solve the following problem:
    \begin{itemize}
        \item[] \textit{A cab was involved in a hit and run accident at night. $85\%$ of the cabs in the city are Green and $15\%$ are Blue. A witness later recalls that the cab was Blue, and we know that this witness’ memory is reliable $80\%$ of the time. Given the statement from the witness, calculate the probability that the cab involved in the accident was actually Blue.}
    \end{itemize}
    Bayes' rule:
    \begin{align*}
        P(A|B)=\frac{P(B|A)P(A)}{P(B)}
    \end{align*}
    \vfill\null
\end{frame}

\begin{frame}{PS8, Ex. 2 (A): Hit-and-run cab (Bayes' Rule)}
    Information so far:
    \begin{enumerate}
        \item P(B): The unconditional chance that a cab is green: $\frac{85}{100}$
        \item P(B): The unconditional chance that a cab is blue: $\frac{15}{100}$
        \item P(obs B\textbar B): The chance of remembering a blue cab, given it was blue: $\frac{80}{100}$
        \item P(obs B\textbar G): The chance of remembering a blue cab, given it was green: $\frac{20}{100}$
    \end{enumerate}
    P(obs B): The unconditional chance that the witness says the cab is blue, so the chance the witness would observe a blue cab and remember it as blue, plus the chance the witness would observe a green cab and remember it as blue:
    \vfill\null
\end{frame}

\begin{frame}{PS8, Ex. 2 (A): Hit-and-run cab (Bayes' Rule)}
    Information so far:
    \begin{enumerate}
        \item P(B): The unconditional chance that a cab is green: $\frac{85}{100}$
        \item P(B): The unconditional chance that a cab is blue: $\frac{15}{100}$
        \item P(obs B\textbar B): The chance of remembering a blue cab, given it was blue: $\frac{80}{100}$
        \item P(obs B\textbar G): The chance of remembering a blue cab, given it was green: $\frac{20}{100}$
    \end{enumerate}
    P(obs B): The unconditional chance that the witness says the cab is blue, so the chance the witness would observe a blue cab and remember it as blue, plus the chance the witness would observe a green cab and remember it as blue:
    \begin{align*}
        P(obs\ B)=P(obs\ B|B)\cdot P(B)+P(obs\ B|G)\cdot P(G)=\frac{80}{100}\cdot\frac{15}{100}+\frac{20}{100}\cdot\frac{85}{100}=\frac{29}{100}
    \end{align*}
    We want to find the chance that the cab is blue, given that the witness says it's blue. Using Bayes' rule, this is the same as the odds that the cab will be blue and the witness says it's blue, divided by the unconditional chance the witness says it blue.
    \vfill\null
\end{frame}

\begin{frame}{PS8, Ex. 2 (A): Hit-and-run cab (Bayes' Rule)}
    Information so far:
    \begin{enumerate}
        \item P(B): The unconditional chance that a cab is green: $\frac{85}{100}$
        \item P(B): The unconditional chance that a cab is blue: $\frac{15}{100}$
        \item P(obs B\textbar B): The chance of remembering a blue cab, given it was blue: $\frac{80}{100}$
        \item P(obs B\textbar G): The chance of remembering a blue cab, given it was green: $\frac{20}{100}$
    \end{enumerate}
    P(obs B): The unconditional chance that the witness says the cab is blue, so the chance the witness would observe a blue cab and remember it as blue, plus the chance the witness would observe a green cab and remember it as blue:
    \begin{align*}
        P(obs\ B)=P(obs\ B|B)\cdot P(B)+P(obs\ B|G)\cdot P(G)=\frac{80}{100}\cdot\frac{15}{100}+\frac{20}{100}\cdot\frac{85}{100}=\frac{29}{100}
    \end{align*}
    We want to find the chance that the cab is blue, given that the witness says it's blue. Using Bayes' rule, this is the same as the odds that the cab will be blue and the witness says it's blue, divided by the unconditional chance the witness says it blue.
    \begin{align*}
        P(B|obs\ B)=\frac{P(obs\ B|B)\cdot P(B)}{P(obs\ B)}=\frac{\frac{80}{100}*\frac{15}{100}}{\frac{29}{100}}=0.414
    \end{align*}
    \vfill\null
\end{frame}



\section{PS8, Ex. 3 (A): Static Bayesian game (Bayesian Nash Equilibria)}

\begin{frame}{PS8, Ex. 3 (A): Static Bayesian game (Bayesian Nash Equilibria)}
    Consider the following static game, where $a$ is a real number:
    \begin{table}
      \begin{tabular}{l|c|c|}
        \multicolumn{1}{c}{} & \multicolumn{2}{c}{} \\
        \multicolumn{1}{c}{} & \multicolumn{1}{c}{L} & \multicolumn{1}{c}{R} \\\cline{2-3}
        U & 2, 1 & 0, $a$ \\\cline{2-3}
        D & 0, 1 & 1, $a$ \\\cline{2-3}
      \end{tabular}
    \end{table}
    \begin{itemize}
        \item[(a)] Suppose that $a=2$. Does any player have a dominant strategy? What about when $a=-2$?
        \item[(b)] Now assume that player 2 knows the value of $a$, but player 1 only knows that $a=2$ with probability $0.5$ and $a=-2$ with probability $0.5$. Explain how this situation can be modeled as a Bayesian game, describing the players, their action spaces, type spaces, beliefs and payoff functions.
        \item[(c)] Find the Bayes-Nash equilibrium of the game described in (b).
    \end{itemize}
    \vfill\null
\end{frame}

\begin{frame}{PS8, Ex. 3.a (A): Static Bayesian game (Bayesian Nash Equilibria)}
    \textit{(a) Suppose that $a = 2$. Does any player have a dominant strategy? What about when $a = -2$?}
    \vspace{-8pt}
    \begin{multicols}{2}
      \begin{itemize}
      \item[(a)] The value of $a$ affects P2's payoff:
      \end{itemize}
      \vfill\null\columnbreak
      \begin{table}
          \begin{tabular}{l|c|c|}
          \multicolumn{1}{c}{} & \multicolumn{2}{c}{$a=2:$} \\
          \multicolumn{1}{c}{} & \multicolumn{1}{c}{L} & \multicolumn{1}{c}{R} \\\cline{2-3}
          U & 2, 1 & 0, 2 \\\cline{2-3}
          D & 0, 1 & 1, 2 \\\cline{2-3}
        \end{tabular}
      \end{table}
      \vspace{-8pt}
      \begin{table}
        \begin{tabular}{l|c|c|}
          \multicolumn{1}{c}{} & \multicolumn{2}{c}{$a=-2:$} \\
          \multicolumn{1}{c}{} & \multicolumn{1}{c}{L} & \multicolumn{1}{c}{R} \\\cline{2-3}
          U & 2, 1 & 0, -2 \\\cline{2-3}
          D & 0, 1 & 1, -2 \\\cline{2-3}
        \end{tabular}
      \end{table}
      \vfill\null
    \end{multicols}
\end{frame}
\begin{frame}{PS8, Ex. 3.a (A): Static Bayesian game (Bayesian Nash Equilibria)}
    \textit{(a) Suppose that $a = 2$. Does any player have a dominant strategy? What about when $a = -2$?}
    \vspace{-8pt}
    \begin{multicols}{2}
      \begin{itemize}
      \item[(a)] The value of $a$ not only affects P2's payoff, but also P2's strategy. For $a=2$, P2 will have $R$ as a dominant strategy; and for $a=-2$, P2 will have $L$ as a dominant strategy.
      \end{itemize}
      \vfill\null\columnbreak
      \begin{table}
          \begin{tabular}{l|c|c|}
          \multicolumn{1}{c}{} & \multicolumn{2}{c}{$a=2:$} \\
          \multicolumn{1}{c}{} & \multicolumn{1}{c}{L} & \multicolumn{1}{c}{\color{blue}R} \\\cline{2-3}
          U & \textcolor{red}{2}, 1 & 0, \textcolor{blue}{2} \\\cline{2-3}
          D & 0, 1 & \textcolor{red}{1}, \textcolor{blue}{2} \\\cline{2-3}
        \end{tabular}
      \end{table}
      \vspace{-8pt}
      \begin{table}
        \begin{tabular}{l|c|c|}
          \multicolumn{1}{c}{} & \multicolumn{2}{c}{$a=-2:$} \\
          \multicolumn{1}{c}{} & \multicolumn{1}{c}{\color{blue}L} & \multicolumn{1}{c}{R} \\\cline{2-3}
          U & \textcolor{red}{2}, \textcolor{blue}{1} & 0, -2 \\\cline{2-3}
          D & 0, \textcolor{blue}{1} & \textcolor{red}{1}, -2 \\\cline{2-3}
        \end{tabular}
      \end{table}
      \vfill\null
    \end{multicols}
\end{frame}


\begin{frame}{PS8, Ex. 3.b (A): Static Bayesian game (Bayesian Nash Equilibria)}
    \textit{(b) Now assume that player 2 knows the value of $a$, but player 1 only knows that $a=2$ with probability $0.5$ and $a=-2$ with probability $0.5$. Explain how this situation can be modeled as a Bayesian game, describing the players, their action spaces, type spaces, beliefs and payoff functions.}
    \vspace{-8pt}
    \begin{multicols}{2}
      \begin{itemize}
      \item[(a)] The value of $a$ not only affects P2's payoff, but also P2's strategy. For $a=2$, P2 will have $R$ as a dominant strategy; and for $a=-2$, P2 will have $L$ as a dominant strategy.
      \end{itemize}
      \vfill\null\columnbreak
      \begin{table}
          \begin{tabular}{l|c|c|}
          \multicolumn{1}{c}{} & \multicolumn{2}{c}{$a=2:$} \\
          \multicolumn{1}{c}{} & \multicolumn{1}{c}{L} & \multicolumn{1}{c}{\color{blue}R} \\\cline{2-3}
          U & \textcolor{red}{2}, 1 & 0, \textcolor{blue}{2} \\\cline{2-3}
          D & 0, 1 & \textcolor{red}{1}, \textcolor{blue}{2} \\\cline{2-3}
        \end{tabular}
      \end{table}
      \vspace{-8pt}
      \begin{table}
        \begin{tabular}{l|c|c|}
          \multicolumn{1}{c}{} & \multicolumn{2}{c}{$a=-2:$} \\
          \multicolumn{1}{c}{} & \multicolumn{1}{c}{\color{blue}L} & \multicolumn{1}{c}{R} \\\cline{2-3}
          U & \textcolor{red}{2}, \textcolor{blue}{1} & 0, -2 \\\cline{2-3}
          D & 0, \textcolor{blue}{1} & \textcolor{red}{1}, -2 \\\cline{2-3}
        \end{tabular}
      \end{table}
      \vfill\null
    \end{multicols}
\end{frame}
\begin{frame}{PS8, Ex. 3.b (A): Static Bayesian game (Bayesian Nash Equilibria)}
    \textit{(b) Now assume that player 2 knows the value of $a$, but player 1 only knows that $a=2$ with probability $0.5$ and $a=-2$ with probability $0.5$. Explain how this situation can be modeled as a Bayesian game, describing the players, their action spaces, type spaces, beliefs and payoff functions.}
    \vspace{-8pt}
    \begin{multicols}{2}
      \begin{itemize}
      \item[(a)] The value of $a$ not only affects P2's payoff, but also P2's strategy. For $a=2$, P2 will have $R$ as a dominant strategy; and for $a=-2$, P2 will have $L$ as a dominant strategy.
      \item[(b)] This can be modelled as a Bayesian game since P2 has two types (where he strongly prefers $L$ and $R$ respectively) and P1 has a belief about the distribution of these types (each happen $\frac{1}{2}$ of the time.).
      \end{itemize}
      \vfill\null\columnbreak
      \begin{table}
          \begin{tabular}{l|c|c|}
          \multicolumn{1}{c}{} & \multicolumn{2}{c}{$a=2:$} \\
          \multicolumn{1}{c}{} & \multicolumn{1}{c}{L} & \multicolumn{1}{c}{\color{blue}R} \\\cline{2-3}
          U & \textcolor{red}{2}, 1 & 0, \textcolor{blue}{2} \\\cline{2-3}
          D & 0, 1 & \textcolor{red}{1}, \textcolor{blue}{2} \\\cline{2-3}
        \end{tabular}
      \end{table}
      \vspace{-8pt}
      \begin{table}
        \begin{tabular}{l|c|c|}
          \multicolumn{1}{c}{} & \multicolumn{2}{c}{$a=-2:$} \\
          \multicolumn{1}{c}{} & \multicolumn{1}{c}{\color{blue}L} & \multicolumn{1}{c}{R} \\\cline{2-3}
          U & \textcolor{red}{2}, \textcolor{blue}{1} & 0, -2 \\\cline{2-3}
          D & 0, \textcolor{blue}{1} & \textcolor{red}{1}, -2 \\\cline{2-3}
        \end{tabular}
      \end{table}
      \vfill\null
    \end{multicols}
\end{frame}
\begin{frame}{PS8, Ex. 3.b (A): Static Bayesian game (Bayesian Nash Equilibria)}
    \textit{(b) Now assume that player 2 knows the value of $a$, but player 1 only knows that $a=2$ with probability $0.5$ and $a=-2$ with probability $0.5$. Explain how this situation can be modeled as a Bayesian game, describing the players, their action spaces, type spaces, beliefs and payoff functions.}
    \vspace{-8pt}
    \begin{multicols}{2}
      \begin{itemize}
      \item[(a)] The value of $a$ not only affects P2's payoff, but also P2's strategy. For $a=2$, P2 will have $R$ as a dominant strategy; and for $a=-2$, P2 will have $L$ as a dominant strategy.
      \item[(b)] This can be modelled as a Bayesian game since P2 has two types (where he strongly prefers $L$ and $R$ respectively) and P1 has a belief about the distribution of these types (each happen $\frac{1}{2}$ of the time.).
          \item[Players:] P1, P2
          \item[Action sp.:] $A_1=(U,D),A_2=(L,R)$
          \item[Type space:] $T_1=(t)$ [one type], $T_2=(t_1:a=2,t_2:a=-2)$
          \item[Beliefs:] $\mathbb{P}_1(a=2)=\mathbb{P}_1(a=-2)=\frac{1}{2}$
      \end{itemize}
      \vfill\null\columnbreak
      \begin{table}
          \begin{tabular}{l|c|c|}
          \multicolumn{1}{c}{} & \multicolumn{2}{c}{$a=2:$} \\
          \multicolumn{1}{c}{} & \multicolumn{1}{c}{L} & \multicolumn{1}{c}{\color{blue}R} \\\cline{2-3}
          U & \textcolor{red}{2}, 1 & 0, \textcolor{blue}{2} \\\cline{2-3}
          D & 0, 1 & \textcolor{red}{1}, \textcolor{blue}{2} \\\cline{2-3}
        \end{tabular}
      \end{table}
      \vspace{-8pt}
      \begin{table}
        \begin{tabular}{l|c|c|}
          \multicolumn{1}{c}{} & \multicolumn{2}{c}{$a=-2:$} \\
          \multicolumn{1}{c}{} & \multicolumn{1}{c}{\color{blue}L} & \multicolumn{1}{c}{R} \\\cline{2-3}
          U & \textcolor{red}{2}, \textcolor{blue}{1} & 0, -2 \\\cline{2-3}
          D & 0, \textcolor{blue}{1} & \textcolor{red}{1}, -2 \\\cline{2-3}
        \end{tabular}
      \end{table}
      \textbf{\textit{Write as type-dependent payoff matrices}}
      \vfill\null
    \end{multicols}
\end{frame}
\begin{frame}{PS8, Ex. 3.b (A): Static Bayesian game (Bayesian Nash Equilibria)}
    \textit{(b) Now assume that player 2 knows the value of $a$, but player 1 only knows that $a=2$ with probability $0.5$ and $a=-2$ with probability $0.5$. Explain how this situation can be modeled as a Bayesian game, describing the players, their action spaces, type spaces, beliefs and payoff functions.}
    \vspace{-8pt}
    \begin{multicols}{2}
      \begin{itemize}
      \item[(a)] The value of $a$ not only affects P2's payoff, but also P2's strategy. For $a=2$, P2 will have $R$ as a dominant strategy; and for $a=-2$, P2 will have $L$ as a dominant strategy.
      \item[(b)] This can be modelled as a Bayesian game since P2 has two types (where he strongly prefers $L$ and $R$ respectively) and P1 has a belief about the distribution of these types (each happen $\frac{1}{2}$ of the time.).
          \item[Players:] P1, P2
          \item[Action sp.:] $A_1=(U,D),A_2=(L,R)$
          \item[Type space:] $T_1=(t)$ [one type], $T_2=(t_1:a=2,t_2:a=-2)$
          \item[Beliefs:] $\mathbb{P}_1(a=2)=\mathbb{P}_1(a=-2)=\frac{1}{2}$
      \end{itemize}
      \vfill\null\columnbreak
      Type-dependent payoff matrices:
      \vspace{-8pt}
      \begin{table}
          \begin{tabular}{l|c|c|}
          \multicolumn{1}{c}{} & \multicolumn{2}{c}{Type $t_1:a=2\ (p=\frac{1}{2})$} \\
          \multicolumn{1}{c}{} & \multicolumn{1}{c}{L} & \multicolumn{1}{c}{R} \\\cline{2-3}
          U & 2, 1 & 0, 2 \\\cline{2-3}
          D & 0, 1 & 1, 2 \\\cline{2-3}
        \end{tabular}
      \end{table}
      \vspace{-8pt}
      \begin{table}
        \begin{tabular}{l|c|c|}
          \multicolumn{1}{c}{} & \multicolumn{2}{c}{Type $t_2:a=-2\ (p=\frac{1}{2})$} \\
          \multicolumn{1}{c}{} & \multicolumn{1}{c}{L} & \multicolumn{1}{c}{R} \\\cline{2-3}
          U & 2, 1 & 0, -2 \\\cline{2-3}
          D & 0, 1 & 1, -2 \\\cline{2-3}
        \end{tabular}
      \end{table}
      \vfill\null
    \end{multicols}
\end{frame}


\begin{frame}{PS8, Ex. 3.c (A): Static Bayesian game (Bayesian Nash Equilibria)}
    \textit{(c) Find the Bayes-Nash equilibrium of the game described in (b).}
    \vspace{-8pt}
    \begin{multicols}{2}
      \begin{itemize}
      \item[(a)] The value of $a$ not only affects P2's payoff, but also P2's strategy. For $a=2$, P2 will have $R$ as a dominant strategy; and for $a=-2$, P2 will have $L$ as a dominant strategy.
      \item[(b)] This can be modelled as a Bayesian game since P2 has two types (where he strongly prefers $L$ and $R$ respectively) and P1 has a belief about the distribution of these types (each happen $\frac{1}{2}$ of the time.).
          \item[Players:] P1, P2
          \item[Action sp.:] $A_1=(U,D),A_2=(L,R)$
          \item[Type space:] $T_1=(t)$ [one type], $T_2=(t_1:a=2,t_2:a=-2)$
          \item[Beliefs:] $\mathbb{P}_1(a=2)=\mathbb{P}_1(a=-2)=\frac{1}{2}$
      \end{itemize}
      \vfill\null\columnbreak
      Type-dependent payoff matrices:
      \vspace{-8pt}
      \begin{table}
          \begin{tabular}{l|c|c|}
          \multicolumn{1}{c}{} & \multicolumn{2}{c}{Type $t_1:a=2\ (p=\frac{1}{2})$} \\
          \multicolumn{1}{c}{} & \multicolumn{1}{c}{L} & \multicolumn{1}{c}{R} \\\cline{2-3}
          U & 2, 1 & 0, 2 \\\cline{2-3}
          D & 0, 1 & 1, 2 \\\cline{2-3}
        \end{tabular}
      \end{table}
      \vspace{-8pt}
      \begin{table}
        \begin{tabular}{l|c|c|}
          \multicolumn{1}{c}{} & \multicolumn{2}{c}{Type $t_2:a=-2\ (p=\frac{1}{2})$} \\
          \multicolumn{1}{c}{} & \multicolumn{1}{c}{L} & \multicolumn{1}{c}{R} \\\cline{2-3}
          U & 2, 1 & 0, -2 \\\cline{2-3}
          D & 0, 1 & 1, -2 \\\cline{2-3}
        \end{tabular}
      \end{table}
      \vfill\null
    \end{multicols}
\end{frame}
\begin{frame}{PS8, Ex. 3.c (A): Static Bayesian game (Bayesian Nash Equilibria)}
    \textit{(c) Find the Bayes-Nash equilibrium of the game described in (b).}
    \vspace{-8pt}
    \begin{multicols}{2}
      \begin{itemize}
      \item[(a)] The value of $a$ not only affects P2's payoff, but also P2's strategy. For $a=2$, P2 will have $R$ as a dominant strategy; and for $a=-2$, P2 will have $L$ as a dominant strategy.
      \item[(b)] This can be modelled as a Bayesian game since P2 has two types (he either has L or R as a dominant strategy) and P1 has a belief about the distribution of these types (Each happen $\frac{1}{2}$ the time.).
          \item[Players:] P1, P2
          \item[Action sp.:] $A_1=(U,D),A_2=(L,R)$
          \item[Type space:] $T_1=(t)$ [one type], $T_2=(t_1:a=2,t_2:a=-2)$
          \item[Beliefs:] $\mathbb{P}_1(a=2)=\mathbb{P}_1(a=-2)=\frac{1}{2}$
          \vfill\null\columnbreak
      \end{itemize}
      Type-dependent payoff matrices:
      \vspace{-8pt}
      \begin{table}
          \begin{tabular}{l|c|c|}
          \multicolumn{1}{c}{} & \multicolumn{2}{c}{Type $t_1:a=2\ (p=\frac{1}{2})$} \\
          \multicolumn{1}{c}{} & \multicolumn{1}{c}{L} & \multicolumn{1}{c}{R} \\\cline{2-3}
          U & 2, 1 & 0, 2 \\\cline{2-3}
          D & 0, 1 & 1, 2 \\\cline{2-3}
        \end{tabular}
      \end{table}
      \vspace{-8pt}
      \begin{table}
        \begin{tabular}{l|c|c|}
          \multicolumn{1}{c}{} & \multicolumn{2}{c}{Type $t_2:a=-2\ (p=\frac{1}{2})$} \\
          \multicolumn{1}{c}{} & \multicolumn{1}{c}{L} & \multicolumn{1}{c}{R} \\\cline{2-3}
          U & 2, 1 & 0, -2 \\\cline{2-3}
          D & 0, 1 & 1, -2 \\\cline{2-3}
        \end{tabular}
      \end{table}
      \vspace{-8pt}
      \begin{itemize}
          \item[(c)] The strategy for P1 is just one action, whereas the strategy for P2 is an action for each of the two types. Using this, we can write up the expected payoff matrix:
      \end{itemize}
      \vspace{-12pt}
      \begin{table}
        \begin{tabular}{l|c|c|c|c|}
          \multicolumn{1}{c}{} & \multicolumn{1}{c}{LL} & \multicolumn{1}{c}{LR} & \multicolumn{1}{c}{RL} & \multicolumn{1}{c}{RR} \\\cline{2-5}
          U & 2, 1 &  1,-$\frac{1}{2}$ & 1, $\frac{3}{2}$ & 0, 0  \\\cline{2-5}
          D & 0, 1  & $\frac{1}{2}$, -$\frac{1}{2}$ & $\frac{1}{2}$, $\frac{3}{2}$ & 1, 0  \\\cline{2-5}
        \end{tabular}
      \end{table}
      \vspace{-4pt}
      \textbf{\textit{Find the Bayesian Nash Equilibria.}}
      \vfill\null
    \end{multicols}
\end{frame}
\begin{frame}{PS8, Ex. 3.c (A): Static Bayesian game (Bayesian Nash Equilibria)}
    \textit{(c) Find the Bayes-Nash equilibrium of the game described in (b).}
    \vspace{-8pt}
    \begin{multicols}{2}
      \begin{itemize}
      \item[(a)] The value of $a$ not only affects P2's payoff, but also P2's strategy. For $a=2$, P2 will have $R$ as a dominant strategy; and for $a=-2$, P2 will have $L$ as a dominant strategy.
      \item[(b)] This can be modelled as a Bayesian game since P2 has two types (he either has L or R as a dominant strategy) and P1 has a belief about the distribution of these types (Each happen $\frac{1}{2}$ the time.).
          \item[Players:] P1, P2
          \item[Action sp.:] $A_1=(U,D),A_2=(L,R)$
          \item[Type space:] $T_1=(t)$ [one type], $T_2=(t_1:a=2,t_2:a=-2)$
          \item[Beliefs:] $\mathbb{P}_1(a=2)=\mathbb{P}_1(a=-2)=\frac{1}{2}$
          \vfill\null\columnbreak
      \end{itemize}
      Type-dependent payoff matrices:
      \vspace{-8pt}
      \begin{table}
          \begin{tabular}{l|c|c|}
          \multicolumn{1}{c}{} & \multicolumn{2}{c}{Type $t_1:a=2\ (p=\frac{1}{2})$} \\
          \multicolumn{1}{c}{} & \multicolumn{1}{c}{L} & \multicolumn{1}{c}{R} \\\cline{2-3}
          U & 2, 1 & 0, 2 \\\cline{2-3}
          D & 0, 1 & 1, 2 \\\cline{2-3}
        \end{tabular}
      \end{table}
      \vspace{-8pt}
      \begin{table}
        \begin{tabular}{l|c|c|}
          \multicolumn{1}{c}{} & \multicolumn{2}{c}{Type $t_2:a=-2\ (p=\frac{1}{2})$} \\
          \multicolumn{1}{c}{} & \multicolumn{1}{c}{L} & \multicolumn{1}{c}{R} \\\cline{2-3}
          U & 2, 1 & 0, -2 \\\cline{2-3}
          D & 0, 1 & 1, -2 \\\cline{2-3}
        \end{tabular}
      \end{table}
      \vspace{-8pt}
      \begin{itemize}
          \item[(c)] The strategy for P1 is just one action, whereas the strategy for P2 is an action for each of the two types. Using this, we can write up the expected payoff matrix:
      \end{itemize}
      \vspace{-12pt}
      \begin{table}
        \begin{tabular}{l|c|c|c|c|}
          \multicolumn{1}{c}{} & \multicolumn{1}{c}{LL} & \multicolumn{1}{c}{LR} & \multicolumn{1}{c}{RL} & \multicolumn{1}{c}{RR} \\\cline{2-5}
          U & \textcolor{red}{2},1 &  \textcolor{red}{1},-$\frac{1}{2}$ & \textcolor{red}{1},\textcolor{blue}{$\frac{3}{2}$} & 0, 0  \\\cline{2-5}
          D & 0, 1  & $\frac{1}{2}$, -$\frac{1}{2}$ & $\frac{1}{2}$, \textcolor{blue}{$\frac{3}{2}$} & \textcolor{red}{1}, 0  \\\cline{2-5}
        \end{tabular}
      \end{table}
      \vspace{-4pt}
      The BNE is: (U,RL)
      \vfill\null
    \end{multicols}
\end{frame}



\section{PS8, Ex. 4: Static Bayesian game (Bayesian Nash Equilibria)}

\begin{frame}{PS8, Ex. 4: Static Bayesian game (Bayesian Nash Equilibria)}
    Exercise 3.4 in Gibbons (p. 169). Find all the pure-strategy Bayesian Nash equilibria in the following static Bayesian game:
    \begin{itemize}
        \item[a.] Nature determines whether the payoffs are as in Game 1 or as in Game 2, each game being equally likely.
        \item[b.] Player 1 learns whether nature has drawn Game 1 or Game 2, but player 2 does not.
        \item[c.] Player 1 chooses either T or B; player 2 simultaneously chooses either L or R.
        \item[d.] Payoffs are given by the game drawn by nature.
    \end{itemize}
    \begin{table}
      \begin{tabular}{l|c|c|}
        \multicolumn{3}{c}{Game 1 $\left(t_1, p=\frac{1}{2}\right)$:}
        \\
        \multicolumn{1}{c}{} & \multicolumn{1}{c}{L} & \multicolumn{1}{c}{R} \\\cline{2-3}
        T & 1, 1 & 0, 0 \\\cline{2-3}
        B & 0, 0 & 0, 0 \\\cline{2-3}
      \end{tabular}
      \begin{tabular}{l|c|c|}
        \multicolumn{3}{c}{Game 2 $\left(t_2, p=\frac{1}{2}\right)$:}
        \\
        \multicolumn{1}{c}{} & \multicolumn{1}{c}{L} & \multicolumn{1}{c}{R} \\\cline{2-3}
        T & 0, 0 & 0, 0 \\\cline{2-3}
        B & 0, 0 & 2, 2 \\\cline{2-3}
      \end{tabular}
    \end{table}
    \vfill\null
\end{frame}

\begin{frame}{PS8, Ex. 4.a: Static Bayesian game (Bayesian Nash Equilibria)}
    Find all the Bayesian Nash equilibria in the following static Bayesian game:
    \begin{table}
      \begin{tabular}{l|c|c|}
        \multicolumn{3}{c}{Game 1 $\left(t_1, p=\frac{1}{2}\right)$:}
        \\
        \multicolumn{1}{c}{} & \multicolumn{1}{c}{L} & \multicolumn{1}{c}{R} \\\cline{2-3}
        T & 1, 1 & 0, 0 \\\cline{2-3}
        B & 0, 0 & 0, 0 \\\cline{2-3}
      \end{tabular}
      \begin{tabular}{l|c|c|}
        \multicolumn{3}{c}{Game 2 $\left(t_2, p=\frac{1}{2}\right)$:}
        \\
        \multicolumn{1}{c}{} & \multicolumn{1}{c}{L} & \multicolumn{1}{c}{R} \\\cline{2-3}
        T & 0, 0 & 0, 0 \\\cline{2-3}
        B & 0, 0 & 2, 2 \\\cline{2-3}
      \end{tabular}
    \end{table}
    \textit{\textbf{Use the fact that each type of game happens half the time to write up the expected payoff matrix for the following possibilities:}}
    \begin{itemize}
        \item P2 plays L and P1 plays T if game is type 1 and T if game is type 2
        \item P2 plays L and P1 plays T if game is type 1 and B if game is type 2
        \item P2 plays L and P1 plays B if game is type 1 and T if game is type 2
        \item P2 plays L and P1 plays B if game is type 1 and B if game is type 2
        \item P2 plays R and P1 plays T if game is type 1 and T if game is type 2
        \item P2 plays R and P1 plays T if game is type 1 and B if game is type 2
        \item P2 plays R and P1 plays B if game is type 1 and T if game is type 2
        \item P2 plays R and P1 plays B if game is type 1 and B if game is type 2
    \end{itemize}
    \vfill\null\null
\end{frame}
\begin{frame}{PS8, Ex. 4.a: Static Bayesian game (Bayesian Nash Equilibria)}
    Find all the Bayesian Nash equilibria in the following static Bayesian game:
    \vspace{-4pt}
    \begin{table}
      \begin{tabular}{l|c|c|}
        \multicolumn{3}{c}{Game 1 $\left(t_1, p=\frac{1}{2}\right)$:}
        \\
        \multicolumn{1}{c}{} & \multicolumn{1}{c}{L} & \multicolumn{1}{c}{R} \\\cline{2-3}
        T & 1, 1 & 0, 0 \\\cline{2-3}
        B & 0, 0 & 0, 0 \\\cline{2-3}
      \end{tabular}
      \begin{tabular}{l|c|c|}
        \multicolumn{3}{c}{Game 2 $\left(t_2, p=\frac{1}{2}\right)$:}
        \\
        \multicolumn{1}{c}{} & \multicolumn{1}{c}{L} & \multicolumn{1}{c}{R} \\\cline{2-3}
        T & 0, 0 & 0, 0 \\\cline{2-3}
        B & 0, 0 & 2, 2 \\\cline{2-3}
      \end{tabular}
    \end{table}
    \vspace{-4pt}
    \begin{itemize}
      \item P2 plays L and P1 plays T if game is type 1 and T if game is type 2
      \item P2 plays L and P1 plays T if game is type 1 and B if game is type 2
      \item P2 plays L and P1 plays B if game is type 1 and T if game is type 2
      \item P2 plays L and P1 plays B if game is type 1 and B if game is type 2
      \item P2 plays R and P1 plays T if game is type 1 and T if game is type 2
      \item P2 plays R and P1 plays T if game is type 1 and B if game is type 2
      \item P2 plays R and P1 plays B if game is type 1 and T if game is type 2
      \item P2 plays R and P1 plays B if game is type 1 and B if game is type 2
    \end{itemize}
    \vspace{-4pt}
    \begin{multicols}{2}
      The expected payoff matrix:
      \vspace{-4pt}
      \begin{table}
        \begin{tabular}{cl|c|c|}
          & \multicolumn{1}{c}{} & \multicolumn{2}{c}{Player 2}\\
          \parbox[t]{1mm}{\multirow{5}{*}{\rotatebox[origin=r]{90}{Player 1}}}
          & \multicolumn{1}{c}{} & \multicolumn{1}{c}{L} & \multicolumn{1}{c}{R} \\\cline{3-4}
          & TT & $\frac{1}{2}$, $\frac{1}{2}$ &  0, 0  \\\cline{3-4}
          & TB & $\frac{1}{2}$, $\frac{1}{2}$  & 1, 1 \\\cline{3-4}
          & BT & 0, 0 & 0, 0 \\\cline{3-4}
          & BB & 0, 0  & 1, 1 \\\cline{3-4}
        \end{tabular}
      \end{table}
      \vfill\null\columnbreak
      \vfill\null\null
      \textbf{\textit{Find all Bayesian Nash Equilibria.}}
    \end{multicols}
\end{frame}
\begin{frame}{PS8, Ex. 4.a: Static Bayesian game (Bayesian Nash Equilibria)}
    Find all the Bayesian Nash equilibria in the following static Bayesian game:
    \vspace{-4pt}
    \begin{table}
      \begin{tabular}{l|c|c|}
        \multicolumn{3}{c}{Game 1 $\left(t_1, p=\frac{1}{2}\right)$:}
        \\
        \multicolumn{1}{c}{} & \multicolumn{1}{c}{L} & \multicolumn{1}{c}{R} \\\cline{2-3}
        T & 1, 1 & 0, 0 \\\cline{2-3}
        B & 0, 0 & 0, 0 \\\cline{2-3}
      \end{tabular}
      \begin{tabular}{l|c|c|}
        \multicolumn{3}{c}{Game 2 $\left(t_2, p=\frac{1}{2}\right)$:}
        \\
        \multicolumn{1}{c}{} & \multicolumn{1}{c}{L} & \multicolumn{1}{c}{R} \\\cline{2-3}
        T & 0, 0 & 0, 0 \\\cline{2-3}
        B & 0, 0 & 2, 2 \\\cline{2-3}
      \end{tabular}
    \end{table}
    \vspace{-4pt}
    \begin{itemize}
      \item P2 plays L and P1 plays T if game is type 1 and T if game is type 2
      \item P2 plays L and P1 plays T if game is type 1 and B if game is type 2
      \item P2 plays L and P1 plays B if game is type 1 and T if game is type 2
      \item P2 plays L and P1 plays B if game is type 1 and B if game is type 2
      \item P2 plays R and P1 plays T if game is type 1 and T if game is type 2
      \item P2 plays R and P1 plays T if game is type 1 and B if game is type 2
      \item P2 plays R and P1 plays B if game is type 1 and T if game is type 2
      \item P2 plays R and P1 plays B if game is type 1 and B if game is type 2
    \end{itemize}
    \vspace{-4pt}
    \begin{multicols}{2}
      The expected payoff matrix:
      \vspace{-4pt}
      \begin{table}
        \begin{tabular}{cl|c|c|}
          & \multicolumn{1}{c}{} & \multicolumn{2}{c}{\color{blue}Player 2}\\
          \parbox[t]{1mm}{\multirow{5}{*}{\rotatebox[origin=r]{90}{\color{red}Player 1}}}
          & \multicolumn{1}{c}{} & \multicolumn{1}{c}{L} & \multicolumn{1}{c}{R} \\\cline{3-4}
          & TT & \textcolor{red}{$\frac{1}{2}$}, \textcolor{blue}{$\frac{1}{2}$} &  0, 0  \\\cline{3-4}
          & TB & \textcolor{red}{$\frac{1}{2}$}, $\frac{1}{2}$  & \textcolor{red}{1}, \textcolor{blue}{1}\\\cline{3-4}
          & BT & 0, \textcolor{blue}{0}  & 0, \textcolor{blue}{0}\\\cline{3-4}
          & BB & 0, 0  & \textcolor{red}{1}, \textcolor{blue}{1}\\\cline{3-4}
        \end{tabular}
      \end{table}
      \vfill\null\columnbreak
      \vfill\null
      This gives the following BNE:
      \begin{align*}
        BNE=\{(TT,L),(TB,R),(BB,R)\}
      \end{align*}
    \end{multicols}
\end{frame}
