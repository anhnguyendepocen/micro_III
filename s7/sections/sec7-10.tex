\section{PS7, Ex. 7: Bertrand duopoly (infinitely repeated game)}

\begin{frame}{PS7, Ex. 7: Bertrand duopoly (infinitely repeated game)}
    Exercise 2.13 in Gibbons (p. 135): Recall the static Bertrand duopoly model (with homogeneous products) from Problem 1.7: the firms name prices simultaneously; demand for firm $i$'s product is $a-p_i$ if $p_i<p_j$, is 0 if $p_i>p_j$, and is $(a-p_i)/2$ if $p_i=p_j$; marginal costs are $c<a$. Consider the infinitely repeated game based on this stage game. Show that the firms can use trigger strategies (that switch forever to the stage-game Nash equilibrium after any deviation) to sustain the monopoly price level in a subgame-perfect Nash equilibrium if and only if $\delta\geq1/2$.
    \vfill\null
\end{frame}

\begin{frame}{PS7, Ex. 7: Bertrand duopoly (infinitely repeated game)}
    Show that the firms can use trigger strategies (that switch forever to the stage-game Nash equilibrium after any deviation) to sustain the monopoly price level in a subgame-perfect Nash equilibrium if and only if $\delta\geq1/2$.
    \begin{multicols}{2}
      \vfill\null\columnbreak
      Information so far:
      \begin{enumerate}
        \item Players: Firm $i,\ i\in1,2$
        \item Strategies: $S_i=\{p_i|p_i\in\mathbb{R}_+\}$
        \item Payoffs:
      \end{enumerate}
      \vspace{-12pt}
      \begin{align*}
        \pi_i(p_i,p_j)&=(price-cost)\cdot demand\\
                      &=\left\{\begin{array}{lcl}
          (p_i-c)(a-p_i)            & \text{if} & p_i<p_j\\
          \frac{1}{2}(p_i-c)(a-p_i) & \text{if} & p_i=p_j\\
          0                         & \text{if} & p_i>p_j
        \end{array}\right.
      \end{align*}
      \vfill\null
    \end{multicols}
\end{frame}

\begin{frame}{PS7, Ex. 7: Bertrand duopoly (infinitely repeated game)}
    Show that the firms can use trigger strategies (that switch forever to the stage-game Nash equilibrium after any deviation) to sustain the monopoly price level in a subgame-perfect Nash equilibrium if and only if $\delta\geq1/2$.
    \vspace{-6pt}
    \begin{multicols}{2}
      \begin{itemize}
        \item[Step a:] Recall the stage game price level.
      \end{itemize}
      \vspace{-4pt}
      \vfill\null\columnbreak
      Information so far:
      \vspace{-4pt}
      \begin{enumerate}
        \item Players: Firm $i,\ i\in1,2$
        \item Strategies: $S_i=\{p_i|p_i\in\mathbb{R}_+\}$
        \item Payoffs:
      \end{enumerate}
      \vspace{-12pt}
      \begin{align*}
        \pi_i(p_i,p_j)&=(price-marginal\ cost)\cdot demand\\
                      &=\left\{\begin{array}{lcl}
          (p_i-c)(a-p_i)            & \text{if} & p_i<p_j\\
          \frac{1}{2}(p_i-c)(a-p_i) & \text{if} & p_i=p_j\\
          0                         & \text{if} & p_i>p_j
        \end{array}\right.
      \end{align*}
      \vfill\null
    \end{multicols}
\end{frame}
\begin{frame}{PS7, Ex. 7: Bertrand duopoly (infinitely repeated game)}
    Show that the firms can use trigger strategies (that switch forever to the stage-game Nash equilibrium after any deviation) to sustain the monopoly price level in a subgame-perfect Nash equilibrium if and only if $\delta\geq1/2$.
    \vspace{-6pt}
    \begin{multicols}{2}
      \begin{itemize}
        \item[Step a:] Recall the stage game price level.
      \end{itemize}
      \vspace{-4pt}
      \vfill\null\columnbreak
      Information so far:
      \vspace{-4pt}
      \begin{enumerate}
        \item Players: Firm $i,\ i\in1,2$
        \item Strategies: $S_i=\{p_i|p_i\in\mathbb{R}_+\}$
        \item Payoffs:
      \end{enumerate}
      \vspace{-12pt}
      \begin{align*}
        \pi_i(p_i,p_j)&=(price-marginal\ cost)\cdot demand\\
                      &=\left\{\begin{array}{lcl}
          (p_i-c)(a-p_i)            & \text{if} & p_i<p_j\\
          \frac{1}{2}(p_i-c)(a-p_i) & \text{if} & p_i=p_j\\
          0                         & \text{if} & p_i>p_j
        \end{array}\right.
      \end{align*}
      \vspace{-16pt}
      \begin{itemize}
        \item[a:] Stage game NE: $p_1^*=p_2^*=c$
      \end{itemize}
      \vfill\null
    \end{multicols}
\end{frame}
\begin{frame}{PS7, Ex. 7: Bertrand duopoly (infinitely repeated game)}
    Show that the firms can use trigger strategies (that switch forever to the stage-game Nash equilibrium after any deviation) to sustain the monopoly price level in a subgame-perfect Nash equilibrium if and only if $\delta\geq1/2$.
    \vspace{-6pt}
    \begin{multicols}{2}
      \begin{itemize}
        \item[Step a:] Recall the stage game price level.
        \item[Step b:] Suggest a trigger strategy that can sustain the monopoly price level $p^M$.
      \end{itemize}
      \vspace{-4pt}
      \vfill\null\columnbreak
      Information so far:
      \vspace{-4pt}
      \begin{enumerate}
        \item Players: Firm $i,\ i\in1,2$
        \item Strategies: $S_i=\{p_i|p_i\in\mathbb{R}_+\}$
        \item Payoffs:
      \end{enumerate}
      \vspace{-12pt}
      \begin{align*}
        \pi_i(p_i,p_j)&=(price-marginal\ cost)\cdot demand\\
                      &=\left\{\begin{array}{lcl}
          (p_i-c)(a-p_i)            & \text{if} & p_i<p_j\\
          \frac{1}{2}(p_i-c)(a-p_i) & \text{if} & p_i=p_j\\
          0                         & \text{if} & p_i>p_j
        \end{array}\right.
      \end{align*}
      \vspace{-16pt}
      \begin{itemize}
        \item[a:] Stage game NE: $p_1^*=p_2^*=c$
      \end{itemize}
      \vfill\null
    \end{multicols}
\end{frame}
\begin{frame}{PS7, Ex. 7: Bertrand duopoly (infinitely repeated game)}
    Show that the firms can use trigger strategies (that switch forever to the stage-game Nash equilibrium after any deviation) to sustain the monopoly price level in a subgame-perfect Nash equilibrium if and only if $\delta\geq1/2$.
    \vspace{-6pt}
    \begin{multicols}{2}
      \begin{itemize}
        \item[Step a:] Recall the stage game price level.
        \item[Step b:] Suggest a trigger strategy that can sustain the monopoly price level $p^M$.
      \end{itemize}
      \vspace{-4pt}
      \vfill\null\columnbreak
      Information so far:
      \vspace{-4pt}
      \begin{enumerate}
        \item Players: Firm $i,\ i\in1,2$
        \item Strategies: $S_i=\{p_i|p_i\in\mathbb{R}_+\}$
        \item Payoffs:
      \end{enumerate}
      \vspace{-12pt}
      \begin{align*}
        \pi_i(p_i,p_j)&=(price-marginal\ cost)\cdot demand\\
                      &=\left\{\begin{array}{lcl}
          (p_i-c)(a-p_i)            & \text{if} & p_i<p_j\\
          \frac{1}{2}(p_i-c)(a-p_i) & \text{if} & p_i=p_j\\
          0                         & \text{if} & p_i>p_j
        \end{array}\right.
      \end{align*}
      \vspace{-16pt}
      \begin{itemize}
        \item[a:] Stage game NE: $p_1^*=p_2^*=c$
        \item[b:] Play $p^M$ in $t=0$ or if it was played in all previous rounds ("normal").
        \item[]   Play $p=c$ if there was a deviation in any previous round ("punishment").
      \end{itemize}
      \vfill\null
    \end{multicols}
\end{frame}
\begin{frame}{PS7, Ex. 7: Bertrand duopoly (infinitely repeated game)}
    Show that the firms can use trigger strategies (that switch forever to the stage-game Nash equilibrium after any deviation) to sustain the monopoly price level in a subgame-perfect Nash equilibrium if and only if $\delta\geq1/2$.
    \vspace{-6pt}
    \begin{multicols}{2}
      \begin{itemize}
        \item[Step a:] Recall the stage game price level.
        \item[Step b:] Suggest a trigger strategy that can sustain the monopoly price level $p^M$.
        \item[Step c:] Check that the trigger strategy (TS) is a NE in the "normal" phase.
      \end{itemize}
      \vspace{-4pt}
      \vfill\null\columnbreak
      Information so far:
      \vspace{-4pt}
      \begin{enumerate}
        \item Players: Firm $i,\ i\in1,2$
        \item Strategies: $S_i=\{p_i|p_i\in\mathbb{R}_+\}$
        \item Payoffs:
      \end{enumerate}
      \vspace{-12pt}
      \begin{align*}
        \pi_i(p_i,p_j)&=(price-marginal\ cost)\cdot demand\\
                      &=\left\{\begin{array}{lcl}
          (p_i-c)(a-p_i)            & \text{if} & p_i<p_j\\
          \frac{1}{2}(p_i-c)(a-p_i) & \text{if} & p_i=p_j\\
          0                         & \text{if} & p_i>p_j
        \end{array}\right.
      \end{align*}
      \vspace{-16pt}
      \begin{itemize}
        \item[a:] Stage game NE: $p_1^*=p_2^*=c$
        \item[b:] Play $p^M$ in $t=0$ or if it was played in all previous rounds ("normal").
        \item[]   Play $p=c$ if there was a deviation in any previous round ("punishment").
      \end{itemize}
      \vfill\null
    \end{multicols}
\end{frame}
\begin{frame}{PS7, Ex. 7: Bertrand duopoly (infinitely repeated game)}
    Show that the firms can use trigger strategies (that switch forever to the stage-game Nash equilibrium after any deviation) to sustain the monopoly price level in a subgame-perfect Nash equilibrium if and only if $\delta\geq1/2$.
    \vspace{-6pt}
    \begin{multicols}{2}
      \begin{itemize}
        \item[Step a:] Recall the stage game price level.
        \item[Step b:] Suggest a trigger strategy that can sustain the monopoly price level $p^M$.
        \item[Step c:] Check that the trigger strategy (TS) is a NE in the "normal" phase:
      \end{itemize}
      \vspace{-4pt}
      I.e. to split the monopoly market (LHS) vs the best deviation which is to slightly underbid, i.e. $p=p^M-\varepsilon\approx p^M$ (RHS).
      \vfill\null\columnbreak
      Information so far:
      \vspace{-4pt}
      \begin{enumerate}
        \item Players: Firm $i,\ i\in1,2$
        \item Strategies: $S_i=\{p_i|p_i\in\mathbb{R}_+\}$
        \item Payoffs:
      \end{enumerate}
      \vspace{-12pt}
      \begin{align*}
        \pi_i(p_i,p_j)&=(price-marginal\ cost)\cdot demand\\
                      &=\left\{\begin{array}{lcl}
          (p_i-c)(a-p_i)            & \text{if} & p_i<p_j\\
          \frac{1}{2}(p_i-c)(a-p_i) & \text{if} & p_i=p_j\\
          0                         & \text{if} & p_i>p_j
        \end{array}\right.
      \end{align*}
      \vspace{-16pt}
      \begin{itemize}
        \item[a:] Stage game NE: $p_1^*=p_2^*=c$
        \item[b:] Play $p^M$ in $t=0$ or if it was played in all previous rounds ("normal").
        \item[]   Play $p=c$ if there was a deviation in any previous round ("punishment").
      \end{itemize}
      \vfill\null
    \end{multicols}
\end{frame}
\begin{frame}{PS7, Ex. 7: Bertrand duopoly (infinitely repeated game)}
    Show that the firms can use trigger strategies (that switch forever to the stage-game Nash equilibrium after any deviation) to sustain the monopoly price level in a subgame-perfect Nash equilibrium if and only if $\delta\geq1/2$.
    \vspace{-6pt}
    \begin{multicols}{2}
      \begin{itemize}
        \item[Step a:] Recall the stage game price level.
        \item[Step b:] Suggest a trigger strategy that can sustain the monopoly price level $p^M$.
        \item[Step c:] Check that the trigger strategy (TS) is a NE in the "normal" phase:
      \end{itemize}
      \vspace{-4pt}
      To split the monopoly market (LHS) vs the best deviation which is to slightly underbid, i.e. $p=p^M-\varepsilon\approx p^M$ (RHS):
      \vspace{-4pt}
      \begin{align*}
        \sum\limits_{t=0}^\infty\frac{1}{2}\pi^M\delta^t&\geq\pi^M+\sum\limits_{t=1}^\infty\frac{1}{2}0\cdot\delta^t\Leftrightarrow\\
        \frac{\frac{1}{2}\pi^M}{1-\delta}&\geq\pi^M\Leftrightarrow\\
        \frac{1}{2}&\geq(1-\delta)\Leftrightarrow\\
        1&\geq2-2\delta\Leftrightarrow\\
        \delta&\geq\frac{1}{2}
      \end{align*}
      \vfill\null\columnbreak
      Information so far:
      \vspace{-4pt}
      \begin{enumerate}
        \item Players: Firm $i,\ i\in1,2$
        \item Strategies: $S_i=\{p_i|p_i\in\mathbb{R}_+\}$
        \item Payoffs:
      \end{enumerate}
      \vspace{-12pt}
      \begin{align*}
        \pi_i(p_i,p_j)&=(price-marginal\ cost)\cdot demand\\
                      &=\left\{\begin{array}{lcl}
          (p_i-c)(a-p_i)            & \text{if} & p_i<p_j\\
          \frac{1}{2}(p_i-c)(a-p_i) & \text{if} & p_i=p_j\\
          0                         & \text{if} & p_i>p_j
        \end{array}\right.
      \end{align*}
      \vspace{-16pt}
      \begin{itemize}
        \item[a:] Stage game NE: $p_1^*=p_2^*=c$
        \item[b:] Play $p^M$ in $t=0$ or if it was played in all previous rounds ("normal").
        \item[]   Play $p=c$ if there was a deviation in any previous round ("punishment").
        \item[c:] The TS is a NE in the "normal" phase for $\delta\geq\frac{1}{2}$
      \end{itemize}
      \vfill\null
    \end{multicols}
\end{frame}
\begin{frame}{PS7, Ex. 7: Bertrand duopoly (infinitely repeated game)}
    Show that the firms can use trigger strategies (that switch forever to the stage-game Nash equilibrium after any deviation) to sustain the monopoly price level in a subgame-perfect Nash equilibrium if and only if $\delta\geq1/2$.
    \vspace{-6pt}
    \begin{multicols}{2}
      \begin{itemize}
        \item[Step a:] Recall the stage game price level.
        \item[Step b:] Suggest a trigger strategy that can sustain the monopoly price level $p^M$.
        \item[Step c:] Check that the trigger strategy (TS) is a NE in the "normal" phase.
        \item[Step d:] Check that the trigger strategy (TS) is a NE in the "punishment" phase.
      \end{itemize}
      \vfill\null\columnbreak
      Information so far:
      \vspace{-4pt}
      \begin{enumerate}
        \item Players: Firm $i,\ i\in1,2$
        \item Strategies: $S_i=\{p_i|p_i\in\mathbb{R}_+\}$
        \item Payoffs:
      \end{enumerate}
      \vspace{-12pt}
      \begin{align*}
        \pi_i(p_i,p_j)&=(price-marginal\ cost)\cdot demand\\
                      &=\left\{\begin{array}{lcl}
          (p_i-c)(a-p_i)            & \text{if} & p_i<p_j\\
          \frac{1}{2}(p_i-c)(a-p_i) & \text{if} & p_i=p_j\\
          0                         & \text{if} & p_i>p_j
        \end{array}\right.
      \end{align*}
      \vspace{-16pt}
      \begin{itemize}
        \item[a:] Stage game NE: $p_1^*=p_2^*=c$
        \item[b:] Play $p^M$ in $t=0$ or if it was played in all previous rounds ("normal").
        \item[]   Play $p=c$ if there was a deviation in any previous round ("punishment").
        \item[c:] The TS is a NE in the "normal" phase for $\delta\geq\frac{1}{2}$
      \end{itemize}
      \vfill\null
    \end{multicols}
\end{frame}
\begin{frame}{PS7, Ex. 7: Bertrand duopoly (infinitely repeated game)}
    Show that the firms can use trigger strategies (that switch forever to the stage-game Nash equilibrium after any deviation) to sustain the monopoly price level in a subgame-perfect Nash equilibrium if and only if $\delta\geq1/2$.
    \vspace{-6pt}
    \begin{multicols}{2}
      \begin{itemize}
        \item[Step a:] Recall the stage game price level.
        \item[Step b:] Suggest a trigger strategy that can sustain the monopoly price level $p^M$.
        \item[Step c:] Check that the trigger strategy (TS) is a NE in the "normal" phase.
        \item[Step d:] Check that the trigger strategy (TS) is a NE in the "punishment" phase:
      \end{itemize}
      \vspace{-4pt}
      Given that $p=c$ is a NE in the stage game, it must also be a NE in the "punishment" phase, i.e. it's a best response for both firms, given the other firm's price of $p=c$.
      \vfill\null\columnbreak
      Information so far:
      \vspace{-4pt}
      \begin{enumerate}
        \item Players: Firm $i,\ i\in1,2$
        \item Strategies: $S_i=\{p_i|p_i\in\mathbb{R}_+\}$
        \item Payoffs:
      \end{enumerate}
      \vspace{-12pt}
      \begin{align*}
        \pi_i(p_i,p_j)&=(price-marginal\ cost)\cdot demand\\
                      &=\left\{\begin{array}{lcl}
          (p_i-c)(a-p_i)            & \text{if} & p_i<p_j\\
          \frac{1}{2}(p_i-c)(a-p_i) & \text{if} & p_i=p_j\\
          0                         & \text{if} & p_i>p_j
        \end{array}\right.
      \end{align*}
      \vspace{-16pt}
      \begin{itemize}
        \item[a:] Stage game NE: $p_1^*=p_2^*=c$
        \item[b:] Play $p^M$ in $t=0$ or if it was played in all previous rounds ("normal").
        \item[]   Play $p=c$ if there was a deviation in any previous round ("punishment").
        \item[c:] The TS is a NE in the "normal" phase for $\delta\geq\frac{1}{2}$
      \end{itemize}
      \vfill\null
    \end{multicols}
\end{frame}
\begin{frame}{PS7, Ex. 7: Bertrand duopoly (infinitely repeated game)}
    Show that the firms can use trigger strategies (that switch forever to the stage-game Nash equilibrium after any deviation) to sustain the monopoly price level in a subgame-perfect Nash equilibrium if and only if $\delta\geq1/2$.
    \vspace{-6pt}
    \begin{multicols}{2}
      \begin{itemize}
        \item[Step a:] Recall the stage game price level.
        \item[Step b:] Suggest a trigger strategy that can sustain the monopoly price level $p^M$.
        \item[Step c:] Check that the trigger strategy (TS) is a NE in the "normal" phase.
        \item[Step d:] Check that the trigger strategy (TS) is a NE in the "punishment" phase:
      \end{itemize}
      \vspace{-4pt}
      Given that $p=c$ is a NE in the stage game, it must also be a NE in the "punishment" phase, i.e. it's a best response for both firms, given the other firm's price of $p=c$.\\\medskip
      Thus, the trigger strategies gives a SPNE where the firms can act together as a monopolist if the they are sufficiently patient, i.e. for $\delta\geq\frac{1}{2}$.
      \vfill\null\columnbreak
      Information so far:
      \vspace{-4pt}
      \begin{enumerate}
        \item Players: Firm $i,\ i\in1,2$
        \item Strategies: $S_i=\{p_i|p_i\in\mathbb{R}_+\}$
        \item Payoffs:
      \end{enumerate}
      \vspace{-12pt}
      \begin{align*}
        \pi_i(p_i,p_j)&=(price-marginal\ cost)\cdot demand\\
                      &=\left\{\begin{array}{lcl}
          (p_i-c)(a-p_i)            & \text{if} & p_i<p_j\\
          \frac{1}{2}(p_i-c)(a-p_i) & \text{if} & p_i=p_j\\
          0                         & \text{if} & p_i>p_j
        \end{array}\right.
      \end{align*}
      \vspace{-16pt}
      \begin{itemize}
        \item[a:] Stage game NE: $p_1^*=p_2^*=c$
        \item[b:] Play $p^M$ in $t=0$ or if it was played in all previous rounds ("normal").
        \item[]   Play $p=c$ if there was a deviation in any previous round ("punishment").
        \item[c:] The TS is a NE in the "normal" phase for $\delta\geq\frac{1}{2}$
        \item[d:] TS is SPNE for $\delta\geq\frac{1}{2}$
      \end{itemize}
      \vfill\null
    \end{multicols}
\end{frame}



\section{PS7, Ex. 8: Trigger strategy (infinitely repeated game)}

\begin{frame}{PS7, Ex. 8: Trigger strategy (infinitely repeated game)}
    The next exercises use the following game $G$:
    \begin{table}
      \begin{tabular}{l|c|c|c|}
        \multicolumn{1}{c}{} & \multicolumn{1}{c}{L} & \multicolumn{1}{c}{M} & \multicolumn{1}{c}{H} \\\cline{2-4}
        L & 10, 10 & 3, 15 & 0, 7 \\\cline{2-4}
        M & 15, 3 & 7, 7 & -4, 5 \\\cline{2-4}
        H & 7, 0 & 5, -4 & -15, -15 \\\cline{2-4}
      \end{tabular}
    \end{table}
    Suppose that the Players play the infinitely repeated game $G(\infty)$ and that they would like to support as a SPNE the 'collusive' outcome in which $(L, L)$ is played every round.
    \begin{itemize}
      \item[(a)] Define a trigger strategy which delivers the collusive outcome in every period where no deviation has been made, and gives $(x_1, x_2)$ forever after a deviation.
      \item[(b)] A necessary (but not sufficient) condition for a SPNE is $x_1 = x_2 = M$. Explain why.
      \vspace{-4pt} \item[(c)] Suppose $\delta = 4/7$. Show by finding a profitable deviation that the above trigger strategy is not a SPNE. \vspace{-6pt}
    \end{itemize}
    \vfill\null
\end{frame}

\begin{frame}{PS7, Ex. 8.a: Trigger strategy (infinitely repeated game)}
    Consider $G(\infty)$, i.e. the infinitely repeated game with stage game $G$: \vspace{-6pt}
    \begin{table}
      \begin{tabular}{l|c|c|c|}
        \multicolumn{1}{c}{} & \multicolumn{1}{c}{L} & \multicolumn{1}{c}{M} & \multicolumn{1}{c}{H} \\\cline{2-4}
        L & 10, 10 & 3, 15 & 0, 7 \\\cline{2-4}
        M & 15, 3 & 7, 7 & -4, 5 \\\cline{2-4}
        H & 7, 0 & 5, -4 & -15, -15 \\\cline{2-4}
      \end{tabular}
    \end{table}
    \begin{itemize}
      \item[(a)] Define a trigger strategy which delivers the collusive outcome in every period where no deviation has been made, and gives $(x_1, x_2)$ forever after a deviation.
    \end{itemize}
    \vfill\null
\end{frame}
\begin{frame}{PS7, Ex. 8.a: Trigger strategy (infinitely repeated game)}
    Consider $G(\infty)$, i.e. the infinitely repeated game with stage game $G$: \vspace{-6pt}
    \begin{table}
      \begin{tabular}{l|c|c|c|}
        \multicolumn{1}{c}{} & \multicolumn{1}{c}{L} & \multicolumn{1}{c}{M} & \multicolumn{1}{c}{H} \\\cline{2-4}
        L & 10, 10 & 3, 15 & 0, 7 \\\cline{2-4}
        M & 15, 3 & 7, 7 & -4, 5 \\\cline{2-4}
        H & 7, 0 & 5, -4 & -15, -15 \\\cline{2-4}
      \end{tabular}
    \end{table}
    \begin{itemize}
      \item[(a)] Define a trigger strategy which delivers the collusive outcome in every period where no deviation has been made, and gives $(x_1, x_2)$ forever after a deviation.
    \end{itemize}
    \begin{multicols}{2}
      \begin{itemize}
        \item[(Step a)] Write up the trigger strategy.
      \end{itemize}
      \vfill\null\columnbreak
      \vfill\null
    \end{multicols}
\end{frame}
\begin{frame}{PS7, Ex. 8.a: Trigger strategy (infinitely repeated game)}
    Consider $G(\infty)$, i.e. the infinitely repeated game with stage game $G$: \vspace{-6pt}
    \begin{table}
      \begin{tabular}{l|c|c|c|}
        \multicolumn{1}{c}{} & \multicolumn{1}{c}{L} & \multicolumn{1}{c}{M} & \multicolumn{1}{c}{H} \\\cline{2-4}
        L & 10, 10 & 3, 15 & 0, 7 \\\cline{2-4}
        M & 15, 3 & 7, 7 & -4, 5 \\\cline{2-4}
        H & 7, 0 & 5, -4 & -15, -15 \\\cline{2-4}
      \end{tabular}
    \end{table}
    \begin{itemize}
      \item[(a)] Define a trigger strategy which delivers the collusive outcome in every period where no deviation has been made, and gives $(x_1, x_2)$ forever after a deviation.
    \end{itemize}
    \begin{multicols}{2}
      \begin{itemize}
        \item[(Step a)] Write up the trigger strategy.
      \end{itemize}
      \vfill\null\columnbreak
      Information so far:
      \begin{enumerate}
        \item Trigger strategy for Player $i\in1,2$: "If $t=1$ or if the outcome in all previous stages was $(L,L)$, play $L$. Otherwise, play $x_i$."
      \end{enumerate}
      \vfill\null
    \end{multicols}
\end{frame}

\begin{frame}{PS7, Ex. 8.b: Trigger strategy (infinitely repeated game)}
    Consider $G(\infty)$, i.e. the infinitely repeated game with stage game $G$: \vspace{-6pt}
    \begin{table}
      \begin{tabular}{l|c|c|c|}
        \multicolumn{1}{c}{} & \multicolumn{1}{c}{L} & \multicolumn{1}{c}{M} & \multicolumn{1}{c}{H} \\\cline{2-4}
        L & 10, 10 & 3, 15 & 0, 7 \\\cline{2-4}
        M & 15, 3 & 7, 7 & -4, 5 \\\cline{2-4}
        H & 7, 0 & 5, -4 & -15, -15 \\\cline{2-4}
      \end{tabular}
    \end{table}
    \begin{itemize}
      \item[(b)] A necessary (but not sufficient) condition for a SPNE is $x_1 = x_2 = M$. Explain why.
    \end{itemize}
    \begin{multicols}{2}
      \vfill\null\columnbreak
      Information so far:
      \begin{enumerate}
        \item Trigger strategy for Player $i\in1,2$: "If $t=1$ or if the outcome in all previous stages was $(L,L)$, play $L$. Otherwise, play $x_i$."
      \end{enumerate}
      \vfill\null
    \end{multicols}
\end{frame}
\begin{frame}{PS7, Ex. 8.b: Trigger strategy (infinitely repeated game)}
    Consider $G(\infty)$, i.e. the infinitely repeated game with stage game $G$: \vspace{-6pt}
    \begin{table}
      \begin{tabular}{l|c|c|c|}
        \multicolumn{1}{c}{} & \multicolumn{1}{c}{L} & \multicolumn{1}{c}{M} & \multicolumn{1}{c}{H} \\\cline{2-4}
        L & 10, 10 & 3, 15 & 0, 7 \\\cline{2-4}
        M & 15, 3 & 7, 7 & -4, 5 \\\cline{2-4}
        H & 7, 0 & 5, -4 & -15, -15 \\\cline{2-4}
      \end{tabular}
    \end{table}
    \begin{itemize}
      \item[(b)] A necessary (but not sufficient) condition for a SPNE is $x_1 = x_2 = M$. Explain why.
    \end{itemize}
    \begin{multicols}{2}
      \begin{itemize}
        \item[(Step a)] Find the PSNE in the stage game $G$.
      \end{itemize}
      \vfill\null\columnbreak
      Information so far:
      \begin{enumerate}
        \item Trigger strategy for Player $i\in1,2$: "If $t=1$ or if the outcome in all previous stages was $(L,L)$, play $L$. Otherwise, play $x_i$."
      \end{enumerate}
      \vfill\null
    \end{multicols}
\end{frame}
\begin{frame}{PS7, Ex. 8.b: Trigger strategy (infinitely repeated game)}
    Consider $G(\infty)$, i.e. the infinitely repeated game with stage game $G$: \vspace{-6pt}
    \begin{table}
      \begin{tabular}{l|c|c|c|}
        \multicolumn{1}{c}{} & \multicolumn{1}{c}{L} & \multicolumn{1}{c}{M} & \multicolumn{1}{c}{H} \\\cline{2-4}
        L & 10, 10 & 3, \textcolor{blue}{15} & \textcolor{red}{0}, 7 \\\cline{2-4}
        M & \textcolor{red}{15}, 3 & \textcolor{red}{7}, \textcolor{blue}{7} & -4, 5 \\\cline{2-4}
        H & 7, \textcolor{blue}{0} & 5, -4 & -15, -15 \\\cline{2-4}
      \end{tabular}
    \end{table}
    \begin{itemize}
      \item[(b)] A necessary (but not sufficient) condition for a SPNE is $x_1 = x_2 = M$. Explain why.
    \end{itemize}
    \begin{multicols}{2}
      \begin{itemize}
        \item[(Step a)] Find the PSNE in the stage game $G$.
      \end{itemize}
      \vfill\null\columnbreak
      Information so far:
      \begin{enumerate}
        \item Trigger strategy for Player $i\in1,2$: "If $t=1$ or if the outcome in all previous stages was $(L,L)$, play $L$. Otherwise, play $x_i$."
        \item Stage game NE: $(M,M)$.
      \end{enumerate}
      \vfill\null
    \end{multicols}
\end{frame}
\begin{frame}{PS7, Ex. 8.b: Trigger strategy (infinitely repeated game)}
    Consider $G(\infty)$, i.e. the infinitely repeated game with stage game $G$: \vspace{-6pt}
    \begin{table}
      \begin{tabular}{l|c|c|c|}
        \multicolumn{1}{c}{} & \multicolumn{1}{c}{L} & \multicolumn{1}{c}{M} & \multicolumn{1}{c}{H} \\\cline{2-4}
        L & 10, 10 & 3, \textcolor{blue}{15} & \textcolor{red}{0}, 7 \\\cline{2-4}
        M & \textcolor{red}{15}, 3 & \textcolor{red}{7}, \textcolor{blue}{7} & -4, 5 \\\cline{2-4}
        H & 7, \textcolor{blue}{0} & 5, -4 & -15, -15 \\\cline{2-4}
      \end{tabular}
    \end{table}
    \begin{itemize}
      \item[(b)] A necessary (but not sufficient) condition for a SPNE is $x_1 = x_2 = M$. Explain why.
    \end{itemize}
    \begin{multicols}{2}
      \begin{itemize}
        \item[(Step a)] Find the PSNE in the stage game $G$.
        \item[(Step b)] Explain.
      \end{itemize}
      \vfill\null\columnbreak
      Information so far:
      \begin{enumerate}
        \item Trigger strategy for Player $i\in1,2$: "If $t=1$ or if the outcome in all previous stages was $(L,L)$, play $L$. Otherwise, play $x_i$."
        \item Stage game NE: $(M,M)$.
      \end{enumerate}
      \vfill\null
    \end{multicols}
\end{frame}
\begin{frame}{PS7, Ex. 8.b: Trigger strategy (infinitely repeated game)}
    Consider $G(\infty)$, i.e. the infinitely repeated game with stage game $G$: \vspace{-6pt}
    \begin{table}
      \begin{tabular}{l|c|c|c|}
        \multicolumn{1}{c}{} & \multicolumn{1}{c}{L} & \multicolumn{1}{c}{M} & \multicolumn{1}{c}{H} \\\cline{2-4}
        L & 10, 10 & 3, \textcolor{blue}{15} & \textcolor{red}{0}, 7 \\\cline{2-4}
        M & \textcolor{red}{15}, 3 & \textcolor{red}{7}, \textcolor{blue}{7} & -4, 5 \\\cline{2-4}
        H & 7, \textcolor{blue}{0} & 5, -4 & -15, -15 \\\cline{2-4}
      \end{tabular}
    \end{table}
    \begin{itemize}
      \item[(b)] A necessary (but not sufficient) condition for a SPNE is $x_1 = x_2 = M$. Explain why.
    \end{itemize}
    \begin{multicols}{2}
      \begin{itemize}
        \item[(Step a)] Find the PSNE in the stage game $G$.
        \item[(Step b)] Explain.
      \end{itemize}
      For a trigger strategy to constitute a SPNE, the threat of (eternal and unchangeable) punishment must be credible, i.e. must be a stage game NE.\\\medskip
      Thus, $x_1 = x_2 = M$ is a necessary (but not sufficient) condition for the trigger strategies to constitute a SPNE.
      \vfill\null\columnbreak
      Information so far:
      \begin{enumerate}
        \item Trigger strategy for Player $i\in1,2$: "If $t=1$ or if the outcome in all previous stages was $(L,L)$, play $L$. Otherwise, play $x_i$."
        \item Unique stage game NE: $(M,M)$.
      \end{enumerate}
      \vfill\null
    \end{multicols}
\end{frame}

\begin{frame}{PS7, Ex. 8.c: Trigger strategy (infinitely repeated game)}
    Consider $G(\infty)$, i.e. the infinitely repeated game with stage game $G$: \vspace{-6pt}
    \begin{table}
      \begin{tabular}{l|c|c|c|}
        \multicolumn{1}{c}{} & \multicolumn{1}{c}{L} & \multicolumn{1}{c}{M} & \multicolumn{1}{c}{H} \\\cline{2-4}
        L & 10, 10 & 3, \textcolor{blue}{15} & \textcolor{red}{0}, 7 \\\cline{2-4}
        M & \textcolor{red}{15}, 3 & \textcolor{red}{7}, \textcolor{blue}{7} & -4, 5 \\\cline{2-4}
        H & 7, \textcolor{blue}{0} & 5, -4 & -15, -15 \\\cline{2-4}
      \end{tabular}
    \end{table}
    \begin{itemize}
      \vspace{-4pt} \item[(c)] Suppose $\delta = 4/7$. Show by finding a profitable deviation that the above trigger strategy is not a SPNE. \vspace{-6pt}
    \end{itemize}
    \begin{multicols}{2}
      \vfill\null\columnbreak
      Information so far:
      \begin{enumerate}
        \item Trigger Strategy (TS): "If $t=1$ or if the outcome in all previous stages was $(L,L)$, play $L$. If not, play $M$."
      \end{enumerate}
      \vfill\null
    \end{multicols}
\end{frame}
\begin{frame}{PS7, Ex. 8.c: Trigger strategy (infinitely repeated game)}
    Consider $G(\infty)$, i.e. the infinitely repeated game with stage game $G$: \vspace{-6pt}
    \begin{table}
      \begin{tabular}{l|c|c|c|}
        \multicolumn{1}{c}{} & \multicolumn{1}{c}{L} & \multicolumn{1}{c}{M} & \multicolumn{1}{c}{H} \\\cline{2-4}
        L & 10, 10 & 3, \textcolor{blue}{15} & \textcolor{red}{0}, 7 \\\cline{2-4}
        M & \textcolor{red}{15}, 3 & \textcolor{red}{7}, \textcolor{blue}{7} & -4, 5 \\\cline{2-4}
        H & 7, \textcolor{blue}{0} & 5, -4 & -15, -15 \\\cline{2-4}
      \end{tabular}
    \end{table}
    \begin{itemize}
      \vspace{-4pt} \item[(c)] Suppose $\delta = 4/7$. Show by finding a profitable deviation that the above trigger strategy is not a SPNE. \vspace{-6pt}
    \end{itemize}
    \begin{multicols}{2}
      \begin{itemize}
        \item[(Step a)] Given Player $j$ plays the Trigger Strategy (TS), write up Player $i$'s Optimal Deviation Strategy (ODS).
      \end{itemize}
      \vfill\null\columnbreak
      Information so far:
      \begin{enumerate}
        \item Trigger Strategy (TS): "If $t=1$ or if the outcome in all previous stages was $(L,L)$, play $L$. If not, play $M$."
      \end{enumerate}
      \vfill\null
    \end{multicols}
\end{frame}
\begin{frame}{PS7, Ex. 8.c: Trigger strategy (infinitely repeated game)}
    Consider $G(\infty)$, i.e. the infinitely repeated game with stage game $G$: \vspace{-6pt}
    \begin{table}
      \begin{tabular}{l|c|c|c|}
        \multicolumn{1}{c}{} & \multicolumn{1}{c}{L} & \multicolumn{1}{c}{M} & \multicolumn{1}{c}{H} \\\cline{2-4}
        L & 10, 10 & 3, \textcolor{blue}{15} & \textcolor{red}{0}, 7 \\\cline{2-4}
        M & \textcolor{red}{15}, 3 & \textcolor{red}{7}, \textcolor{blue}{7} & -4, 5 \\\cline{2-4}
        H & 7, \textcolor{blue}{0} & 5, -4 & -15, -15 \\\cline{2-4}
      \end{tabular}
    \end{table}
    \begin{itemize}
      \vspace{-4pt} \item[(c)] Suppose $\delta = 4/7$. Show by finding a profitable deviation that the above trigger strategy is not a SPNE. \vspace{-6pt}
    \end{itemize}
    \begin{multicols}{2}
      \begin{itemize}
        \item[(Step a)] Given Player $j$ plays the Trigger Strategy (TS), write up Player $i$'s Optimal Deviation Strategy (ODS).
      \end{itemize}
      \vfill\null\columnbreak
      Information so far:
      \begin{enumerate}
        \item Trigger Strategy (TS): "If $t=1$ or if the outcome in all previous stages was $(L,L)$, play $L$. If not, play $M$."
        \item Optimal Deviation Strategy (ODS): "Always play $M$."
      \end{enumerate}
      \vfill\null
    \end{multicols}
\end{frame}
\begin{frame}{PS7, Ex. 8.c: Trigger strategy (infinitely repeated game)}
    Consider $G(\infty)$, i.e. the infinitely repeated game with stage game $G$: \vspace{-6pt}
    \begin{table}
      \begin{tabular}{l|c|c|c|}
        \multicolumn{1}{c}{} & \multicolumn{1}{c}{L} & \multicolumn{1}{c}{M} & \multicolumn{1}{c}{H} \\\cline{2-4}
        L & 10, 10 & 3, \textcolor{blue}{15} & \textcolor{red}{0}, 7 \\\cline{2-4}
        M & \textcolor{red}{15}, 3 & \textcolor{red}{7}, \textcolor{blue}{7} & -4, 5 \\\cline{2-4}
        H & 7, \textcolor{blue}{0} & 5, -4 & -15, -15 \\\cline{2-4}
      \end{tabular}
    \end{table}
    \begin{itemize}
      \vspace{-4pt} \item[(c)] Suppose $\delta = 4/7$. Show by finding a profitable deviation that the above trigger strategy is not a SPNE. \vspace{-6pt}
    \end{itemize}
    \begin{multicols}{2}
      \begin{itemize}
        \item[(Step a)] Given Player $j$ plays the Trigger Strategy (TS), write up Player $i$'s Optimal Deviation Strategy (ODS).
        \item[(Step b)] Given Player $j$ plays TS, write up Player $i$'s respective payoffs from playing TS and ODS.
      \end{itemize}
      \vfill\null\columnbreak
      Information so far:
      \begin{enumerate}
        \item Trigger Strategy (TS): "If $t=1$ or if the outcome in all previous stages was $(L,L)$, play $L$. If not, play $M$."
        \item Optimal Deviation Strategy (ODS): "Always play $M$."
      \end{enumerate}
      \vfill\null
    \end{multicols}
\end{frame}
\begin{frame}{PS7, Ex. 8.c: Trigger strategy (infinitely repeated game)}
    Consider $G(\infty)$, i.e. the infinitely repeated game with stage game $G$: \vspace{-6pt}
    \begin{table}
      \begin{tabular}{l|c|c|c|}
        \multicolumn{1}{c}{} & \multicolumn{1}{c}{L} & \multicolumn{1}{c}{M} & \multicolumn{1}{c}{H} \\\cline{2-4}
        L & 10, 10 & 3, \textcolor{blue}{15} & \textcolor{red}{0}, 7 \\\cline{2-4}
        M & \textcolor{red}{15}, 3 & \textcolor{red}{7}, \textcolor{blue}{7} & -4, 5 \\\cline{2-4}
        H & 7, \textcolor{blue}{0} & 5, -4 & -15, -15 \\\cline{2-4}
      \end{tabular}
    \end{table}
    \begin{itemize}
      \vspace{-4pt} \item[(c)] Suppose $\delta = 4/7$. Show by finding a profitable deviation that the above trigger strategy is not a SPNE. \vspace{-6pt}
    \end{itemize}
    \begin{multicols}{2}
      \begin{itemize}
        \item[(Step a)] Given Player $j$ plays the Trigger Strategy (TS), write up Player $i$'s Optimal Deviation Strategy (ODS).
        \item[(Step b)] Given Player $j$ plays TS, write up Player $i$'s respective payoffs from playing TS and ODS.
      \end{itemize}
      Player $i$'s payoff from playing TS:
      \begin{align*}
        10+10\delta+10\delta^2+...=\sum_{t=1}^\infty10\delta^{t-1}=\frac{10}{1-\delta}
      \end{align*}
      \vfill\null\columnbreak
      Information so far:
      \begin{enumerate}
        \item Trigger Strategy (TS): "If $t=1$ or if the outcome in all previous stages was $(L,L)$, play $L$. If not, play $M$."
        \item Optimal Deviation Strategy (ODS): "Always play $M$."
        \item $U_i(TS,TS)=\frac{10}{1-\delta}$.
      \end{enumerate}
      \vfill\null
    \end{multicols}
\end{frame}
\begin{frame}{PS7, Ex. 8.c: Trigger strategy (infinitely repeated game)}
    Consider $G(\infty)$, i.e. the infinitely repeated game with stage game $G$: \vspace{-6pt}
    \begin{table}
      \begin{tabular}{l|c|c|c|}
        \multicolumn{1}{c}{} & \multicolumn{1}{c}{L} & \multicolumn{1}{c}{M} & \multicolumn{1}{c}{H} \\\cline{2-4}
        L & 10, 10 & 3, \textcolor{blue}{15} & \textcolor{red}{0}, 7 \\\cline{2-4}
        M & \textcolor{red}{15}, 3 & \textcolor{red}{7}, \textcolor{blue}{7} & -4, 5 \\\cline{2-4}
        H & 7, \textcolor{blue}{0} & 5, -4 & -15, -15 \\\cline{2-4}
      \end{tabular}
    \end{table}
    \begin{itemize}
      \vspace{-4pt} \item[(c)] Suppose $\delta = 4/7$. Show by finding a profitable deviation that the above trigger strategy is not a SPNE. \vspace{-6pt}
    \end{itemize}
    \begin{multicols}{2}
      \begin{itemize}
        \item[(Step a)] Given Player $j$ plays the Trigger Strategy (TS), write up Player $i$'s Optimal Deviation Strategy (ODS).
        \item[(Step b)] Given Player $j$ plays TS, write up Player $i$'s respective payoffs from playing TS and ODS.
      \end{itemize}
      Player $i$'s payoff from playing TS:
      \vspace{-4pt}
      \begin{align*}
        10+10\delta+10\delta^2+...=\sum_{t=1}^\infty10\delta^{t-1}=\frac{10}{1-\delta}
      \end{align*}
      Player $i$'s payoff from playing ODS:
      \vspace{-4pt}
      \begin{align*}
        15+7\delta+7\delta^2+...=15+\sum_{\bm{t=2}}^\infty7\delta^{t-1}=15+\frac{7\delta}{1-\delta}
      \end{align*}
      \vfill\null\columnbreak
      Information so far:
      \begin{enumerate}
        \item Trigger Strategy (TS): "If $t=1$ or if the outcome in all previous stages was $(L,L)$, play $L$. If not, play $M$."
        \item Optimal Deviation Strategy (ODS): "Always play $M$."
        \item $U_i(TS,TS)=\frac{10}{1-\delta}$
        \item $U_i(ODS,TS)=15+\frac{7\delta}{1-\delta}$
      \end{enumerate}
      \vfill\null
    \end{multicols}
\end{frame}
\begin{frame}{PS7, Ex. 8.c: Trigger strategy (infinitely repeated game)}
    Consider $G(\infty)$, i.e. the infinitely repeated game with stage game $G$: \vspace{-6pt}
    \begin{table}
      \begin{tabular}{l|c|c|c|}
        \multicolumn{1}{c}{} & \multicolumn{1}{c}{L} & \multicolumn{1}{c}{M} & \multicolumn{1}{c}{H} \\\cline{2-4}
        L & 10, 10 & 3, \textcolor{blue}{15} & \textcolor{red}{0}, 7 \\\cline{2-4}
        M & \textcolor{red}{15}, 3 & \textcolor{red}{7}, \textcolor{blue}{7} & -4, 5 \\\cline{2-4}
        H & 7, \textcolor{blue}{0} & 5, -4 & -15, -15 \\\cline{2-4}
      \end{tabular}
    \end{table}
    \begin{itemize}
      \vspace{-4pt} \item[(c)] Suppose $\delta = 4/7$. Show by finding a profitable deviation that the above trigger strategy is not a SPNE. \vspace{-6pt}
    \end{itemize}
    \begin{multicols}{2}
      \begin{itemize}
        \item[(Step a)] Given Player $j$ plays the Trigger Strategy (TS), write up Player $i$'s Optimal Deviation Strategy (ODS).
        \item[(Step b)] Given Player $j$ plays TS, write up Player $i$'s respective payoffs from playing TS and ODS.
        \item[(Step c)] Show that the deviation is preferred for $\delta=4/7$.
      \end{itemize}
      \vfill\null\columnbreak
      Information so far:
      \begin{enumerate}
        \item Trigger Strategy (TS): "If $t=1$ or if the outcome in all previous stages was $(L,L)$, play $L$. If not, play $M$."
        \item Optimal Deviation Strategy (ODS): "Always play $M$."
        \item $U_i(TS,TS)=\frac{10}{1-\delta}$
        \item $U_i(ODS,TS)=15+\frac{7\delta}{1-\delta}$
      \end{enumerate}
      \vfill\null
    \end{multicols}
\end{frame}
\begin{frame}{PS7, Ex. 8.c: Trigger strategy (infinitely repeated game)}
    Consider $G(\infty)$, i.e. the infinitely repeated game with stage game $G$: \vspace{-6pt}
    \begin{table}
      \begin{tabular}{l|c|c|c|}
        \multicolumn{1}{c}{} & \multicolumn{1}{c}{L} & \multicolumn{1}{c}{M} & \multicolumn{1}{c}{H} \\\cline{2-4}
        L & 10, 10 & 3, \textcolor{blue}{15} & \textcolor{red}{0}, 7 \\\cline{2-4}
        M & \textcolor{red}{15}, 3 & \textcolor{red}{7}, \textcolor{blue}{7} & -4, 5 \\\cline{2-4}
        H & 7, \textcolor{blue}{0} & 5, -4 & -15, -15 \\\cline{2-4}
      \end{tabular}
    \end{table}
    \begin{itemize}
      \vspace{-4pt} \item[(c)] Suppose $\delta = 4/7$. Show by finding a profitable deviation that the above trigger strategy is not a SPNE. \vspace{-6pt}
    \end{itemize}
    \begin{multicols}{2}
      \begin{itemize}
        \item[(Step a)] Given Player $j$ plays the Trigger Strategy (TS), write up Player $i$'s Optimal Deviation Strategy (ODS).
        \item[(Step b)] Given Player $j$ plays TS, write up Player $i$'s respective payoffs from playing TS and ODS.
        \item[(Step c)] Show that the deviation is preferred for $\delta=4/7$:
      \end{itemize}
      \vspace{-8pt}
      \begin{align*}
        U_i\left(ODS,TS\right)&>U_i\left(TS,TS\right)\\
        \Rightarrow15+\frac{7\frac{4}{7}}{1-\frac{4}{7}}&>\frac{10}{1-\frac{4}{7}},&&\text{for }\delta=\frac{4}{7}\\
        \Rightarrow\frac{73}{3}&>\frac{70}{3}&&\textit{Q.E.D.}
      \end{align*}
      \vfill\null\columnbreak
      Information so far:
      \begin{enumerate}
        \item Trigger Strategy (TS): "If $t=1$ or if the outcome in all previous stages was $(L,L)$, play $L$. If not, play $M$."
        \item Optimal Deviation Strategy (ODS): "Always play $M$."
        \item $U_i(TS,TS)=\frac{10}{1-\delta}$
        \item $U_i(ODS,TS)=15+\frac{7\delta}{1-\delta}$
      \end{enumerate}
      \vfill\null
    \end{multicols}
\end{frame}



\section{PS7, Ex. 9: Optimal punishment strategy (infinitely repeated game)}

\begin{frame}{PS7, Ex. 9: Optimal punishment strategy (infinitely repeated game)}
    We continue analyzing $G(\infty)$. As in \sout{Lecture 8} (Lecture 6, slides 50-68), consider the strategy profile $(OP,OP)$, where $OP$ stands for optimal punishment...\\\medskip
    \textit{[See the lecture slides and the full description of the exercise in the problem set.]}\\\medskip
    Stage game $G$:\vspace{-6pt}
    \begin{table}
      \begin{tabular}{l|c|c|c|}
        \multicolumn{1}{c}{} & \multicolumn{1}{c}{L} & \multicolumn{1}{c}{M} & \multicolumn{1}{c}{H} \\\cline{2-4}
        L & 10, 10 & 3, \textcolor{blue}{15} & \textcolor{red}{0}, 7 \\\cline{2-4}
        M & \textcolor{red}{15}, 3 & \textcolor{red}{7}, \textcolor{blue}{7} & -4, 5 \\\cline{2-4}
        H & 7, \textcolor{blue}{0} & 5, -4 & -15, -15 \\\cline{2-4}
      \end{tabular}
    \end{table}
\end{frame}

\begin{frame}{PS7, Ex. 9.a: Optimal punishment strategy (infinitely repeated game)}
    Consider $G(\infty)$, i.e. the infinitely repeated game with stage game $G$: \vspace{-6pt}
    \begin{table}
      \begin{tabular}{l|c|c|c|}
        \multicolumn{1}{c}{} & \multicolumn{1}{c}{L} & \multicolumn{1}{c}{M} & \multicolumn{1}{c}{H} \\\cline{2-4}
        L & 10, 10 & 3, \textcolor{blue}{15} & \textcolor{red}{0}, 7 \\\cline{2-4}
        M & \textcolor{red}{15}, 3 & \textcolor{red}{7}, \textcolor{blue}{7} & -4, 5 \\\cline{2-4}
        H & 7, \textcolor{blue}{0} & 5, -4 & -15, -15 \\\cline{2-4}
      \end{tabular}
    \end{table}
    \vspace{-4pt}
    \begin{itemize}
      \item[(a)] Discuss how this increased leniency over time may give player 1 an incentive to accept his punishment (and actually play according to $Q_D$, rather than deviate again).
    \end{itemize}
    \vfill\null
\end{frame}
\begin{frame}{PS7, Ex. 9.a: Optimal punishment strategy (infinitely repeated game)}
    Consider $G(\infty)$ with stage game $G$, underlining $(Q^D,Q^P)$ in the 'tough' stage: \vspace{-6pt}
    \begin{table}
      \begin{tabular}{l|c|c|c|}
        \multicolumn{1}{c}{} & \multicolumn{1}{c}{L} & \multicolumn{1}{c}{M} & \multicolumn{1}{c}{\textbf{\underline{H}}} \\\cline{2-4}
        L & 10, 10 & 3, \textcolor{blue}{15} & \textcolor{red}{0}, 7 \\\cline{2-4}
        \textbf{\underline{M}} & \textcolor{red}{15}, 3 & \textcolor{red}{7}, \textcolor{blue}{7} & \textbf{\underline{-4, 5}} \\\cline{2-4}
        H & 7, \textcolor{blue}{0} & 5, -4 & -15, -15 \\\cline{2-4}
      \end{tabular}
    \end{table}
    \vspace{-4pt}
    \begin{itemize}
      \item[(a)] Discuss how this increased leniency over time may give player 1 an incentive to accept his punishment (and actually play according to $Q^D$, rather than deviate again).
    \end{itemize}
    \vspace{-8pt}
    \begin{multicols}{2}
      The 'tough' stage (the \nth{1} round of punishment):\vspace{-4pt}
      \begin{itemize}
        \item P1 earns -4 from $(M, H)$ and 3 from $(L, M)$ in all later rounds.
        \item He can deviate by playing $L$ and earn 0, but then the punishment will start over again and P1 will therefore stay in the 'tough' stage.
        \item Both -4 and 0 is less than 3, which is why this punishment structure leaves P1 without an incentive to deviate from the 'punishment path' $(Q^D,Q^P)$.
      \end{itemize}
      \vfill\null\columnbreak
      \vfill\null
    \end{multicols}
\end{frame}
\begin{frame}{PS7, Ex. 9.a: Optimal punishment strategy (infinitely repeated game)}
    Consider $G(\infty)$ with stage game $G$, underlining $(Q^D,Q^P)$ in the 'mild' stage: \vspace{-6pt}
    \begin{table}
      \begin{tabular}{l|c|c|c|}
        \multicolumn{1}{c}{} & \multicolumn{1}{c}{L} & \multicolumn{1}{c}{\textbf{\underline{M}}} & \multicolumn{1}{c}{H} \\\cline{2-4}
        \textbf{\underline{L}} & 10, 10 & \textbf{\underline{3, \textcolor{blue}{15}}} & \textcolor{red}{0}, 7 \\\cline{2-4}
        M & \textcolor{red}{15}, 3 & \textcolor{red}{7}, \textcolor{blue}{7} & -4, 5 \\\cline{2-4}
        H & 7, \textcolor{blue}{0} & 5, -4 & -15, -15 \\\cline{2-4}
      \end{tabular}
    \end{table}
    \vspace{-6pt}
    \begin{itemize}
      \item[(a)] Discuss how this increased leniency over time may give player 1 an incentive to accept his punishment (and actually play according to $Q^D$, rather than deviate again).
    \end{itemize}
    \vspace{-8pt}
    \begin{multicols}{2}
      The 'tough' stage (the \nth{1} round of punishment):\vspace{-6pt}
      \begin{itemize}
        \item P1 earns -4 from $(M, H)$ and 3 from $(L, M)$ in all later rounds.
        \item He can deviate by playing $L$ and earn 0, but then the punishment will start over again and P1 will therefore stay in the 'tough' stage.
        \item Both -4 and 0 is less than 3, which is why this punishment structure leaves P1 without an incentive to deviate from the 'punishment path' $(Q^D,Q^P)$.
      \end{itemize}
      \vfill\null\columnbreak
      The 'mild' stage (from the \nth{2} round of punishment):\vspace{-6pt}
      \begin{itemize}
        \item P1 earns 3 from $(L, M)$ in this and all subsequent rounds.
        \item He can deviate by playing $M$ and earn 7, but then the punishment will start over again and P1 will earn -4 in the 'tough' stage (or alternatively 0 by deviating again).
        \item Both -4 and 0 is less than 3, which is why this punishment structure provides P1 with an incentive to stay in the 'mild stage'.
      \end{itemize}
      \vfill\null
    \end{multicols}
\end{frame}

\begin{frame}{PS7, Ex. 9.b: Optimal punishment strategy (infinitely repeated game)}
    \begin{multicols}{2}
      The 'tough' stage (the \nth{1} round of punishment):\vspace{-4pt}
      \begin{itemize}
        \item P1 earns -4 from $(M, H)$ and 3 from $(L, M)$ in all later rounds.
        \item He can deviate by playing $L$ and earn 0, but then the punishment will start over again and P1 will therefore stay in the 'tough' stage.
        \item Both -4 and 0 is less than 3, which is why this punishment structure leaves P1 without an incentive to deviate from the punishment path.
      \end{itemize}
      \vfill\null\columnbreak
      The 'mild' stage (from the \nth{2} round of punishment):\vspace{-4pt}
      \begin{itemize}
        \item P1 earns 3 from $(L, M)$ in this and all subsequent rounds.
        \item He can deviate by playing $M$ and earn 7, but then the punishment will start over again and P1 will earn -4 in the 'tough' stage (or alternatively 0 by deviating again).
        \item Both -4 and 0 is less than 3, which is why this punishment structure provides P1 with an incentive to stay in the 'mild stage'.
      \end{itemize}
      \vfill\null
    \end{multicols}
    \vspace{-24pt}
    \begin{itemize}
      \item[(b)] \textbf{\textit{How does your answer relate to the following quote from Wikipedia?}}
      \item[] \textit{The “carrot and stick” approach is an idiom that refers to a policy of offering a combination of rewards and punishment to induce behavior. It is named in reference to a cart driver dangling a carrot in front of a mule and holding a stick behind it. The mule would move towards the carrot because it wants the reward of food, while also moving away from the stick behind it, since it does not want the punishment of pain, thus drawing the cart.}
    \end{itemize}
\end{frame}
\begin{frame}{PS7, Ex. 9.b: Optimal punishment strategy (infinitely repeated game)}
    \begin{multicols}{2}
      The 'tough' stage (the \nth{1} round of punishment):\vspace{-6pt}
      \begin{itemize}
        \item P1 earns -4 from $(M, H)$ and 3 from $(L, M)$ in all later rounds.
        \item He can deviate by playing $L$ and earn 0, but then the punishment will start over again and P1 will therefore stay in the 'tough' stage.
        \item Both -4 and 0 is less than 3, which is why this punishment structure leaves P1 without an incentive to deviate from the punishment path.
      \end{itemize}
      \vfill\null\columnbreak
      The 'mild' stage (from the \nth{2} round of punishment):\vspace{-6pt}
      \begin{itemize}
        \item P1 earns 3 from $(L, M)$ in this and all subsequent rounds.
        \item He can deviate by playing $M$ and earn 7, but then the punishment will start over again and P1 will earn -4 in the 'tough' stage (or alternatively 0 by deviating again).
        \item Both -4 and 0 is less than 3, which is why this punishment structure provides P1 with an incentive to stay in the 'mild stage'.
      \end{itemize}
      \vfill\null
    \end{multicols}
    \vspace{-28pt}
    \begin{itemize}
      \item[(b)] How does your answer relate to the following quote from Wikipedia?
      \item[] \textit{The “carrot and stick” approach is an idiom that refers to a policy of offering a combination of rewards and punishment to induce behavior. It is named in reference to a cart driver dangling a carrot in front of a mule and holding a stick behind it. The mule would move towards the carrot because it wants the reward of food, while also moving away from the stick behind it, since it does not want the punishment of pain, thus drawing the cart.}
      \item[] \textbf{\textit{In other words: Which stage is \textcolor{orange}{"the carrot"} and which is \textcolor{brown}{"the stick"}? Explain.}}
    \end{itemize}
\end{frame}
\begin{frame}{PS7, Ex. 9.b: Optimal punishment strategy (infinitely repeated game)}
    \begin{itemize}
      \item[(b)] How does your answer relate to the following quote from Wikipedia?
      \item[] \textit{The “carrot and stick” approach is an idiom that refers to a policy of offering a combination of rewards and punishment to induce behavior. It is named in reference to a cart driver dangling a carrot in front of a mule and holding a stick behind it. The mule would move towards the carrot because it wants the reward of food, while also moving away from the stick behind it, since it does not want the punishment of pain, thus drawing the cart.}
    \end{itemize}
    \vspace{-4pt}
    \begin{multicols}{2}
      The 'tough' stage (the \nth{1} round of punishment):
      \begin{itemize}
        \item Is the \textcolor{brown}{"the stick"} that harshly punishes any deviation. During the 'mild' stage, the threat of a future 'tough punishment' should discourage deviation from the 'punishment path'.
      \end{itemize}
      \vfill\null\columnbreak
      The 'mild' stage (from the \nth{2} round of punishment):
      \vfill\null
    \end{multicols}
\end{frame}
\begin{frame}{PS7, Ex. 9.b: Optimal punishment strategy (infinitely repeated game)}
    \begin{itemize}
      \item[(b)] How does your answer relate to the following quote from Wikipedia?
      \item[] \textit{The “carrot and stick” approach is an idiom that refers to a policy of offering a combination of rewards and punishment to induce behavior. It is named in reference to a cart driver dangling a carrot in front of a mule and holding a stick behind it. The mule would move towards the carrot because it wants the reward of food, while also moving away from the stick behind it, since it does not want the punishment of pain, thus drawing the cart.}
    \end{itemize}
    \vspace{-4pt}
    \begin{multicols}{2}
      The 'tough' stage (the \nth{1} round of punishment):
      \begin{itemize}
        \item Is the \textcolor{brown}{"the stick"} that harshly punishes any deviation. During the 'mild' stage, the threat of a future 'tough punishment' should discourage deviation from the 'punishment path'.
      \end{itemize}
      \vfill\null\columnbreak
      The 'mild' stage (from the \nth{2} round of punishment):
      \begin{itemize}
        \item Is \textcolor{orange}{"the carrot"} as the promise of a future 'mild punishment' should be regarded as a reward for accepting the punishment in the 'tough' stage without deviating from the 'punishment path'.
      \end{itemize}
      % In the 'tough' stage (the \nth{1} round of punishment):
      % \begin{itemize}
      %   \item The promise of a future 'mild punishment' is \textcolor{orange}{the carrot} that should be regarded as a reward for accepting the punishment without deviating from the 'punishment path'.
      % \end{itemize}
      % \vfill\null\columnbreak
      % In the 'mild' stage (from the \nth{2} round of punishment):
      % \begin{itemize}
      %   \item The threat of a future 'tough punishment' is \textcolor{brown}{the stick} that should discourage any further deviation while being a direct punishment for deviating in the first place.
      % \end{itemize}
      \vfill\null
    \end{multicols}
\end{frame}



\section{PS7, Ex. 10: Is the punishment credible? (infinitely repeated game)}

\begin{frame}{PS7, Ex. 10: Is the punishment credible? (infinitely repeated game)}
    We continue analyzing $G(\infty)$. Complete the proof that $(OP,OP)$ is a SPNE when $\delta=4/7$. We checked the first three points of the road map in Lecture 6 (slide 56). The last two points consist of checking that it is optimal for Player 2 to punish Player 1 after a deviation. In particular, you need to check that Player 2 has no profitable deviation when he is in the first and in the second round of punishing Player 1.\\\bigskip
    \textit{[The roadmap is summed up on the next two slides]}
    \vfill\null
\end{frame}
\begin{frame}{PS7, Ex. 10: Is the punishment credible? (infinitely repeated game)}
    We continue analyzing $G(\infty)$. Complete the proof that $(OP,OP)$ is a SPNE when $\delta=4/7$. We checked the first three points of the road map in Lecture 6 (slide 56). The last two points consist of checking that it is optimal for Player 2 to punish Player 1 after a deviation. In particular, you need to check that Player 2 has no profitable deviation when he is in the first and in the second round of punishing Player 1.
    \begin{multicols}{2}
      In the lecture it was checked that Player 1 will not deviate from $(OP,OP)$ in:
      \begin{enumerate}
        \item Round 1, or if $(L,L)$ was played in all previous rounds.
        \item The \nth{1} round of being punished.
        \item Subsequent rounds of being punished.
      \end{enumerate}
      \vfill\null\columnbreak
      Underlining $(Q^D,Q^P)$ in the \nth{1} round of punishment (the 'tough' stage):
      \vspace{-6pt}
      \begin{table}
        \begin{tabular}{l|c|c|c|}
          \multicolumn{1}{c}{} & \multicolumn{1}{c}{L} & \multicolumn{1}{c}{M} & \multicolumn{1}{c}{\textbf{\underline{H}}} \\\cline{2-4}
          L & 10, 10 & 3, \textcolor{blue}{15} & \textcolor{red}{0}, 7 \\\cline{2-4}
          \textbf{\underline{M}} & \textcolor{red}{15}, 3 & \textcolor{red}{7}, \textcolor{blue}{7} & \textbf{\underline{-4, 5}} \\\cline{2-4}
          H & 7, \textcolor{blue}{0} & 5, -4 & -15, -15 \\\cline{2-4}
        \end{tabular}
      \end{table}
      Underlining $(Q^D,Q^P)$ in subsequent rounds of punishment (the 'mild' stage):
      \vspace{-6pt}
      \begin{table}
        \begin{tabular}{l|c|c|c|}
          \multicolumn{1}{c}{} & \multicolumn{1}{c}{L} & \multicolumn{1}{c}{\textbf{\underline{M}}} & \multicolumn{1}{c}{H} \\\cline{2-4}
          \textbf{\underline{L}} & 10, 10 & \textbf{\underline{3, \textcolor{blue}{15}}} & \textcolor{red}{0}, 7 \\\cline{2-4}
          M & \textcolor{red}{15}, 3 & \textcolor{red}{7}, \textcolor{blue}{7} & -4, 5 \\\cline{2-4}
          H & 7, \textcolor{blue}{0} & 5, -4 & -15, -15 \\\cline{2-4}
        \end{tabular}
      \end{table}
      \vfill\null
    \end{multicols}
    \vfill\null
\end{frame}
\begin{frame}{PS7, Ex. 10: Is the punishment credible? (infinitely repeated game)}
    We continue analyzing $G(\infty)$. Complete the proof that $(OP,OP)$ is a SPNE when $\delta=4/7$. We checked the first three points of the road map in Lecture 6 (slide 56). The last two points consist of checking that it is optimal for Player 2 to punish Player 1 after a deviation. In particular, you need to check that Player 2 has no profitable deviation when he is in the first and in the second round of punishing Player 1.
    \begin{multicols}{2}
      In the lecture it was checked that Player 1 will not deviate from $(OP,OP)$ in:
      \begin{enumerate}
        \item Round 1, or if $(L,L)$ was played in all previous rounds.
        \item The \nth{1} round of being punished.
        \item Subsequent rounds of being punished.
      \end{enumerate}
      \textbf{\textit{Check that Player 2 will not deviate:}}
      \begin{itemize}
        \item[4.] When he is in the \nth{1} round (the 'tough' stage) of punishing Player 1.
        \item[5.] When he is in subsequent rounds ('mild' stage) of punishing Player 1.
      \end{itemize}
      Remember to use $\delta=4/7$.
      \vfill\null\columnbreak
      Underlining $(Q^D,Q^P)$ in the \nth{1} round of punishment (the 'tough' stage):
      \vspace{-6pt}
      \begin{table}
        \begin{tabular}{l|c|c|c|}
          \multicolumn{1}{c}{} & \multicolumn{1}{c}{L} & \multicolumn{1}{c}{M} & \multicolumn{1}{c}{\textbf{\underline{H}}} \\\cline{2-4}
          L & 10, 10 & 3, \textcolor{blue}{15} & \textcolor{red}{0}, 7 \\\cline{2-4}
          \textbf{\underline{M}} & \textcolor{red}{15}, 3 & \textcolor{red}{7}, \textcolor{blue}{7} & \textbf{\underline{-4, 5}} \\\cline{2-4}
          H & 7, \textcolor{blue}{0} & 5, -4 & -15, -15 \\\cline{2-4}
        \end{tabular}
      \end{table}
      Underlining $(Q^D,Q^P)$ in subsequent rounds of punishment (the 'mild' stage):
      \vspace{-6pt}
      \begin{table}
        \begin{tabular}{l|c|c|c|}
          \multicolumn{1}{c}{} & \multicolumn{1}{c}{L} & \multicolumn{1}{c}{\textbf{\underline{M}}} & \multicolumn{1}{c}{H} \\\cline{2-4}
          \textbf{\underline{L}} & 10, 10 & \textbf{\underline{3, \textcolor{blue}{15}}} & \textcolor{red}{0}, 7 \\\cline{2-4}
          M & \textcolor{red}{15}, 3 & \textcolor{red}{7}, \textcolor{blue}{7} & -4, 5 \\\cline{2-4}
          H & 7, \textcolor{blue}{0} & 5, -4 & -15, -15 \\\cline{2-4}
        \end{tabular}
      \end{table}
      \vfill\null
    \end{multicols}
    \vfill\null
\end{frame}

\begin{frame}{PS7, Ex. 10: Is the punishment credible? (infinitely repeated game)}
    Use the roadmap to complete the proof that $(OP,OP)$ is a SPNE when $\delta=4/7$.\vspace{-4pt}
    \begin{multicols}{2}
      Check that Player 2 will not deviate:
      \begin{itemize}
        \item[4.] When he is in the \nth{1} round (the 'tough' stage) of punishing Player 1.
      \end{itemize}
      \textbf{\textit{First, calculate Player 2's expected utility from sticking to $\bm{Q^P}$ for $\bm{\delta=4/7}$ (i.e. from not deviating).}}
      \vfill\null\columnbreak
      \vspace{-6pt}
      \begin{table}
        \begin{tabular}{l|c|c|c|}
          \multicolumn{1}{c}{} & \multicolumn{1}{c}{L} & \multicolumn{1}{c}{M} & \multicolumn{1}{c}{\textbf{\underline{H}}} \\\cline{2-4}
          L & 10, 10 & 3, \textcolor{blue}{15} & \textcolor{red}{0}, 7 \\\cline{2-4}
          \textbf{\underline{M}} & \textcolor{red}{15}, 3 & \textcolor{red}{7}, \textcolor{blue}{7} & \textbf{\underline{-4, 5}} \\\cline{2-4}
          H & 7, \textcolor{blue}{0} & 5, -4 & -15, -15 \\\cline{2-4}
        \end{tabular}
      \end{table}
      \vfill\null
    \end{multicols}
    \vfill\null
\end{frame}
\begin{frame}{PS7, Ex. 10: Is the punishment credible? (infinitely repeated game)}
    Use the roadmap to complete the proof that $(OP,OP)$ is a SPNE when $\delta=4/7$.\vspace{-4pt}
    \begin{multicols}{2}
      Check that Player 2 will not deviate:
      \begin{itemize}
        \item[4.] When he is in the \nth{1} round (the 'tough' stage) of punishing Player 1.
      \end{itemize}
      \vfill\null\columnbreak
      \vspace{-6pt}
      \begin{table}
        \begin{tabular}{l|c|c|c|}
          \multicolumn{1}{c}{} & \multicolumn{1}{c}{L} & \multicolumn{1}{c}{M} & \multicolumn{1}{c}{\textbf{\underline{H}}} \\\cline{2-4}
          L & 10, 10 & 3, \textcolor{blue}{15} & \textcolor{red}{0}, 7 \\\cline{2-4}
          \textbf{\underline{M}} & \textcolor{red}{15}, 3 & \textcolor{red}{7}, \textcolor{blue}{7} & \textbf{\underline{-4, 5}} \\\cline{2-4}
          H & 7, \textcolor{blue}{0} & 5, -4 & -15, -15 \\\cline{2-4}
        \end{tabular}
      \end{table}
    \end{multicols}
    \vspace{-20pt}
    If Player 2 does not deviate from $Q^P$, his expected utility is:
    \vspace{-6pt}
    \begin{align*}
      U_2(\underbrace{Q^D}_{s_1};\underbrace{Q^P}_{s_2})=5+15\delta+15\delta^2+...
                  =5+\sum_{\bm{t=2}}^\infty15\delta^{t-1}
                  =5+\frac{15\delta}{1-\delta}
                  =25,\text{ for }\delta=\frac{4}{7}
    \end{align*}
    \vfill\null
\end{frame}
\begin{frame}{PS7, Ex. 10: Is the punishment credible? (infinitely repeated game)}
    Use the roadmap to complete the proof that $(OP,OP)$ is a SPNE when $\delta=4/7$.\vspace{-4pt}
    \begin{multicols}{2}
      Check that Player 2 will not deviate:
      \begin{itemize}
        \item[4.] When he is in the \nth{1} round (the 'tough' stage) of punishing Player 1.
      \end{itemize}
      \vfill\null\columnbreak
      \vspace{-6pt}
      \begin{table}
        \begin{tabular}{l|c|c|c|}
          \multicolumn{1}{c}{} & \multicolumn{1}{c}{L} & \multicolumn{1}{c}{M} & \multicolumn{1}{c}{\textbf{\underline{H}}} \\\cline{2-4}
          L & 10, 10 & 3, \textcolor{blue}{15} & \textcolor{red}{0}, 7 \\\cline{2-4}
          \textbf{\underline{M}} & \textcolor{red}{15}, 3 & \textcolor{red}{7}, \textcolor{blue}{7} & \textbf{\underline{-4, 5}} \\\cline{2-4}
          H & 7, \textcolor{blue}{0} & 5, -4 & -15, -15 \\\cline{2-4}
        \end{tabular}
      \end{table}
    \end{multicols}
    \vspace{-20pt}
    If Player 2 does not deviate from $Q^P$, his expected utility is:
    \vspace{-6pt}
    \begin{align*}
      U_2(\underbrace{Q^D}_{s_1};\underbrace{Q^P}_{s_2})=5+15\delta+15\delta^2+...
                  =5+\sum_{\bm{t=2}}^\infty15\delta^{t-1}
                  =5+\frac{15\delta}{1-\delta}
                  =25,\text{ for }\delta=\frac{4}{7}
    \end{align*}
    \textbf{\textit{What is Player 2's expected utility from his best possible deviation when he is in the \nth{1} round of punishing Player 1?}}
    \vfill\null
\end{frame}
\begin{frame}{PS7, Ex. 10: Is the punishment credible? (infinitely repeated game)}
    Use the roadmap to complete the proof that $(OP,OP)$ is a SPNE when $\delta=4/7$.\vspace{-4pt}
    \begin{multicols}{2}
      Check that Player 2 will not deviate:
      \begin{itemize}
        \item[4.] When he is in the \nth{1} round (the 'tough' stage) of punishing Player 1.
      \end{itemize}
      \vfill\null\columnbreak
      \vspace{-6pt}
      \begin{table}
        \begin{tabular}{l|c|c|c|}
          \multicolumn{1}{c}{} & \multicolumn{1}{c}{L} & \multicolumn{1}{c}{M} & \multicolumn{1}{c}{\textbf{\underline{H}}} \\\cline{2-4}
          L & 10, 10 & 3, \textcolor{blue}{15} & \textcolor{red}{0}, 7 \\\cline{2-4}
          \textbf{\underline{M}} & \textcolor{red}{15}, 3 & \textcolor{red}{7}, \textcolor{blue}{7} & \textbf{\underline{-4, 5}} \\\cline{2-4}
          H & 7, \textcolor{blue}{0} & 5, -4 & -15, -15 \\\cline{2-4}
        \end{tabular}
      \end{table}
    \end{multicols}
    \vspace{-20pt}
    If Player 2 does not deviate from $Q^P$, his expected utility is:
    \vspace{-6pt}
    \begin{align*}
      U_2(\underbrace{Q^D}_{s_1};\underbrace{Q^P}_{s_2})=5+15\delta+15\delta^2+...
                  =5+\sum_{\bm{t=2}}^\infty15\delta^{t-1}
                  =5+\frac{15\delta}{1-\delta}
                  =25,\text{ for }\delta=\frac{4}{7}
    \end{align*}
    If Player 2 deviates from $Q^P$ to play $M$, from the next round Player 1 will instead force Player 2 to take the punishment and play according to $Q^D$ forever:
    \begin{align*}
      U_2(\underbrace{M,Q^P}_{s_1'};\underbrace{M,Q^D}_{s_2'})&=7+\delta U_2(Q^P;Q^D)
                      =7+\delta\left(-4+\sum_{\bm{t=3}}^\infty3\delta^{t-1}\right)
                      =7+\delta\underbrace{\left(-4+\frac{3\delta}{1-\delta}\right)}_\text{=0 for $\delta=4/7$}\\
                      &=7,\text{ for }\delta=4/7
    \end{align*}
    \textbf{\textit{Does Player 2 have an incentive to deviate when he is in the \nth{1} round of punishing Player 1?}}
    \vfill\null
\end{frame}
\begin{frame}{PS7, Ex. 10: Is the punishment credible? (infinitely repeated game)}
    Use the roadmap to complete the proof that $(OP,OP)$ is a SPNE when $\delta=4/7$.\vspace{-4pt}
    \begin{multicols}{2}
      Check that Player 2 will not deviate:
      \begin{itemize}
        \item[4.] When he is in the \nth{1} round (the 'tough' stage) of punishing Player 1.
      \end{itemize}
      \vfill\null\columnbreak
      \vspace{-6pt}
      \begin{table}
        \begin{tabular}{l|c|c|c|}
          \multicolumn{1}{c}{} & \multicolumn{1}{c}{L} & \multicolumn{1}{c}{M} & \multicolumn{1}{c}{\textbf{\underline{H}}} \\\cline{2-4}
          L & 10, 10 & 3, \textcolor{blue}{15} & \textcolor{red}{0}, 7 \\\cline{2-4}
          \textbf{\underline{M}} & \textcolor{red}{15}, 3 & \textcolor{red}{7}, \textcolor{blue}{7} & \textbf{\underline{-4, 5}} \\\cline{2-4}
          H & 7, \textcolor{blue}{0} & 5, -4 & -15, -15 \\\cline{2-4}
        \end{tabular}
      \end{table}
      \vfill\null
    \end{multicols}
    \vspace{-24pt}
    If Player 2 does not deviate from $Q^P$, his expected utility is:
    \vspace{-6pt}
    \begin{align*}
      U_2(\underbrace{Q^D}_{s_1};\underbrace{Q^P}_{s_2})=5+15\delta+15\delta^2+...
                  =5+\sum_{\bm{t=2}}^\infty15\delta^{t-1}
                  =5+\frac{15\delta}{1-\delta}
                  =25,\text{ for }\delta=\frac{4}{7}
    \end{align*}
    If Player 2 deviates from $Q^P$ to play $M$, from the next round Player 1 will instead force Player 2 to take the punishment and play according to $Q^D$ forever:
    \begin{align*}
      U_2(\underbrace{M,Q^P}_{s_1'};\underbrace{M,Q^D}_{s_2'})&=7+\delta U_2(Q^P;Q^D)
                      =7+\delta\left(-4+\sum_{\bm{t=3}}^\infty3\delta^{t-1}\right)
                      =7+\delta\underbrace{\left(-4+\frac{3\delta}{1-\delta}\right)}_\text{=0 for $\delta=4/7$}\\
                      &=7,\text{ for }\delta=4/7
    \end{align*}
    \intuition{As $U_2(Q^D;Q^P)=\bm{25>7}=U_2(M,Q^P;M,Q^D)$, \textbf{Player 2 has no incentive to deviate.}\\\medskip
    I.e. in the \nth{1} round of punishing Player 1, Player 2 expects higher utility from playing according to $Q^P$ (25) than from deviating (7) for $\delta=4/7$.}
    \vfill\null
\end{frame}

\begin{frame}{PS7, Ex. 10: Is the punishment credible? (infinitely repeated game)}
  Use the roadmap to complete the proof that $(OP,OP)$ is a SPNE when $\delta=4/7$.\vspace{-4pt}
  \begin{multicols}{2}
    Check that Player 2 will not deviate:
    \begin{itemize}
      \item[5.] When he is in subsequent rounds ('mild' stage) of punishing Player 1.
    \end{itemize}
    \vfill\null\columnbreak
    \vspace{-6pt}
    \begin{table}
      \begin{tabular}{l|c|c|c|}
        \multicolumn{1}{c}{} & \multicolumn{1}{c}{L} & \multicolumn{1}{c}{\textbf{\underline{M}}} & \multicolumn{1}{c}{H} \\\cline{2-4}
        \textbf{\underline{L}} & 10, 10 & \textbf{\underline{3, \textcolor{blue}{15}}} & \textcolor{red}{0}, 7 \\\cline{2-4}
        M & \textcolor{red}{15}, 3 & \textcolor{red}{7}, \textcolor{blue}{7} & -4, 5 \\\cline{2-4}
        H & 7, \textcolor{blue}{0} & 5, -4 & -15, -15 \\\cline{2-4}
      \end{tabular}
    \end{table}
    \vfill\null
  \end{multicols}
    \vfill\null
\end{frame}
\begin{frame}{PS7, Ex. 10: Is the punishment credible? (infinitely repeated game)}
  Use the roadmap to complete the proof that $(OP,OP)$ is a SPNE when $\delta=4/7$.\vspace{-4pt}
  \begin{multicols}{2}
    Check that Player 2 will not deviate:
    \begin{itemize}
      \item[5.] When he is in subsequent rounds ('mild' stage) of punishing Player 1.
    \end{itemize}
    \vfill\null\columnbreak
    \vspace{-6pt}
    \begin{table}
      \begin{tabular}{l|c|c|c|}
        \multicolumn{1}{c}{} & \multicolumn{1}{c}{L} & \multicolumn{1}{c}{\textbf{\underline{M}}} & \multicolumn{1}{c}{H} \\\cline{2-4}
        \textbf{\underline{L}} & 10, 10 & \textbf{\underline{3, \textcolor{blue}{15}}} & \textcolor{red}{0}, 7 \\\cline{2-4}
        M & \textcolor{red}{15}, 3 & \textcolor{red}{7}, \textcolor{blue}{7} & -4, 5 \\\cline{2-4}
        H & 7, \textcolor{blue}{0} & 5, -4 & -15, -15 \\\cline{2-4}
      \end{tabular}
    \end{table}
    \vspace{-8pt}
  \end{multicols}
    In each subsequent round, the outcome from $(Q^D,Q^P)$ is $(L,M)$ with payoffs $(3,15)$.
    \vfill\null
\end{frame}
\begin{frame}{PS7, Ex. 10: Is the punishment credible? (infinitely repeated game)}
    Use the roadmap to complete the proof that $(OP,OP)$ is a SPNE when $\delta=4/7$.\vspace{-4pt}
    \begin{multicols}{2}
      Check that Player 2 will not deviate:
      \begin{itemize}
        \item[5.] When he is in subsequent rounds ('mild' stage) of punishing Player 1.
      \end{itemize}
      \vfill\null\columnbreak
      \vspace{-6pt}
      \begin{table}
        \begin{tabular}{l|c|c|c|}
          \multicolumn{1}{c}{} & \multicolumn{1}{c}{L} & \multicolumn{1}{c}{\textbf{\underline{M}}} & \multicolumn{1}{c}{H} \\\cline{2-4}
          \textbf{\underline{L}} & 10, 10 & \textbf{\underline{3, \textcolor{blue}{15}}} & \textcolor{red}{0}, 7 \\\cline{2-4}
          M & \textcolor{red}{15}, 3 & \textcolor{red}{7}, \textcolor{blue}{7} & -4, 5 \\\cline{2-4}
          H & 7, \textcolor{blue}{0} & 5, -4 & -15, -15 \\\cline{2-4}
        \end{tabular}
      \end{table}
      \vspace{-8pt}
    \end{multicols}
    In each subsequent round, the outcome from $(Q^D,Q^P)$ is $(L,M)$ with payoffs $(3,15)$.\\\medskip
    \textbf{\textit{Does player 2 have an incentive to deviate during this 'mild' stage?}}
    \vfill\null
\end{frame}
\begin{frame}{PS7, Ex. 10: Is the punishment credible? (infinitely repeated game)}
  Use the roadmap to complete the proof that $(OP,OP)$ is a SPNE when $\delta=4/7$.\vspace{-4pt}
  \begin{multicols}{2}
    Check that Player 2 will not deviate:
    \begin{itemize}
      \item[5.] When he is in subsequent rounds ('mild' stage) of punishing Player 1.
    \end{itemize}
    \vfill\null\columnbreak
    \vspace{-6pt}
    \begin{table}
      \begin{tabular}{l|c|c|c|}
        \multicolumn{1}{c}{} & \multicolumn{1}{c}{L} & \multicolumn{1}{c}{\textbf{\underline{M}}} & \multicolumn{1}{c}{H} \\\cline{2-4}
        \textbf{\underline{L}} & 10, 10 & \textbf{\underline{3, \textcolor{blue}{15}}} & \textcolor{red}{0}, 7 \\\cline{2-4}
        M & \textcolor{red}{15}, 3 & \textcolor{red}{7}, \textcolor{blue}{7} & -4, 5 \\\cline{2-4}
        H & 7, \textcolor{blue}{0} & 5, -4 & -15, -15 \\\cline{2-4}
      \end{tabular}
    \end{table}
    \vspace{-8pt}
  \end{multicols}
    In each subsequent round, the outcome from $(Q^D,Q^P)$ is $(L,M)$ with payoffs $(3,15)$.\\\medskip
    \intuition{From the \nth{2} round of punishing Player 1, Player 2 expects to earn 15 in every round which is the highest possible payoff in the stage game.\\\medskip
    I.e. in the 'mild' stage there is no incentive to deviate from $Q^P$ for any value of $\delta\geq0$.}
    \vfill\null
\end{frame}

\begin{frame}{PS7, Ex. 10: Is the punishment credible? (infinitely repeated game)}
  Use the roadmap to complete the proof that $(OP,OP)$ is a SPNE when $\delta=4/7$.\vspace{-4pt}
  \begin{multicols}{2}
    In the lecture it was checked that Player 1 will not deviate from $(OP,OP)$ in:
    \begin{enumerate}
      \item Round 1, or if $(L,L)$ was played in all previous rounds.
      \item The \nth{1} round of being punished.
      \item Subsequent rounds of being punished.
    \end{enumerate}
    In this exercise we have checked that Player 2 will not deviate:
    \begin{itemize}
      \item[4.] When he is in the \nth{1} round of punishing Player 1.
      \item[5.] When he is in subsequent rounds of punishing Player 1.
    \end{itemize}
    \vfill\null\columnbreak
    \vfill\null
  \end{multicols}
  \vfill\null
\end{frame}
\begin{frame}{PS7, Ex. 10: Is the punishment credible? (infinitely repeated game)}
  Use the roadmap to complete the proof that $(OP,OP)$ is a SPNE when $\delta=4/7$.\vspace{-4pt}
  \begin{multicols}{2}
    In the lecture it was checked that Player 1 will not deviate from $(OP,OP)$ in:
    \begin{enumerate}
      \item \textbf{Round 1, or if $\bm{(L,L)}$ was played in all previous rounds.}
      \item The \nth{1} round of being punished.
      \item Subsequent rounds of being punished.
    \end{enumerate}
    In this exercise we have checked that Player 2 will not deviate:
    \begin{itemize}
      \item[4.] When he is in the \nth{1} round of punishing Player 1.
      \item[5.] When he is in subsequent rounds of punishing Player 1.
    \end{itemize}
    \vfill\null\columnbreak
    \textbf{Condition 1} secures that $(OP,OP)$ is optimal \textbf{\textit{on}} the equilibrium path.
    \vfill\null
  \end{multicols}
  \vfill\null
\end{frame}
\begin{frame}{PS7, Ex. 10: Is the punishment credible? (infinitely repeated game)}
  Use the roadmap to complete the proof that $(OP,OP)$ is a SPNE when $\delta=4/7$.\vspace{-4pt}
  \begin{multicols}{2}
    In the lecture it was checked that Player 1 will not deviate from $(OP,OP)$ in:
    \begin{enumerate}
      \item Round 1, or if $(L,L)$ was played in all previous rounds.
      \item \textbf{The \nth{1} round of being punished.}
      \item \textbf{Subsequent rounds of being punished.}
    \end{enumerate}
    In this exercise we have checked that Player 2 will not deviate:
    \begin{itemize}
      \item[4.] \textbf{When he is in the \nth{1} round of punishing Player 1.}
      \item[5.] \textbf{When he is in subsequent rounds of punishing Player 1.}
    \end{itemize}
    \vfill\null\columnbreak
    Condition 1 secures that $(OP,OP)$ is optimal \textbf{\textit{on}} the equilibrium path.\\\medskip
    \textbf{Conditions 2-5} secure that $(OP,OP)$ is optimal \textbf{\textit{off}} the equilibrium path.
    \vfill\null
  \end{multicols}
  \vfill\null
\end{frame}
\begin{frame}{PS7, Ex. 10: Is the punishment credible? (infinitely repeated game)}
  Use the roadmap to complete the proof that $(OP,OP)$ is a SPNE when $\delta=4/7$.\vspace{-4pt}
  \begin{multicols}{2}
    In the lecture it was checked that Player 1 will not deviate from $(OP,OP)$ in:
    \begin{enumerate}
      \item Round 1, or if $(L,L)$ was played in all previous rounds.
      \item The \nth{1} round of being punished.
      \item Subsequent rounds of being punished.
    \end{enumerate}
    In this exercise we have checked that Player 2 will not deviate:
    \begin{itemize}
      \item[4.] When he is in the \nth{1} round of punishing Player 1.
      \item[5.] When he is in subsequent rounds of punishing Player 1.
    \end{itemize}
    \vfill\null\columnbreak
    Condition 1 secures that $(OP,OP)$ is optimal \textbf{\textit{on}} the equilibrium path.\\\medskip
    Conditions 2-5 secure that $(OP,OP)$ is optimal \textbf{\textit{off}} the equilibrium path.\\\medskip
    \textbf{As all conditions hold, we can conclude that $\bm{(OP,OP)}$ is a SPNE for $\bm{\delta = 4/7}$}.
    \vfill\null
  \end{multicols}
  \vfill\null
\end{frame}
