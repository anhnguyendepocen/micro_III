\section{PS7, Ex. 1 (A): Imperfect recall (imperfect information)}

\begin{frame}{PS7, Ex. 1 (A): Imperfect recall (imperfect information)}
  \begin{multicols}{2}
    In this course we normally consider games in which there is ’perfect recall’: players can always remember what they themselves have done in the past.\\\medskip
    We have seen an example in class of a game with ’imperfect recall’ where the player forgets his own actions. But what would a game where he forgets the opponent’s actions look like? Construct a game with two players. The timing is as follows: Player 1 moves first, then Player 2, and then Player 2 again. Everytime they move, the players choose one of two actions: $\{L, R\}$.\\\medskip
    Draw the game tree and construct the information sets such that (a) Player 2 observes Player 1’s action the first time he moves, but (b) when Player 2 moves the second time, he has forgotten what Player 1 chose. However, he recalls his own action.
    \vfill\null\columnbreak
    \vfill\null
  \end{multicols}
\end{frame}
\begin{frame}{PS7, Ex. 1 (A): Imperfect recall (imperfect information)}
  \begin{multicols}{2}
    In this course we normally consider games in which there is ’perfect recall’: players can always remember what they themselves have done in the past.\\\medskip
    We have seen an example in class of a game with ’imperfect recall’ where the player forgets his own actions. But what would a game where he forgets the opponent’s actions look like? Construct a game with two players. The timing is as follows: Player 1 moves first, then Player 2, and then Player 2 again. Everytime they move, the players choose one of two actions: $\{L, R\}$.\\\medskip
    Draw the game tree and construct the information sets such that (a) Player 2 observes Player 1’s action the first time he moves, but (b) when Player 2 moves the second time, he has forgotten what Player 1 chose. However, he recalls his own action.
    \vfill\null\columnbreak
    \vfill\null
    \begin{figure}[!h]
      \center
      \def\svgwidth{\columnwidth}
      \import{figures/}{1.pdf_tex}
    \end{figure}
  \end{multicols}
\end{frame}



\section{PS7, Ex. 2 (A): Three conditions for a subgame (imperfect information)}

\begin{frame}{PS7, Ex. 2 (A): Three conditions for a subgame (imperfect information)}
  Recall that under imperfect information we have three conditions that define a subgame. Construct an example of a violation of each of the three conditions (pick different examples than those seen in the lectures).
  \vfill\null
\end{frame}

\begin{frame}{PS7, Ex. 2 (A): Three conditions for a subgame (imperfect information)}
  Recall that under imperfect information we have three conditions that define a subgame. Construct an example of a violation of each of the three conditions (pick different examples than those seen in the lectures).
  \begin{multicols}{2}
    Under imperfect information, a subgame must satisfy three properties:
    \begin{enumerate}
      \item It begins at a decision node $n$ that is a singleton information set.
    \end{enumerate}
    \vfill\null\columnbreak
    Example of violation of condition 1:
    \vfill\null
  \end{multicols}
\end{frame}
\begin{frame}{PS7, Ex. 2 (A): Three conditions for a subgame (imperfect information)}
  Recall that under imperfect information we have three conditions that define a subgame. Construct an example of a violation of each of the three conditions (pick different examples than those seen in the lectures).
  \begin{multicols}{2}
    Under imperfect information, a subgame must satisfy three properties:
    \begin{enumerate}
      \item It begins at a decision node $n$ that is a singleton information set.
    \end{enumerate}
    \vfill\null\columnbreak
    Example of violation of condition 1:
    \begin{figure}[!h]
      \center
      \def\svgwidth{\columnwidth}
      \import{figures/}{2a.pdf_tex}
    \end{figure}
    The purple decision node to the right is not a singleton information set.
    \vfill\null
  \end{multicols}
\end{frame}
