\section{PS7, Ex. 1 (A): Imperfect recall (imperfect information)}

\begin{frame}{PS7, Ex. 1 (A): Imperfect recall (imperfect information)}
  \begin{multicols}{2}
    In this course we normally consider games in which there is ’perfect recall’: players can always remember what they themselves have done in the past.\\\medskip
    We have seen an example in class of a game with ’imperfect recall’ where the player forgets his own actions. But what would a game where he forgets the opponent’s actions look like? Construct a game with two players. The timing is as follows: Player 1 moves first, then Player 2, and then Player 2 again. Everytime they move, the players choose one of two actions: $\{L, R\}$.\\\medskip
    Draw the game tree and construct the information sets such that (a) Player 2 observes Player 1’s action the first time he moves, but (b) when Player 2 moves the second time, he has forgotten what Player 1 chose. However, he recalls his own action.
    \vfill\null\columnbreak
    \vfill\null
  \end{multicols}
\end{frame}
\begin{frame}{PS7, Ex. 1 (A): Imperfect recall (imperfect information)}
  \begin{multicols}{2}
    In this course we normally consider games in which there is ’perfect recall’: players can always remember what they themselves have done in the past.\\\medskip
    We have seen an example in class of a game with ’imperfect recall’ where the player forgets his own actions. But what would a game where he forgets the opponent’s actions look like? Construct a game with two players. The timing is as follows: Player 1 moves first, then Player 2, and then Player 2 again. Everytime they move, the players choose one of two actions: $\{L, R\}$.\\\medskip
    Draw the game tree and construct the information sets such that (a) Player 2 observes Player 1’s action the first time he moves, but (b) when Player 2 moves the second time, he has forgotten what Player 1 chose. However, he recalls his own action.
    \vfill\null\columnbreak
    \vfill\null
    \begin{figure}[!h]
      \center
      \def\svgwidth{\columnwidth}
      \import{figures/}{1.pdf_tex}
    \end{figure}
  \end{multicols}
\end{frame}



\section{PS7, Ex. 2 (A): Three conditions for a subgame (imperfect information)}

\begin{frame}{PS7, Ex. 2 (A): Three conditions for a subgame (imperfect information)}
  Recall that under imperfect information we have three conditions that define a subgame. Construct an example of a violation of each of the three conditions (pick different examples than those seen in the lectures).
  \vfill\null
\end{frame}

\begin{frame}{PS7, Ex. 2 (A): Three conditions for a subgame (imperfect information)}
  Recall that under imperfect information we have three conditions that define a subgame. Construct an example of a violation of each of the three conditions (pick different examples than those seen in the lectures).
  \begin{multicols}{2}
    Under imperfect information, a subgame must satisfy three properties:
    \begin{enumerate}
      \item It begins at a decision node $n$ that is a singleton information set.
    \end{enumerate}
    \vfill\null\columnbreak
    Example of violation of condition 1:
    \vspace{-4pt}
    \begin{figure}[!h]
      \center
      \def\svgwidth{\columnwidth}
      \import{figures/}{2.pdf_tex}
    \end{figure}
    \vfill\null
  \end{multicols}
\end{frame}
\begin{frame}{PS7, Ex. 2 (A): Three conditions for a subgame (imperfect information)}
  Recall that under imperfect information we have three conditions that define a subgame. Construct an example of a violation of each of the three conditions (pick different examples than those seen in the lectures).
  \begin{multicols}{2}
    Under imperfect information, a subgame must satisfy three properties:
    \begin{enumerate}
      \item It begins at a decision node $n$ that is a singleton information set.
    \end{enumerate}
    \vfill\null\columnbreak
    Example of violation of condition 1:
    \vspace{-12pt}
    \begin{figure}[!h]
      \center
      \def\svgwidth{1.1\columnwidth}
      \import{figures/}{2a.pdf_tex}
    \end{figure}
    The purple decision node to the right is not a singleton information set (nor is the orange decision node to the left).
    \vfill\null
  \end{multicols}
\end{frame}

\begin{frame}{PS7, Ex. 2 (A): Three conditions for a subgame (imperfect information)}
  Recall that under imperfect information we have three conditions that define a subgame. Construct an example of a violation of each of the three conditions (pick different examples than those seen in the lectures).
  \begin{multicols}{2}
    Under imperfect information, a subgame must satisfy three properties:
    \begin{enumerate}
      \item It begins at a decision node $n$ that is a singleton information set.
      \item \textbf{It includes all following decision and terminal nodes following \textit{n} in the game tree}, but no nodes that do not follow $n$.
    \end{enumerate}
    \vfill\null\columnbreak
    Example of violation of the first part of condition 2:
    \vspace{-4pt}
    \begin{figure}[!h]
      \center
      \def\svgwidth{\columnwidth}
      \import{figures/}{2b.pdf_tex}
    \end{figure}
    \vfill\null
  \end{multicols}
\end{frame}
\begin{frame}{PS7, Ex. 2 (A): Three conditions for a subgame (imperfect information)}
  Recall that under imperfect information we have three conditions that define a subgame. Construct an example of a violation of each of the three conditions (pick different examples than those seen in the lectures).
  \begin{multicols}{2}
    Under imperfect information, a subgame must satisfy three properties:
    \begin{enumerate}
      \item It begins at a decision node $n$ that is a singleton information set.
      \item \textbf{It includes all following decision and terminal nodes following \textit{n} in the game tree}, but no nodes that do not follow $n$.
    \end{enumerate}
    \vfill\null\columnbreak
    Example of violation of the first part of condition 2:
    \vspace{-20pt}
    \begin{figure}[!h]
      \center
      \def\svgwidth{1.1\columnwidth}
      \import{figures/}{2b1.pdf_tex}
    \end{figure}
    For a subgame containing the blue decision node $n$, all following decision nodes must be included.
    \vfill\null
  \end{multicols}
\end{frame}
\begin{frame}{PS7, Ex. 2 (A): Three conditions for a subgame (imperfect information)}
  Recall that under imperfect information we have three conditions that define a subgame. Construct an example of a violation of each of the three conditions (pick different examples than those seen in the lectures).
  \begin{multicols}{2}
    Under imperfect information, a subgame must satisfy three properties:
    \begin{enumerate}
      \item It begins at a decision node $n$ that is a singleton information set.
      \item It includes all following decision and terminal nodes following \textit{n} in the game tree, \textbf{but no nodes that do not follow \textit{n}.}
    \end{enumerate}
    \vfill\null\columnbreak
    Example of violation of the second part of condition 2:
    \vspace{-4pt}
    \begin{figure}[!h]
      \center
      \def\svgwidth{\columnwidth}
      \import{figures/}{2b.pdf_tex}
    \end{figure}
    \vfill\null
  \end{multicols}
\end{frame}
\begin{frame}{PS7, Ex. 2 (A): Three conditions for a subgame (imperfect information)}
  Recall that under imperfect information we have three conditions that define a subgame. Construct an example of a violation of each of the three conditions (pick different examples than those seen in the lectures).
  \begin{multicols}{2}
    Under imperfect information, a subgame must satisfy three properties:
    \begin{enumerate}
      \item It begins at a decision node $n$ that is a singleton information set.
      \item It includes all following decision and terminal nodes following \textit{n} in the game tree, \textbf{but no nodes that do not follow \textit{n}.}
    \end{enumerate}
    \vfill\null\columnbreak
    Example of violation of the second part of condition 2:
    \vspace{-20pt}
    \begin{figure}[!h]
      \center
      \def\svgwidth{1.1\columnwidth}
      \import{figures/}{2b2.pdf_tex}
    \end{figure}
    Regardless of whether the orange or the purple node is chosen as the first decision node $n$, the other decision node does not follow $n$, and therefore cannot be part of the subgame.
    \vfill\null
  \end{multicols}
\end{frame}


\begin{frame}{PS7, Ex. 2 (A): Three conditions for a subgame (imperfect information)}
  Recall that under imperfect information we have three conditions that define a subgame. Construct an example of a violation of each of the three conditions (pick different examples than those seen in the lectures).
  \begin{multicols}{2}
    Under imperfect information, a subgame must satisfy three properties:
    \begin{enumerate}
      \item It begins at a decision node $n$ that is a singleton information set.
      \item It includes all following decision and terminal nodes following $n$ in the game tree, but no nodes that do not follow $n$.
      \item It does not "cut" any information set: if a decision node $n'$ follows $n$ in the game tree, then all other nodes in the information set including $n'$ must also follow $n$ (and so be included in the subgame).
    \end{enumerate}
    \vfill\null\columnbreak
    Example of violation of condition 3:
    \vspace{-4pt}
    \begin{figure}[!h]
      \center
      \def\svgwidth{\columnwidth}
      \import{figures/}{2.pdf_tex}
    \end{figure}
    \vfill\null
  \end{multicols}
\end{frame}
\begin{frame}{PS7, Ex. 2 (A): Three conditions for a subgame (imperfect information)}
  Recall that under imperfect information we have three conditions that define a subgame. Construct an example of a violation of each of the three conditions (pick different examples than those seen in the lectures).
  \begin{multicols}{2}
    Under imperfect information, a subgame must satisfy three properties:
    \begin{enumerate}
      \item It begins at a decision node $n$ that is a singleton information set.
      \item It includes all following decision and terminal nodes following $n$ in the game tree, but no nodes that do not follow $n$.
      \item It does not "cut" any information set: if a decision node $n'$ follows $n$ in the game tree, then all other nodes in the information set including $n'$ must also follow $n$ (and so be included in the subgame).
    \end{enumerate}
    \vfill\null\columnbreak
    Example of violation of condition 3:
    \vspace{-4pt}
    \begin{figure}[!h]
      \center
      \def\svgwidth{\columnwidth}
      \import{figures/}{2c.pdf_tex}
    \end{figure}
    The orange decision node to the left is part of the same information set as the purple node to the right, so it must be included in the same subgame.
    \vfill\null
  \end{multicols}
\end{frame}
