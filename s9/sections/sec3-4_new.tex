\section{The expected highest and second highest draw from a uniform distribution}

\begin{frame}{The expected highest and second highest draw from a uniform distribution}
    To find \textit{seller's expected revenue} from a sealed bid auction (e.g. bidders simultaneously submit their bids in sealed envelopes without knowing the bids of others) with symmetric bidders with valuation drawn from a uniform distribution, there are two different approaches:
    \begin{enumerate}
      \item One approach is to derive each \textit{bidder's expected payment} as a function of her valuation and then integrate this expression using the PDF to get the \textit{ex-ante expected payment} of each bidder which can then be added together to find seller's expected revenue.
      \item However, a more simple approach is to for $N$ number of bidders to calculate the expected highest value (first-price sealed bid auction) or the expected second-highest value (second-price sealed bid auction). Plugging the value into the bid-function gives the seller's expected revenue.
    \end{enumerate}
    Deriving the bid-function (best-response function) is a prerequisite for both approaches.
\end{frame}

\begin{frame}{The expected highest and second highest draw from a uniform distribution}
    First, let $X=x_1,x_2,...,x_N$ be $N$ independent and identically distributed (i.i.d.) draws from the \textit{standard} uniform distribution: $x\sim U(0, 1)$.
    \vfill\null
\end{frame}
\begin{frame}{The expected highest and second highest draw from a uniform distribution}
    First, let $X=x_1,x_2,...,x_N$ be $N$ independent and identically distributed (i.i.d.) draws from the \textit{standard} uniform distribution: $x\sim U(0, 1)$.\\\smallskip
    The expected highest draw $Y_1$ and the expected second-highest draw $Y_2$ for all $N$ draws are: \vspace{-8pt}
    \begin{multicols}{2}
      \begin{align*}
        \mathbb{E}[Y_1]&=\frac{N}{N+1},\ \text{ where }Y_1=max(X)\\
        \mathbb{E}[Y_2]&=\frac{N-1}{N+1},\ \text{ where }Y_2=max(X\neq Y_1)
      \end{align*}
      \vfill\null\columnbreak
      \vfill\null
    \end{multicols}
\end{frame}
\begin{frame}{The expected highest and second highest draw from a uniform distribution}
    First, let $X=x_1,x_2,...,x_N$ be $N$ independent and identically distributed (i.i.d.) draws from the \textit{standard} uniform distribution: $x\sim U(0, 1)$.\\\smallskip
    The expected highest draw $Y_1$ and the expected second-highest draw $Y_2$ for all $N$ draws are: \vspace{-8pt}
    \begin{multicols}{2}
      \begin{align*}
        \mathbb{E}[Y_1]&=\frac{N}{N+1},\ \text{ where }Y_1=max(X)\\
        \mathbb{E}[Y_2]&=\frac{N-1}{N+1},\ \text{ where }Y_2=max(X\neq Y_1)
      \end{align*}
      \vfill\null\columnbreak
      \begin{figure}[!h]
        \center
        \def\svgwidth{1.1\columnwidth}
        \import{figures/}{E[Y_1].pdf_tex}
      \end{figure}
      \vfill\null
    \end{multicols}
\end{frame}
\begin{frame}{The expected highest and second highest draw from a uniform distribution}
    First, let $X=x_1,x_2,...,x_N$ be $N$ independent and identically distributed (i.i.d.) draws from the \textit{standard} uniform distribution: $x\sim U(0, 1)$.\\\smallskip
    The expected highest draw $Y_1$ and the expected second-highest draw $Y_2$ for all $N$ draws are: \vspace{-10pt}
    \begin{multicols}{2}
      \begin{align*}
        \mathbb{E}[Y_1]&=\frac{N}{N+1},\ \text{ where }Y_1=max(X)\\
        \mathbb{E}[Y_2]&=\frac{N-1}{N+1},\ \text{ where }Y_2=max(X\neq Y_1)
      \end{align*}
      \vfill\null\columnbreak
      \begin{figure}[!h]
        \center
        \def\svgwidth{1.1\columnwidth}
        \import{figures/}{E[Y_1].pdf_tex}
      \end{figure}
      \vfill\null
    \end{multicols} \vspace{-20pt}
    Generalized, let $X=x_1,x_2,...,x_N$ be $N$ independent and identically distributed (i.i.d.) draws from a uniform distribution: $x\sim U(a, b)$. The expected highest draw $Y_1$ and the expected second-highest draw $Y_2$ for all $N$ draws are:
    \begin{align*}
      \mathbb{E}[Y_1]&=a+N\frac{b-a}{N+1},&&\text{where }Y_1=max(X)\\
      \mathbb{E}[Y_2]&=a+(N-1)\frac{b-a}{N+1},&&\text{where }Y_2=max(X\neq Y_1)
    \end{align*}
\end{frame}
\begin{frame}{The expected highest and second highest draw from a uniform distribution}
    First, let $X=x_1,x_2,...,x_N$ be $N$ independent and identically distributed (i.i.d.) draws from the \textit{standard} uniform distribution: $x\sim U(0, 1)$.\\\smallskip
    The expected highest draw $Y_1$ and the expected second-highest draw $Y_2$ for all $N$ draws are: \vspace{-16pt}
    \begin{multicols}{2}
      \begin{align*}
        \mathbb{E}[Y_1]&=\frac{N}{N+1},\ \text{ where }Y_1=max(X)\\
        \mathbb{E}[Y_2]&=\frac{N-1}{N+1},\ \text{ where }Y_2=max(X\neq Y_1)
      \end{align*}
      \vfill\null\columnbreak
      \begin{figure}[!h]
        \center
        \def\svgwidth{1.1\columnwidth}
        \import{figures/}{E[Y_1].pdf_tex}
      \end{figure}
      \vfill\null
    \end{multicols} \vspace{-24pt}
    Generalized, let $X=x_1,x_2,...,x_N$ be $N$ independent and identically distributed (i.i.d.) draws from a uniform distribution: $x\sim U(a, b)$. The expected highest draw $Y_1$ and the expected second-highest draw $Y_2$ for all $N$ draws are: \vspace{-4pt}
    \begin{align*}
      \mathbb{E}[Y_1]&=a+N\frac{b-a}{N+1},&&\text{where }Y_1=max(X)\\
      \mathbb{E}[Y_2]&=a+(N-1)\frac{b-a}{N+1},&&\text{where }Y_2=max(X\neq Y_1)
    \end{align*}
    E.g. for $N=1$, $\mathbb{E}[Y_1]$ simply collapses to the expression for the expected mean, $\mu$:\vspace{-4pt}
    \begin{align*}
      \mathbb{E}[Y_1]=a+N\frac{b-a}{N+1}=a+1\frac{b-a}{1+1}=\frac{2a}{2}+\frac{b-a}{2}=\frac{a+b}{2}\equiv\mu
    \end{align*}
\end{frame}

\begin{frame}{The expected highest and second highest draw from a uniform distribution}
    Applied to auctions, consider $N$ number of bidders where each bidder $i$ has the value $v_i$ that is independently drawn from the same uniform distribution $v_i\sim U(a,b)$.\\\medskip
    Then the expected highest value $Y_1$ and the expected second-highest value $Y_2$ for all $N$ bidders are:
    \begin{align*}
      \mathbb{E}[Y_1]&=a+N\frac{b-a}{N+1},&&\text{ where }Y_1=max(V),\ V=v_1,v_2,...,v_N\\
      \mathbb{E}[Y_2]&=a+(N-1)\frac{b-a}{N+1},&&\text{ where }Y_2=max(V\neq Y_1)
    \end{align*}
    To calculate the seller's expected revenue, insert the expected highest value $\mathbb{E}[Y_1]$ (first-price sealed bid auction) or the expected second-highest value $\mathbb{E}[Y_2]$ (second-price sealed bid auction) in the derived bid-function.
    \vfill\null
\end{frame}



\section{PS8, Ex. 3: First- and second-price sealed bid auctions with two bidders}

\begin{frame}{PS8, Ex. 3: First- and second-price sealed bid auctions with two bidders}
    \begin{multicols}{2}
      Consider a first-price sealed bid auction with two bidders, who have valuations $v_1$ and $v_2$, respectively. These values are distributed independently uniformly with
      \begin{align*}
        v_i\sim U(1,3)
      \end{align*}
      Thus, the values are \textit{private}.
      \begin{itemize}
        \item[(a)] Show that there is a symmetric Bayesian Nash Equilibrium in linear strategies: $b_i(v_i) = cv_i + d$.\\
                   Find \textit{c} and \textit{d}.
        \item[(b)] Calculate the revenue to the seller.
      \end{itemize}
      \vfill\null\columnbreak
      \begin{itemize}
        \item[(c)] Suppose now that the object is sold by a \textit{second-price sealed bid auction}.
        \begin{itemize}\normalsize
          \item[i.]   Suppose player 2 bids his valuation: $b_2(v_2) = v_2$. Write down the expected payoffs to player 1 from bidding $b_1$.
          \item[ii.]  Using your previous answer, argue that there is a symmetric Bayesian Nash Equilibrium (BNE) in which both players bid their valuation.
          \item[iii.] Calculate the revenue to the seller from this equilibrium. Compare to the answer in (b).
        \end{itemize}
      \end{itemize}
      \vfill\null
    \end{multicols}
    \vspace{-8pt}
    \textit{[Consider a uniform distribution $x\sim U(a, b)$. Try to write up the probability density function (PDF), cumulative distribution function (CDF), and the expected highest value $Y_1$ and second-highest value $Y_2$ from N draws of x.]}
\end{frame}
\begin{frame}{PS8, Ex. 3: First- and second-price sealed bid auctions with two bidders}
    \begin{multicols}{2}
      Consider a first-price sealed bid auction with two bidders, who have valuations $v_1$ and $v_2$, respectively. These values are distributed independently uniformly with
      \begin{align*}
        v_i\sim U(1,3)
      \end{align*}
      Thus, the values are \textit{private}.
      \begin{itemize}
        \item[(a)] Show that there is a symmetric Bayesian Nash Equilibrium in linear strategies: $b_i(v_i) = cv_i + d\ (*)$.\\
                   Find \textit{c} and \textit{d}.
        \item[(b)] Calculate the revenue to the seller.
      \end{itemize}
      \vfill\null\columnbreak
      \begin{itemize}
        \item[(c)] Suppose now that the object is sold by a \textit{second-price sealed bid auction}.
        \begin{itemize}\normalsize
          \item[i.]   Suppose player 2 bids his valuation: $b_2(v_2) = v_2$. Write down the expected payoffs to player 1 from bidding $b_1$.
          \item[ii.]  Using your previous answer, argue that there is a symmetric Bayesian Nash Equilibrium (BNE) in which both players bid their valuation.
          \item[iii.] Calculate the revenue to the seller from this equilibrium. Compare to the answer in (b).
        \end{itemize}
      \end{itemize}
      \vfill\null
    \end{multicols}
    \vspace{-16pt}
    Standard results for $N$ draws from a uniform distribution $x\sim U(a, b):$
    \begin{enumerate}
      \item[PDF:] Probability density function: $f(x)=\frac{1}{b-a}$
      \item[CDF:] Cumulative distribution function: $F(x)=\frac{x-a}{b-a}\Rightarrow\mathbb{P}(c>x)=\frac{c-a}{b-a}$
      \item[$\mathbb{E}(Y_1)$] $=a+N\frac{b-a}{N+1}$ where $Y_1=\max(X),\ X=x_1,x_2,...,x_N$
      \item[$\mathbb{E}(Y_2)$] $=a+(N-1)\frac{b-a}{N+1}$ where $Y_2=\max(X\neq Y_1)$
    \end{enumerate}
    \vfill\null
\end{frame}


\begin{frame}{PS8, Ex. 3.a: First- and second-price sealed bid auctions with two bidders}
    Consider a first-price sealed bid auction with two bidders, who have valuations $v_1$ and $v_2$, respectively. These values are distributed independently uniformly with $v_i\sim U(1,3)$, thus, the values are \textit{private}.
    \vspace{-4pt}
    \begin{itemize}
      \item[(a)] Show that there is a symmetric Bayesian Nash Equilibrium in linear strategies: $b_i(v_i) = cv_i + d\ (*)$. Find \textit{c} and \textit{d}.
    \end{itemize}
    \vspace{-8pt}
    \begin{multicols}{2}
      \begin{itemize}
        \item[\nth{1} step:] \textbf{Assuming bidder \textit{j} follows the proposed strategy $b_j(v_j) = cv_j + d$, calculate bidder \textit{i}'s expected payoff from bidding $b_i$.}
      \end{itemize}
      \vfill\null\columnbreak
      For $N$ draws from $x\sim U(a, b):$
      \vspace{-6pt}
      \begin{enumerate}
        \item[PDF:] $f(x)=\frac{1}{b-a}$
        \item[CDF:] $F(x)=\frac{x-a}{b-a}\Rightarrow\mathbb{P}(c>x)=\frac{c-a}{b-a}$
        \item[$\mathbb{E}(Y_1)$] $=a$+$N\frac{b-a}{N+1}$, $Y_1=\max(X$=$x_1,..,x_N)$
        \item[$\mathbb{E}(Y_2)$] $=a$+$(N$-1$)\frac{b-a}{N+1}$, $Y_2=\max(X\neq Y_1)$
      \end{enumerate}
      \vfill\null
    \end{multicols}
\end{frame}
\begin{frame}{PS8, Ex. 3.a: First- and second-price sealed bid auctions with two bidders}
    Consider a first-price sealed bid auction with two bidders, who have valuations $v_1$ and $v_2$, respectively. These values are distributed independently uniformly with $v_i\sim U(1,3)$, thus, the values are \textit{private}.
    \vspace{-4pt}
    \begin{itemize}
      \item[(a)] Show that there is a symmetric Bayesian Nash Equilibrium in linear strategies: $b_i(v_i) = cv_i + d\ (*)$. Find \textit{c} and \textit{d}.
    \end{itemize}
    \vspace{-8pt}
    \begin{multicols}{2}
      \begin{itemize}
        \item[\nth{1} step:] \textbf{Assuming bidder \textit{j} follows the proposed strategy $b_j(v_j) = cv_j + d$, calculate bidder \textit{i}'s expected payoff from bidding $b_i$:}
      \end{itemize}
      \vspace{-4pt}
      \begin{align*}
        \mathbb{E}[u_i(b_i,v_i)]&=\mathbb{P}(i\ wins|b_i)(v_i-b_i)
      \end{align*}
      \vfill\null\columnbreak
      For $N$ draws from $x\sim U(a, b):$
      \vspace{-6pt}
      \begin{enumerate}
        \item[PDF:] $f(x)=\frac{1}{b-a}$
        \item[CDF:] $F(x)=\frac{x-a}{b-a}\Rightarrow\mathbb{P}(c>x)=\frac{c-a}{b-a}$
        \item[$\mathbb{E}(Y_1)$] $=a$+$N\frac{b-a}{N+1}$, $Y_1=\max(X$=$x_1,..,x_N)$
        \item[$\mathbb{E}(Y_2)$] $=a$+$(N$-1$)\frac{b-a}{N+1}$, $Y_2=\max(X\neq Y_1)$
      \end{enumerate}
      \vfill\null
    \end{multicols}
\end{frame}
\begin{frame}{PS8, Ex. 3.a: First- and second-price sealed bid auctions with two bidders}
    Consider a first-price sealed bid auction with two bidders, who have valuations $v_1$ and $v_2$, respectively. These values are distributed independently uniformly with $v_i\sim U(1,3)$, thus, the values are \textit{private}.
    \vspace{-4pt}
    \begin{itemize}
      \item[(a)] Show that there is a symmetric Bayesian Nash Equilibrium in linear strategies: $b_i(v_i) = cv_i + d\ (*)$. Find \textit{c} and \textit{d}.
    \end{itemize}
    \vspace{-8pt}
    \begin{multicols}{2}
      \begin{itemize}
        \item[\nth{1} step:] Assuming bidder \textit{j} follows the proposed strategy $b_j(v_j) = cv_j + d$, calculate bidder \textit{i}'s expected payoff from bidding $b_i$:
      \end{itemize}
      \vspace{-4pt}
      \begin{align*}
        \mathbb{E}[u_i(b_i,v_i)]&=\mathbb{P}(i\ wins|b_i)(v_i-b_i)\\
                                &=\mathbb{P}(b_i>b_j(v_j))(v_i-b_i)\\
                                &=\mathbb{P}(b_i>cv_j+d)(v_i-b_i)\\
                                &=\mathbb{P}\left(\frac{b_i-d}{c}>v_j\right)(v_i-b_i)\\
                                &=\frac{\frac{b_i-d}{c}-1}{3-1}(v_i-b_i),\text{ using CDF}\\
                                &=\frac{b_i-d-c}{2c}(v_i-b_i)
      \end{align*}
      \vfill\null\columnbreak
      For $N$ draws from $x\sim U(a, b):$
      \vspace{-6pt}
      \begin{enumerate}
        \item[PDF:] $f(x)=\frac{1}{b-a}$
        \item[CDF:] $F(x)=\frac{x-a}{b-a}\Rightarrow\mathbb{P}(c>x)=\frac{c-a}{b-a}$
        \item[$\mathbb{E}(Y_1)$] $=a$+$N\frac{b-a}{N+1}$, $Y_1=\max(X$=$x_1,..,x_N)$
        \item[$\mathbb{E}(Y_2)$] $=a$+$(N$-1$)\frac{b-a}{N+1}$, $Y_2=\max(X\neq Y_1)$
      \end{enumerate}
      \vfill\null
    \end{multicols}
\end{frame}
\begin{frame}{PS8, Ex. 3.a: First- and second-price sealed bid auctions with two bidders}
    Consider a first-price sealed bid auction with two bidders, who have valuations $v_1$ and $v_2$, respectively. These values are distributed independently uniformly with $v_i\sim U(1,3)$, thus, the values are \textit{private}.
    \vspace{-4pt}
    \begin{itemize}
      \item[(a)] Show that there is a symmetric Bayesian Nash Equilibrium in linear strategies: $b_i(v_i) = cv_i + d\ (*)$. Find \textit{c} and \textit{d}.
    \end{itemize}
    \vspace{-8pt}
    \begin{multicols}{2}
      \begin{itemize}
        \item[\nth{1} step:] Assuming bidder \textit{j} follows the proposed strategy $b_j(v_j) = cv_j + d$, calculate bidder \textit{i}'s expected payoff from bidding $b_i$:
      \end{itemize}
      \vspace{-4pt}
      \begin{align*}
        \mathbb{E}[u_i(b_i,v_i)]&=\mathbb{P}(i\ wins|b_i)(v_i-b_i)\\
                                &=\mathbb{P}(b_i>b_j(v_j))(v_i-b_i)\\
                                &=\mathbb{P}(b_i>cv_j+d)(v_i-b_i)\\
                                &=\mathbb{P}\left(\frac{b_i-d}{c}>v_j\right)(v_i-b_i)\\
                                &=\frac{\frac{b_i-d}{c}-1}{3-1}(v_i-b_i),\text{ using CDF}\\
                                &=\frac{b_i-d-c}{2c}(v_i-b_i)
      \end{align*}
      \vspace{-10pt}
      \begin{itemize}
        \item[\nth{2} step:] \textbf{Take the FOC and SOC wrt. $b_i$.}
      \end{itemize}
      \vfill\null\columnbreak
      For $N$ draws from $x\sim U(a, b):$
      \vspace{-6pt}
      \begin{enumerate}
        \item[PDF:] $f(x)=\frac{1}{b-a}$
        \item[CDF:] $F(x)=\frac{x-a}{b-a}\Rightarrow\mathbb{P}(c>x)=\frac{c-a}{b-a}$
        \item[$\mathbb{E}(Y_1)$] $=a$+$N\frac{b-a}{N+1}$, $Y_1=\max(X$=$x_1,..,x_N)$
        \item[$\mathbb{E}(Y_2)$] $=a$+$(N$-1$)\frac{b-a}{N+1}$, $Y_2=\max(X\neq Y_1)$
      \end{enumerate}
      \vfill\null
    \end{multicols}
\end{frame}
\begin{frame}{PS8, Ex. 3.a: First- and second-price sealed bid auctions with two bidders}
    Consider a first-price sealed bid auction with two bidders, who have valuations $v_1$ and $v_2$, respectively. These values are distributed independently uniformly with $v_i\sim U(1,3)$, thus, the values are \textit{private}.
    \vspace{-4pt}
    \begin{itemize}
      \item[(a)] Show that there is a symmetric Bayesian Nash Equilibrium in linear strategies: $b_i(v_i) = cv_i + d\ (*)$. Find \textit{c} and \textit{d}.
    \end{itemize}
    \vspace{-8pt}
    \begin{multicols}{2}
      \begin{itemize}
        \item[\nth{1} step:] Assuming bidder \textit{j} follows the proposed strategy $b_j(v_j) = cv_j + d$, calculate bidder \textit{i}'s expected payoff from bidding $b_i$:
      \end{itemize}
      \vspace{-4pt}
      \begin{align*}
        \mathbb{E}[u_i(b_i,v_i)]&=\mathbb{P}(i\ wins|b_i)(v_i-b_i)\\
                                &=\mathbb{P}(b_i>b_j(v_j))(v_i-b_i)\\
                                &=\mathbb{P}(b_i>cv_j+d)(v_i-b_i)\\
                                &=\mathbb{P}\left(\frac{b_i-d}{c}>v_j\right)(v_i-b_i)\\
                                &=\frac{\frac{b_i-d}{c}-1}{3-1}(v_i-b_i),\text{ using CDF}\\
                                &=\frac{b_i-d-c}{2c}(v_i-b_i)
      \end{align*}
      \vspace{-10pt}
      \begin{itemize}
        \item[\nth{2} step:] \textbf{Take the FOC and SOC wrt. $b_i$.}
      \end{itemize}
      \vfill\null\columnbreak
      For $N$ draws from $x\sim U(a, b):$
      \vspace{-6pt}
      \begin{enumerate}
        \item[PDF:] $f(x)=\frac{1}{b-a}$
        \item[CDF:] $F(x)=\frac{x-a}{b-a}\Rightarrow\mathbb{P}(c>x)=\frac{c-a}{b-a}$
        \item[$\mathbb{E}(Y_1)$] $=a$+$N\frac{b-a}{N+1}$, $Y_1=\max(X$=$x_1,..,x_N)$
        \item[$\mathbb{E}(Y_2)$] $=a$+$(N$-1$)\frac{b-a}{N+1}$, $Y_2=\max(X\neq Y_1)$
      \end{enumerate}
      \vspace{-6pt}
      Results:
      \vspace{-6pt}
      \begin{itemize}
        \item[\nth{2}:] FOC: $\frac{1}{2c}[(v_i-2b_i)+(d+c)]=0$
      \end{itemize}
      \vfill\null
    \end{multicols}
\end{frame}
\begin{frame}{PS8, Ex. 3.a: First- and second-price sealed bid auctions with two bidders}
    Consider a first-price sealed bid auction with two bidders, who have valuations $v_1$ and $v_2$, respectively. These values are distributed independently uniformly with $v_i\sim U(1,3)$, thus, the values are \textit{private}.
    \vspace{-4pt}
    \begin{itemize}
      \item[(a)] Show that there is a symmetric Bayesian Nash Equilibrium in linear strategies: $b_i(v_i) = cv_i + d\ (*)$. Find \textit{c} and \textit{d}.
    \end{itemize}
    \vspace{-8pt}
    \begin{multicols}{2}
      \begin{itemize}
        \item[\nth{1} step:] Assuming bidder \textit{j} follows the proposed strategy $b_j(v_j) = cv_j + d$, calculate bidder \textit{i}'s expected payoff from bidding $b_i$:
      \end{itemize}
      \vspace{-4pt}
      \begin{align*}
        \mathbb{E}[u_i(b_i,v_i)]&=\mathbb{P}(i\ wins|b_i)(v_i-b_i)\\
                                &=\mathbb{P}(b_i>b_j(v_j))(v_i-b_i)\\
                                &=\mathbb{P}(b_i>cv_j+d)(v_i-b_i)\\
                                &=\mathbb{P}\left(\frac{b_i-d}{c}>v_j\right)(v_i-b_i)\\
                                &=\frac{\frac{b_i-d}{c}-1}{3-1}(v_i-b_i),\text{ using CDF}\\
                                &=\frac{b_i-d-c}{2c}(v_i-b_i)
      \end{align*}
      \vspace{-10pt}
      \begin{itemize}
        \item[\nth{2} step:] Take the FOC and SOC wrt. $b_i$.
      \end{itemize}
      \vfill\null\columnbreak
      For $N$ draws from $x\sim U(a, b):$
      \vspace{-6pt}
      \begin{enumerate}
        \item[PDF:] $f(x)=\frac{1}{b-a}$
        \item[CDF:] $F(x)=\frac{x-a}{b-a}\Rightarrow\mathbb{P}(c>x)=\frac{c-a}{b-a}$
        \item[$\mathbb{E}(Y_1)$] $=a$+$N\frac{b-a}{N+1}$, $Y_1=\max(X$=$x_1,..,x_N)$
        \item[$\mathbb{E}(Y_2)$] $=a$+$(N$-1$)\frac{b-a}{N+1}$, $Y_2=\max(X\neq Y_1)$
      \end{enumerate}
      \vspace{-6pt}
      Results:
      \vspace{-6pt}
      \begin{itemize}
        \item[\nth{2}:] FOC: $\frac{1}{2c}[(v_i-2b_i)+(d+c)]=0$\\
                        SOC: $-\frac{1}{c}=0$\\
                        i.e. expected utility is concave in $b_i$.
      \end{itemize}
      \vfill\null
    \end{multicols}
\end{frame}
\begin{frame}{PS8, Ex. 3.a: First- and second-price sealed bid auctions with two bidders}
    Consider a first-price sealed bid auction with two bidders, who have valuations $v_1$ and $v_2$, respectively. These values are distributed independently uniformly with $v_i\sim U(1,3)$, thus, the values are \textit{private}.
    \vspace{-4pt}
    \begin{itemize}
      \item[(a)] Show that there is a symmetric Bayesian Nash Equilibrium in linear strategies: $b_i(v_i) = cv_i + d\ (*)$. Find \textit{c} and \textit{d}.
    \end{itemize}
    \vspace{-8pt}
    \begin{multicols}{2}
      \begin{itemize}
        \item[\nth{1} step:] Assuming bidder \textit{j} follows the proposed strategy $b_j(v_j) = cv_j + d$, calculate bidder \textit{i}'s expected payoff from bidding $b_i$:
      \end{itemize}
      \vspace{-12pt}
      \begin{align*}
        \mathbb{E}[u_i(b_i,v_i)]&=\mathbb{P}(i\ wins|b_i)(v_i-b_i)\\
                                &=\mathbb{P}(b_i>b_j(v_j))(v_i-b_i)\\
                                &=\mathbb{P}(b_i>cv_j+d)(v_i-b_i)\\
                                &=\mathbb{P}\left(\frac{b_i-d}{c}>v_j\right)(v_i-b_i)\\
                                &=\frac{\frac{b_i-d}{c}-1}{3-1}(v_i-b_i),\text{ using CDF}\\
                                &=\frac{b_i-d-c}{2c}(v_i-b_i)
      \end{align*}
      \vspace{-18pt}
      \begin{itemize}
        \item[\nth{2} step:] Take the FOC and SOC wrt. $b_i$.
        \item[\nth{3} step:] \textbf{To find $c^*$ and $d^*$, compare the best-response function $b_i(v_i)$ to $(*)$.}
      \end{itemize}
      \vfill\null\columnbreak
      For $N$ draws from $x\sim U(a, b):$
      \vspace{-6pt}
      \begin{enumerate}
        \item[PDF:] $f(x)=\frac{1}{b-a}$
        \item[CDF:] $F(x)=\frac{x-a}{b-a}\Rightarrow\mathbb{P}(c>x)=\frac{c-a}{b-a}$
        \item[$\mathbb{E}(Y_1)$] $=a$+$N\frac{b-a}{N+1}$, $Y_1=\max(X$=$x_1,..,x_N)$
        \item[$\mathbb{E}(Y_2)$] $=a$+$(N$-1$)\frac{b-a}{N+1}$, $Y_2=\max(X\neq Y_1)$
      \end{enumerate}
      \vspace{-6pt}
      Results:
      \vspace{-6pt}
      \begin{itemize}
        \item[\nth{2}:] FOC: $\frac{1}{2c}[(v_i-2b_i)+(d+c)]=0$\\
                        SOC: $-\frac{1}{c}=0$\\
                        i.e. expected utility is concave in $b_i$.
      \end{itemize}
      \vfill\null
    \end{multicols}
\end{frame}
\begin{frame}{PS8, Ex. 3.a: First- and second-price sealed bid auctions with two bidders}
    Consider a first-price sealed bid auction with two bidders, who have valuations $v_1$ and $v_2$, respectively. These values are distributed independently uniformly with $v_i\sim U(1,3)$, thus, the values are \textit{private}.
    \vspace{-4pt}
    \begin{itemize}
      \item[(a)] Show that there is a symmetric Bayesian Nash Equilibrium in linear strategies: $b_i(v_i) = cv_i + d\ (*)$. Find \textit{c} and \textit{d}.
    \end{itemize}
    \vspace{-8pt}
    \begin{multicols}{2}
      \begin{itemize}
        \item[\nth{1} step:] Assuming bidder \textit{j} follows the proposed strategy $b_j(v_j) = cv_j + d$, calculate bidder \textit{i}'s expected payoff from bidding $b_i$:
      \end{itemize}
      \vspace{-12pt}
      \begin{align*}
        \mathbb{E}[u_i(b_i,v_i)]&=\mathbb{P}(i\ wins|b_i)(v_i-b_i)\\
                                &=\mathbb{P}(b_i>b_j(v_j))(v_i-b_i)\\
                                &=\mathbb{P}(b_i>cv_j+d)(v_i-b_i)\\
                                &=\mathbb{P}\left(\frac{b_i-d}{c}>v_j\right)(v_i-b_i)\\
                                &=\frac{\frac{b_i-d}{c}-1}{3-1}(v_i-b_i),\text{ CDF}\\
                                &=\frac{b_i-d-c}{2c}(v_i-b_i)
      \end{align*}
      \vspace{-18pt}
      \begin{itemize}
        \item[\nth{2} step:] Take the FOC and SOC wrt. $b_i$.
        \item[\nth{3} step:] \textbf{To find $c^*$ and $d^*$, compare the best-response function $b_i(v_i)$ to $(*)$.}
      \end{itemize}
      \vfill\null\columnbreak
      For $N$ draws from $x\sim U(a, b):$
      \vspace{-6pt}
      \begin{enumerate}
        \item[PDF:] $f(x)=\frac{1}{b-a}$
        \item[CDF:] $F(x)=\frac{x-a}{b-a}\Rightarrow\mathbb{P}(c>x)=\frac{c-a}{b-a}$
        \item[$\mathbb{E}(Y_1)$] $=a$+$N\frac{b-a}{N+1}$, $Y_1=\max(X$=$x_1,..,x_N)$
        \item[$\mathbb{E}(Y_2)$] $=a$+$(N$-1$)\frac{b-a}{N+1}$, $Y_2=\max(X\neq Y_1)$
      \end{enumerate}
      \vspace{-6pt}
      Results:
      \vspace{-6pt}
      \begin{itemize}
        \item[\nth{2}:] FOC: $\frac{1}{2c}[(v_i-2b_i)+(d+c)]=0$\\
                        SOC: $-\frac{1}{c}=0$\\
                        i.e. expected utility is concave in $b_i$.
        \item[\nth{3}:] From the FOC, the BR function is:\vspace{-6pt}
                        \begin{align*}
                          2b_i&=v_i+d+c\Rightarrow\\
                          b_i(v_i)&=\underbrace{\frac{1}{2}}_{c^*}v_1+\underbrace{\frac{1}{2}(d+c)}_{d^*}
                        \end{align*}
      \end{itemize}
      \vfill\null
    \end{multicols}
\end{frame}
\begin{frame}{PS8, Ex. 3.a: First- and second-price sealed bid auctions with two bidders}
    Consider a first-price sealed bid auction with two bidders, who have valuations $v_1$ and $v_2$, respectively. These values are distributed independently uniformly with $v_i\sim U(1,3)$, thus, the values are \textit{private}.
    \vspace{-4pt}
    \begin{itemize}
      \item[(a)] Show that there is a symmetric Bayesian Nash Equilibrium in linear strategies: $b_i(v_i) = cv_i + d\ (*)$. Find \textit{c} and \textit{d}.
    \end{itemize}
    \vspace{-8pt}
    \begin{multicols}{2}
      \begin{itemize}
        \item[\nth{1} step:] Assuming bidder \textit{j} follows the proposed strategy $b_j(v_j) = cv_j + d$, calculate bidder \textit{i}'s expected payoff from bidding $b_i$:
      \end{itemize}
      \vspace{-12pt}
      \begin{align*}
        \mathbb{E}[u_i(b_i,v_i)]&=\mathbb{P}(i\ wins|b_i)(v_i-b_i)\\
                                &=\mathbb{P}(b_i>b_j(v_j))(v_i-b_i)\\
                                &=\mathbb{P}(b_i>cv_j+d)(v_i-b_i)\\
                                &=\mathbb{P}\left(\frac{b_i-d}{c}>v_j\right)(v_i-b_i)\\
                                &=\frac{\frac{b_i-d}{c}-1}{3-1}(v_i-b_i),\text{ CDF}\\
                                &=\frac{b_i-d-c}{2c}(v_i-b_i)
      \end{align*}
      \vspace{-18pt}
      \begin{itemize}
        \item[\nth{2} step:] Take the FOC and SOC wrt. $b_i$.
        \item[\nth{3} step:] To find $c^*$ and $d^*$, compare the best-response function $b_i(v_i)$ to $(*)$.
      \end{itemize}
      \vfill\null\columnbreak
      For $N$ draws from $x\sim U(a, b):$
      \vspace{-6pt}
      \begin{enumerate}
        \item[PDF:] $f(x)=\frac{1}{b-a}$
        \item[CDF:] $F(x)=\frac{x-a}{b-a}\Rightarrow\mathbb{P}(c>x)=\frac{c-a}{b-a}$
        \item[$\mathbb{E}(Y_1)$] $=a$+$N\frac{b-a}{N+1}$, $Y_1=\max(X$=$x_1,..,x_N)$
        \item[$\mathbb{E}(Y_2)$] $=a$+$(N$-1$)\frac{b-a}{N+1}$, $Y_2=\max(X\neq Y_1)$
        \item[\nth{2}:] FOC: $\frac{1}{2c}[(v_i-2b_i)+(d+c)]=0$\\
                        SOC: $-\frac{1}{c}=0$\\
                        i.e. expected utility is concave in $b_i$.
        \item[\nth{3}:] From the FOC, the BR function is:\vspace{-6pt}
                        \begin{align*}
                          b_i(v_i)=\underbrace{\frac{1}{2}}_{c^*}v_1+\underbrace{\frac{1}{2}(d+c)}_{d^*}
                        \end{align*}
        \item[]         \vspace{-6pt} Inserting the first term in the second term, $d^*=\frac{1}{2}(d^*+c^*)=\frac{1}{2}(d^*+\frac{1}{2})$, which solves for $c^*=d^*=\frac{1}{2}$.
      \end{enumerate}
      \vfill\null
    \end{multicols}
\end{frame}


\begin{frame}{PS8, Ex. 3.b: First- and second-price sealed bid auctions with two bidders}
    \begin{multicols}{2}
      \begin{itemize}
        \item[(b)] Calculate the revenue to the seller.
      \end{itemize}
      \vfill\null\columnbreak
      For $N$ draws from $x\sim U(a, b):$
      \vspace{-6pt}
      \begin{enumerate}
        \item[PDF:] $f(x)=\frac{1}{b-a}$
        \item[CDF:] $F(x)=\frac{x-a}{b-a}\Rightarrow\mathbb{P}(c>x)=\frac{c-a}{b-a}$
        \item[$\mathbb{E}(Y_1)$] $=a$+$N\frac{b-a}{N+1}$, $Y_1=\max(X$=$x_1,..,x_N)$
        \item[$\mathbb{E}(Y_2)$] $=a$+$(N$-1$)\frac{b-a}{N+1}$, $Y_2=\max(X\neq Y_1)$
      \end{enumerate}
      \vspace{-6pt}
      Results so far:
      \vspace{-6pt}
      \begin{itemize}
        \item[($*$)]  $b_i(v_i) = cv_i+d$
        \item[($**$)] $\mathbb{P}(i\ wins|v_i)=\frac{b_i(v_i)-d-c}{2c}=\frac{cv_i-c}{2c}$
        \item[(3.a)]    $c^*=d^*=\frac{1}{2}$
      \end{itemize}
      \vfill\null
    \end{multicols}
\end{frame}
\begin{frame}{PS8, Ex. 3.b: First- and second-price sealed bid auctions with two bidders}
    \begin{multicols}{2}
      \begin{itemize}
        \item[(b)] Calculate the revenue to the seller.
        \item[\nth{1} step:] \textbf{Calculate the expected highest value $v_i$ for $N=2$ draws from the uniform distribution $v_i\sim U(1,3)$.}
      \end{itemize}
      \vfill\null\columnbreak
      For $N$ draws from $x\sim U(a, b):$
      \vspace{-6pt}
      \begin{enumerate}
        \item[PDF:] $f(x)=\frac{1}{b-a}$
        \item[CDF:] $F(x)=\frac{x-a}{b-a}\Rightarrow\mathbb{P}(c>x)=\frac{c-a}{b-a}$
        \item[$\mathbb{E}(Y_1)$] $=a$+$N\frac{b-a}{N+1}$, $Y_1=\max(X$=$x_1,..,x_N)$
        \item[$\mathbb{E}(Y_2)$] $=a$+$(N$-1$)\frac{b-a}{N+1}$, $Y_2=\max(X\neq Y_1)$
      \end{enumerate}
      \vspace{-6pt}
      Results so far:
      \vspace{-6pt}
      \begin{itemize}
        \item[($*$)]  $b_i(v_i) = cv_i+d$
        \item[($**$)] $\mathbb{P}(i\ wins|v_i)=\frac{b_i(v_i)-d-c}{2c}=\frac{cv_i-c}{2c}$
        \item[(3.a)]    $c^*=d^*=\frac{1}{2}$
      \end{itemize}
      \vfill\null
    \end{multicols}
\end{frame}
\begin{frame}{PS8, Ex. 3.b: First- and second-price sealed bid auctions with two bidders}
    \begin{multicols}{2}
      \begin{itemize}
        \item[(b)] Calculate the revenue to the seller.
        \item[\nth{1} step:] Calculate the expected highest value $v_i$ for $N=2$ draws from the uniform distribution $v_i\sim U(1,3)$:
      \end{itemize}
      \vspace{-8pt}
      \begin{align*}
        \mathbb{E}[Y_1]&=a+N\frac{b-a}{N+1}\\
                       &=1+2\frac{3-1}{2+1}\\
                       &=1+\frac{4}{3}=\frac{7}{3}
      \end{align*}
      \vfill\null\columnbreak
      For $N$ draws from $x\sim U(a, b):$
      \vspace{-6pt}
      \begin{enumerate}
        \item[PDF:] $f(x)=\frac{1}{b-a}$
        \item[CDF:] $F(x)=\frac{x-a}{b-a}\Rightarrow\mathbb{P}(c>x)=\frac{c-a}{b-a}$
        \item[$\mathbb{E}(Y_1)$] $=a$+$N\frac{b-a}{N+1}$, $Y_1=\max(X$=$x_1,..,x_N)$
        \item[$\mathbb{E}(Y_2)$] $=a$+$(N$-1$)\frac{b-a}{N+1}$, $Y_2=\max(X\neq Y_1)$
      \end{enumerate}
      \vspace{-6pt}
      Results so far:
      \vspace{-6pt}
      \begin{itemize}
        \item[($*$)]  $b_i(v_i) = cv_i+d$
        \item[($**$)] $\mathbb{P}(i\ wins|v_i)=\frac{b_i(v_i)-d-c}{2c}=\frac{cv_i-c}{2c}$
        \item[(3.a)]    $c^*=d^*=\frac{1}{2}$
        \item[\nth{1}:] $\mathbb{E}[Y_1]=\frac{7}{3}$
      \end{itemize}
      \vfill\null
    \end{multicols}
\end{frame}
\begin{frame}{PS8, Ex. 3.b: First- and second-price sealed bid auctions with two bidders}
    \begin{multicols}{2}
      \begin{itemize}
        \item[(b)] Calculate the revenue to the seller.
        \item[\nth{1} step:] Calculate the expected highest value $v_i$ for $N=2$ draws from the uniform distribution $v_i\sim U(1,3)$:
      \end{itemize}
      \vspace{-8pt}
      \begin{align*}
        \mathbb{E}[Y_1]&=a+N\frac{b-a}{N+1}\\
                       &=1+2\frac{3-1}{2+1}\\
                       &=1+\frac{4}{3}=\frac{7}{3}
      \end{align*}
      \vspace{-12pt}
      \begin{itemize}
        \item[\nth{2} step:] \textbf{Insert the expected highest value $\mathbb{E}[Y_1]$ in the bid-function $(*)$  to find the seller's expected revenue.}
      \end{itemize}
      \vfill\null\columnbreak
      For $N$ draws from $x\sim U(a, b):$
      \vspace{-6pt}
      \begin{enumerate}
        \item[PDF:] $f(x)=\frac{1}{b-a}$
        \item[CDF:] $F(x)=\frac{x-a}{b-a}\Rightarrow\mathbb{P}(c>x)=\frac{c-a}{b-a}$
        \item[$\mathbb{E}(Y_1)$] $=a$+$N\frac{b-a}{N+1}$, $Y_1=\max(X$=$x_1,..,x_N)$
        \item[$\mathbb{E}(Y_2)$] $=a$+$(N$-1$)\frac{b-a}{N+1}$, $Y_2=\max(X\neq Y_1)$
      \end{enumerate}
      \vspace{-6pt}
      Results so far:
      \vspace{-6pt}
      \begin{itemize}
        \item[($*$)]  $b_i(v_i) = cv_i+d$
        \item[($**$)] $\mathbb{P}(i\ wins|v_i)=\frac{b_i(v_i)-d-c}{2c}=\frac{cv_i-c}{2c}$
        \item[(3.a)]    $c^*=d^*=\frac{1}{2}$
        \item[\nth{1}:] $\mathbb{E}[Y_1]=\frac{7}{3}$
      \end{itemize}
      \vfill\null
    \end{multicols}
\end{frame}
\begin{frame}{PS8, Ex. 3.b: First- and second-price sealed bid auctions with two bidders}
    \begin{multicols}{2}
      \begin{itemize}
        \item[(b)] Calculate the revenue to the seller.
        \item[\nth{1} step:] Calculate the expected highest value $v_i$ for $N=2$ draws from the uniform distribution $v_i\sim U(1,3)$:
      \end{itemize}
      \vspace{-8pt}
      \begin{align*}
        \mathbb{E}[Y_1]&=a+N\frac{b-a}{N+1}\\
                       &=1+2\frac{3-1}{2+1}\\
                       &=1+\frac{4}{3}=\frac{7}{3}
      \end{align*}
      \vspace{-12pt}
      \begin{itemize}
        \item[\nth{2} step:] Insert the expected highest value $\mathbb{E}[Y_1]$ in the bid-function $(*)$  to find the seller's expected revenue:
      \end{itemize}
      \vspace{-8pt}
      \begin{align*}
        \mathbb{E}[revenue]&=b_i\left(\frac{7}{3}\right)\\
                           &= c^*\frac{7}{3}+d^*\\
                           &= \frac{1}{2}\cdot\frac{7}{3}+\frac{1}{2},&&\text{using (3.a)}\\
                           &= \frac{7}{6}+\frac{3}{6}=\frac{10}{6}=\frac{5}{3}
      \end{align*}
      \vfill\null\columnbreak
      For $N$ draws from $x\sim U(a, b):$
      \vspace{-6pt}
      \begin{enumerate}
        \item[PDF:] $f(x)=\frac{1}{b-a}$
        \item[CDF:] $F(x)=\frac{x-a}{b-a}\Rightarrow\mathbb{P}(c>x)=\frac{c-a}{b-a}$
        \item[$\mathbb{E}(Y_1)$] $=a$+$N\frac{b-a}{N+1}$, $Y_1=\max(X$=$x_1,..,x_N)$
        \item[$\mathbb{E}(Y_2)$] $=a$+$(N$-1$)\frac{b-a}{N+1}$, $Y_2=\max(X\neq Y_1)$
      \end{enumerate}
      \vspace{-6pt}
      Results so far:
      \vspace{-6pt}
      \begin{itemize}
        \item[($*$)]  $b_i(v_i) = cv_i+d$
        \item[($**$)] $\mathbb{P}(i\ wins|v_i)=\frac{b_i(v_i)-d-c}{2c}=\frac{cv_i-c}{2c}$
        \item[(3.a)]    $c^*=d^*=\frac{1}{2}$
        \item[\nth{1}:] $\mathbb{E}[Y_1]=\frac{7}{3}$
      \end{itemize}
      \vfill\null
    \end{multicols}
\end{frame}


\begin{frame}{PS8, Ex. 3.c: First- and second-price sealed bid auctions with two bidders}
    \begin{multicols}{2}
      \begin{itemize}
        \item[(c)] Suppose now that the object is sold by a \textit{second-price sealed bid auction}.
        \begin{itemize}\normalsize
          \item[i.]   Suppose player 2 bids his valuation: $b_2(v_2) = v_2$. Write down the expected payoffs to player 1 from bidding $b_1$.
          \item[ii.]  Using your previous answer, argue that there is a symmetric Bayesian Nash Equilibrium (BNE) in which both players bid their valuation.
          \item[iii.] Calculate the revenue to the seller from this equilibrium. Compare to the answer in (b).
        \end{itemize}
      \end{itemize}
      For $N$ draws from $x\sim U(a, b):$
      \vspace{-6pt}
      \begin{enumerate}
        \item[PDF:] $f(x)=\frac{1}{b-a}$
        \item[CDF:] $F(x)=\frac{x-a}{b-a}\Rightarrow\mathbb{P}(c>x)=\frac{c-a}{b-a}$
        \item[$\mathbb{E}(Y_1)$] $=a$+$N\frac{b-a}{N+1}$, $Y_1=\max(X$=$x_1,..,x_N)$
        \item[$\mathbb{E}(Y_2)$] $=a$+$(N$-1$)\frac{b-a}{N+1}$, $Y_2=\max(X\neq Y_1)$
      \end{enumerate}
      \vfill\null\columnbreak
      \vfill\null
    \end{multicols}
\end{frame}


\begin{frame}{PS8, Ex. 3.c.i: First- and second-price sealed bid auctions with two bidders}
    \begin{multicols}{2}
      \begin{itemize}
        \item[(c)] Suppose now that the object is sold by a \textit{second-price sealed bid auction}.
        \begin{itemize}\normalsize
          \item[i.]   Suppose player 2 bids his valuation: $b_2(v_2) = v_2$. Write down the expected payoffs to player 1 from bidding $b_1$.
          \item[ii.]  \textbf{Using your previous answer, argue that there is a symmetric Bayesian Nash Equilibrium (BNE) in which both players bid their valuation.}
          \item[iii.] Calculate the revenue to the seller from this equilibrium. Compare to the answer in (b).
        \end{itemize}
      \end{itemize}
      For $N$ draws from $x\sim U(a, b):$
      \vspace{-6pt}
      \begin{enumerate}
        \item[PDF:] $f(x)=\frac{1}{b-a}$
        \item[CDF:] $F(x)=\frac{x-a}{b-a}\Rightarrow\mathbb{P}(c>x)=\frac{c-a}{b-a}$
        \item[$\mathbb{E}(Y_1)$] $=a$+$N\frac{b-a}{N+1}$, $Y_1=\max(X$=$x_1,..,x_N)$
        \item[$\mathbb{E}(Y_2)$] $=a$+$(N$-1$)\frac{b-a}{N+1}$, $Y_2=\max(X\neq Y_1)$
      \end{enumerate}
      \vfill\null\columnbreak
      \begin{itemize}
        \item[(i)] The expected payoffs of P1 given $b_2$:
      \end{itemize}
      \vspace{-12pt}
      \begin{align*}
        u_1(b_1,b_2)=\left\{\begin{array}{lcl}
          v_1-b_2     & \text{if} & b_1>b_2 \\
          (v_1-b_2)/2 & \text{if} & b_1=b_2 \\
          0           & \text{if} & b_1<b_2
        \end{array}\right.
      \end{align*}
      \vfill\null
    \end{multicols}
\end{frame}


\begin{frame}{PS8, Ex. 3.c.ii: First- and second-price sealed bid auctions with two bidders}
    \begin{multicols}{2}
      \begin{itemize}
        \item[(c)] Suppose now that the object is sold by a \textit{second-price sealed bid auction}.
        \begin{itemize}\normalsize
          \item[i.]   Suppose player 2 bids his valuation: $b_2(v_2) = v_2$. Write down the expected payoffs to player 1 from bidding $b_1$.
          \item[ii.]  Using your previous answer, argue that there is a symmetric Bayesian Nash Equilibrium (BNE) in which both players bid their valuation.
          \item[iii.] \textbf{Calculate the revenue to the seller from this equilibrium. Compare to the answer in (b).}
        \end{itemize}
      \end{itemize}
      For $N$ draws from $x\sim U(a, b):$
      \vspace{-6pt}
      \begin{enumerate}
        \item[PDF:] $f(x)=\frac{1}{b-a}$
        \item[CDF:] $F(x)=\frac{x-a}{b-a}\Rightarrow\mathbb{P}(c>x)=\frac{c-a}{b-a}$
        \item[$\mathbb{E}(Y_1)$] $=a$+$N\frac{b-a}{N+1}$, $Y_1=\max(X$=$x_1,..,x_N)$
        \item[$\mathbb{E}(Y_2)$] $=a$+$(N$-1$)\frac{b-a}{N+1}$, $Y_2=\max(X\neq Y_1)$
      \end{enumerate}
      \vfill\null\columnbreak
      \begin{itemize}
        \item[(i)] The expected payoffs of P1 given $b_2$:
      \end{itemize}
      \vspace{-12pt}
      \begin{align*}
        u_1(b_1,b_2)=\left\{\begin{array}{lcl}
          v_1-b_2     & \text{if} & b_1>b_2 \\
          (v_1-b_2)/2 & \text{if} & b_1=b_2 \\
          0           & \text{if} & b_1<b_2
        \end{array}\right.
      \end{align*}
      \vspace{-18pt}
      \begin{itemize}
        \item[(ii)] P1 wins: Payoff is independent of $b_1$ unless $b_1<b_2$, in which case P1 no longer wins, thus, gets zero payoff.
        \item[] P1 looses: Payoff is independent of $b_1$ unless $b_1>b_2$, in which case P1 wins instead but bids more than her evaluation and gets negative payoff.
        \item[] i.e. there is no incentive to deviate from $BNE=(b_1^*,b_2^*)=\{(v_1,v_2)\}$.
      \end{itemize}
      \vfill\null
    \end{multicols}
\end{frame}


\begin{frame}{PS8, Ex. 3.c.iii: First- and second-price sealed bid auctions with two bidders}
    \begin{multicols}{2}
      \begin{itemize}
        \item[(c)] Suppose now that the object is sold by a \textit{second-price sealed bid auction}.
        \begin{itemize}\normalsize
          \item[i.]   Suppose player 2 bids his valuation: $b_2(v_2) = v_2$. Write down the expected payoffs to player 1 from bidding $b_1$.
          \item[ii.]  Using your previous answer, argue that there is a symmetric Bayesian Nash Equilibrium (BNE) in which both players bid their valuation.
          \item[iii.] \textbf{Calculate the revenue to the seller from this equilibrium. Compare to the answer in (b).}
        \end{itemize}
      \end{itemize}
      For $N$ draws from $x\sim U(a, b):$
      \vspace{-6pt}
      \begin{enumerate}
        \item[PDF:] $f(x)=\frac{1}{b-a}$
        \item[CDF:] $F(x)=\frac{x-a}{b-a}\Rightarrow\mathbb{P}(c>x)=\frac{c-a}{b-a}$
        \item[$\mathbb{E}(Y_1)$] $=a$+$N\frac{b-a}{N+1}$, $Y_1=\max(X$=$x_1,..,x_N)$
        \item[$\mathbb{E}(Y_2)$] $=a$+$(N$-1$)\frac{b-a}{N+1}$, $Y_2=\max(X\neq Y_1)$
      \end{enumerate}
      \vfill\null\columnbreak
      \begin{itemize}
        \item[(i)] The expected payoffs of P1 given $b_2$:
      \end{itemize}
      \vspace{-16pt}
      \begin{align*}
        u_1(b_1,b_2)=\left\{\begin{array}{lcl}
          v_1-b_2     & \text{if} & b_1>b_2 \\
          (v_1-b_2)/2 & \text{if} & b_1=b_2 \\
          0           & \text{if} & b_1<b_2
        \end{array}\right.
      \end{align*}
      \vspace{-18pt}
      \begin{itemize}
        \item[(ii)] There is no incentive to deviate from $BNE=(b_1^*,b_2^*)=\{(v_1,v_2)\}$.
        \item[(iii)] \textbf{Calculate player \textit{i}'s expected payment in the BNE.}
      \end{itemize}
      \vfill\null
    \end{multicols}
\end{frame}
\begin{frame}{PS8, Ex. 3.c.iii: First- and second-price sealed bid auctions with two bidders}
    \begin{multicols}{2}
      \begin{itemize}
        \item[(c)] Suppose now that the object is sold by a \textit{second-price sealed bid auction}.
        \begin{itemize}\normalsize
          \item[i.]   Suppose player 2 bids his valuation: $b_2(v_2) = v_2$. Write down the expected payoffs to player 1 from bidding $b_1$.
          \item[ii.]  Using your previous answer, argue that there is a symmetric Bayesian Nash Equilibrium (BNE) in which both players bid their valuation.
          \item[iii.] \textbf{Calculate the revenue to the seller from this equilibrium. Compare to the answer in (b).}
        \end{itemize}
      \end{itemize}
      For $N$ draws from $x\sim U(a, b):$
      \vspace{-6pt}
      \begin{enumerate}
        \item[PDF:] $f(x)=\frac{1}{b-a}$
        \item[CDF:] $F(x)=\frac{x-a}{b-a}\Rightarrow\mathbb{P}(c>x)=\frac{c-a}{b-a}$
        \item[$\mathbb{E}(Y_1)$] $=a$+$N\frac{b-a}{N+1}$, $Y_1=\max(X$=$x_1,..,x_N)$
        \item[$\mathbb{E}(Y_2)$] $=a$+$(N$-1$)\frac{b-a}{N+1}$, $Y_2=\max(X\neq Y_1)$
      \end{enumerate}
      \vfill\null\columnbreak
      \begin{itemize}
        \item[(i)] The expected payoffs of P1 given $b_2$:
      \end{itemize}
      \vspace{-16pt}
      \begin{align*}
        u_1(b_1,b_2)=\left\{\begin{array}{lcl}
          v_1-b_2     & \text{if} & b_1>b_2 \\
          (v_1-b_2)/2 & \text{if} & b_1=b_2 \\
          0           & \text{if} & b_1<b_2
        \end{array}\right.
      \end{align*}
      \vspace{-18pt}
      \begin{itemize}
        \item[(ii)] There is no incentive to deviate from $BNE=(b_1^*,b_2^*)=\{(v_1,v_2)\}$.
        \item[(iii)] \textbf{Calculate player \textit{i}'s expected payment in the BNE:}
      \end{itemize}
      \vspace{-12pt}
      \begin{align*}
        m_i(v_i)&=\mathbb{P}(i\ wins|v_i)\cdot\mathbb{E}[b_j^*(v_j)|b_j^*(v_j)<b_i^*(v_i)]
      \end{align*}
      \vfill\null
    \end{multicols}
\end{frame}
\begin{frame}{PS8, Ex. 3.c.iii: First- and second-price sealed bid auctions with two bidders}
    \begin{multicols}{2}
      \begin{itemize}
        \item[(c)] Suppose now that the object is sold by a \textit{second-price sealed bid auction}.
        \begin{itemize}\normalsize
          \item[i.]   Suppose player 2 bids his valuation: $b_2(v_2) = v_2$. Write down the expected payoffs to player 1 from bidding $b_1$.
          \item[ii.]  Using your previous answer, argue that there is a symmetric Bayesian Nash Equilibrium (BNE) in which both players bid their valuation.
          \item[iii.] \textbf{Calculate the revenue to the seller from this equilibrium. Compare to the answer in (b).}
        \end{itemize}
      \end{itemize}
      For $N$ draws from $x\sim U(a, b):$
      \vspace{-6pt}
      \begin{enumerate}
        \item[PDF:] $f(x)=\frac{1}{b-a}$
        \item[CDF:] $F(x)=\frac{x-a}{b-a}\Rightarrow\mathbb{P}(c>x)=\frac{c-a}{b-a}$
        \item[$\mathbb{E}(Y_1)$] $=a$+$N\frac{b-a}{N+1}$, $Y_1=\max(X$=$x_1,..,x_N)$
        \item[$\mathbb{E}(Y_2)$] $=a$+$(N$-1$)\frac{b-a}{N+1}$, $Y_2=\max(X\neq Y_1)$
      \end{enumerate}
      \vfill\null\columnbreak
      \begin{itemize}
        \item[(i)] The expected payoffs of P1 given $b_2$:
      \end{itemize}
      \vspace{-16pt}
      \begin{align*}
        u_1(b_1,b_2)=\left\{\begin{array}{lcl}
          v_1-b_2     & \text{if} & b_1>b_2 \\
          (v_1-b_2)/2 & \text{if} & b_1=b_2 \\
          0           & \text{if} & b_1<b_2
        \end{array}\right.
      \end{align*}
      \vspace{-18pt}
      \begin{itemize}
        \item[(ii)] There is no incentive to deviate from $BNE=(b_1^*,b_2^*)=\{(v_1,v_2)\}$.
        \item[(iii)] Player \textit{i}'s expected payment in BNE:
      \end{itemize}
      \vspace{-12pt}
      \begin{align*}
        m_i(v_i)&=\mathbb{P}(i\ wins|v_i)\cdot\mathbb{E}[b_j^*(v_j)|b_j^*(v_j)<b_i^*(v_i)]\\
                &=\mathbb{P}(v_i>v_j)\cdot\mathbb{E}[v_j|v_j<v_i]\\
                &=\frac{v_i-1}{3-1}\cdot\frac{1+v_i}{2},\text{ using CDF and Mean}\\
                &=\frac{v_i+v_i^2-1^2-v_i}{2^2}=\frac{v_i^2-1}{4}
      \end{align*}
      \vfill\null
    \end{multicols}
\end{frame}
\begin{frame}{PS8, Ex. 3.c.iii: First- and second-price sealed bid auctions with two bidders}
    \begin{multicols}{2}
      \begin{itemize}
        \item[(c)] Suppose now that the object is sold by a \textit{second-price sealed bid auction}.
        \begin{itemize}\normalsize
          \item[i.]   Suppose player 2 bids his valuation: $b_2(v_2) = v_2$. Write down the expected payoffs to player 1 from bidding $b_1$.
          \item[ii.]  Using your previous answer, argue that there is a symmetric Bayesian Nash Equilibrium (BNE) in which both players bid their valuation.
          \item[iii.] Calculate the revenue to the seller from this equilibrium. Compare to the answer in (b).
        \end{itemize}
      \end{itemize}
      For $N$ draws from $x\sim U(a, b):$
      \vspace{-6pt}
      \begin{enumerate}
        \item[PDF:] $f(x)=\frac{1}{b-a}$
        \item[CDF:] $F(x)=\frac{x-a}{b-a}\Rightarrow\mathbb{P}(c>x)=\frac{c-a}{b-a}$
        \item[$\mathbb{E}(Y_1)$] $=a$+$N\frac{b-a}{N+1}$, $Y_1=\max(X$=$x_1,..,x_N)$
        \item[$\mathbb{E}(Y_2)$] $=a$+$(N$-1$)\frac{b-a}{N+1}$, $Y_2=\max(X\neq Y_1)$
      \end{enumerate}
      \vfill\null\columnbreak
      \begin{itemize}
        \item[(i)] The expected payoffs of P1 given $b_2$:
      \end{itemize}
      \vspace{-16pt}
      \begin{align*}
        u_1(b_1,b_2)=\left\{\begin{array}{lcl}
          v_1-b_2     & \text{if} & b_1>b_2 \\
          (v_1-b_2)/2 & \text{if} & b_1=b_2 \\
          0           & \text{if} & b_1<b_2
        \end{array}\right.
      \end{align*}
      \vspace{-18pt}
      \begin{itemize}
        \item[(ii)] There is no incentive to deviate from $BNE=(b_1^*,b_2^*)=\{(v_1,v_2)\}$.
        \item[(iii)] Player \textit{i}'s expected payment in BNE:
      \end{itemize}
      \vspace{-12pt}
      \begin{align*}
        m_i(v_i)&=\mathbb{P}(i\ wins|v_i)\cdot\mathbb{E}[b_j^*(v_j)|b_j^*(v_j)<b_i^*(v_i)]\\
                &=\mathbb{P}(v_i>v_j)\cdot\mathbb{E}[v_j|v_j<v_i]\\
                &=\frac{v_i-1}{3-1}\cdot\frac{1+v_i}{2},\text{ using CDF and Mean}\\
                &=\frac{v_i+v_i^2-1^2-v_i}{2^2}=\frac{v_i^2-1}{4}
      \end{align*}
      As this is the same as in (3.b), we know:
      \vspace{-8pt}
      \begin{align*}\vspace{-4pt}
        \text{Ex-ante expected payment}&=\mathbb{E}[m_i(v_i)]=\frac{5}{6}\\
        \text{Seller's revenue}&=2\cdot\mathbb{E}[m_i(v_i)]=\frac{5}{3}
      \end{align*}
      Thus, the outcome is the exact same as for the \textit{first-price sealed bid auction}.
      \vfill\null
    \end{multicols}
\end{frame}



\section{PS8, Ex. 4: First-price sealed bid auctions with three bidders}

\begin{frame}{PS8, Ex. 4: First-price sealed bid auctions with three bidders}
    Consider the auction setting of the previous exercise. But now suppose that there are three identical bidders, $i = 1, 2, 3$, with values $v_i$ where
    \begin{align*}
      v_i\sim U(1, 3)
    \end{align*}
    and the values are independent, i.e. private. The auction is first-price sealed bid.
    \begin{itemize}
      \item[(a)] Again, show that there is a symmetric Bayesian Nash Equilibrium in linear strategies: $b_i(v_i) = cv_i + d\ (*)$. Find \textit{c} and \textit{d}.
      \item[(b)] Do you expect seller to earn a higher or a lower revenue than in the previous auction? What is causing this effect?
      \item[(c)] (More difficult). Calculate the revenue to the seller.
    \end{itemize}
    \vfill\null
\end{frame}


\begin{frame}{PS8, Ex. 4.a: First-price sealed bid auctions with three bidders}
    \begin{itemize}
      \item[(a)] For three bidders, show that there is a symmetric Bayesian Nash Equilibrium in linear strategies: $b_i(v_i) = cv_i + d\ (*)$. Find \textit{c} and \textit{d}.
      \item[Hint:] Use that $v_j$ and $v_k$ are independent (private) to write bidder \textit{i}'s expected payoff in the proposed equilibrium.
    \end{itemize}
    \vfill\null
\end{frame}
\begin{frame}{PS8, Ex. 4.a: First-price sealed bid auctions with three bidders}
    \begin{itemize}
      \item[(a)] For three bidders, show that there is a symmetric Bayesian Nash Equilibrium in linear strategies: $b_i(v_i) = cv_i + d\ (*)$. Find \textit{c} and \textit{d}.
      \item[Hint:] Use that $v_j$ and $v_k$ are independent (private) to write bidder \textit{i}'s expected payoff in the proposed equilibrium:
    \end{itemize}
    \vspace{-10pt}
    \begin{align*}
      \mathbb{E}[u_i(b_i,v_i)]
      &=\mathbb{P}(i\ wins|b_i)(v_i-b_i)
    \end{align*}
    \vfill\null
\end{frame}
\begin{frame}{PS8, Ex. 4.a: First-price sealed bid auctions with three bidders}
    \begin{itemize}
      \item[(a)] For three bidders, show that there is a symmetric Bayesian Nash Equilibrium in linear strategies: $b_i(v_i) = cv_i + d\ (*)$. Find \textit{c} and \textit{d}.
      \item[Hint:] Use that $v_j$ and $v_k$ are independent (private) to write \textit{i}'s expected payoff in eq.:
    \end{itemize}
    \vspace{-10pt}
    \begin{align*}
      \mathbb{E}[u_i(b_i,v_i)]
      &=\mathbb{P}(i\ wins|b_i)(v_i-b_i)\\
      &=\mathbb{P}\left(b_i>b_j(v_j),b_i>b_k(v_k)\right)(v_i-b_i)\\
      &=\mathbb{P}(b_i>cv_j+d,b_i>cv_k+d)(v_i-b_i),&&\text{using }(*)\\
      &=\mathbb{P}\left(\frac{b_i-d}{c}>v_j,\frac{b_i-d}{c}>v_k\right)(v_i-b_i)\\
      &=\mathbb{P}\left(\frac{b_i-d}{c}>v_j\right)\times\mathbb{P}\left(\frac{b_i-d}{c}>v_j\right)(v_i-b_i)\\
      &=\mathbb{P}\left(\frac{b_i-d}{c}>v_j\right)\times\mathbb{P}\left(\frac{b_i-d}{c}>v_j\right)(v_i-b_i)\\
      &=\left(\frac{b_i-d-c}{2c}\right)^2(v_i-b_i),&&\text{using ex. (3.a)}
    \end{align*}
    \vfill\null
\end{frame}
\begin{frame}{PS8, Ex. 4.a: First-price sealed bid auctions with three bidders}
    \begin{itemize}
      \item[(a)] For three bidders, show that there is a symmetric Bayesian Nash Equilibrium in linear strategies: $b_i(v_i) = cv_i + d\ (*)$. Find \textit{c} and \textit{d}.
      \item[Hint:] Use that $v_j$ and $v_k$ are independent (private) to write \textit{i}'s expected payoff in eq.:
    \end{itemize}
    \vspace{-10pt}
    \begin{align*}
      \mathbb{E}[u_i(b_i,v_i)]
      &=\mathbb{P}(i\ wins|b_i)(v_i-b_i)\\
      &=\mathbb{P}\left(b_i>b_j(v_j),b_i>b_k(v_k)\right)(v_i-b_i)\\
      &=\mathbb{P}(b_i>cv_j+d,b_i>cv_k+d)(v_i-b_i),&&\text{using }(*)\\
      &=\mathbb{P}\left(\frac{b_i-d}{c}>v_j,\frac{b_i-d}{c}>v_k\right)(v_i-b_i)\\
      &=\mathbb{P}\left(\frac{b_i-d}{c}>v_j\right)\times\mathbb{P}\left(\frac{b_i-d}{c}>v_j\right)(v_i-b_i)\\
      &=\mathbb{P}\left(\frac{b_i-d}{c}>v_j\right)\times\mathbb{P}\left(\frac{b_i-d}{c}>v_j\right)(v_i-b_i)\\
      &=\left(\frac{b_i-d-c}{2c}\right)^2(v_i-b_i),&&\text{using ex. (3.a)}
    \end{align*}
    \vspace{-8pt}
    \textbf{Take the FOC and isolate $b_i^{**}(v_i)$.}
    \vfill\null
\end{frame}
\begin{frame}{PS8, Ex. 4.a: First-price sealed bid auctions with three bidders}
    \begin{itemize}
      \item[(a)] For three bidders, show that there is a symmetric Bayesian Nash Equilibrium in linear strategies: $b_i(v_i) = cv_i + d\ (*)$. Find \textit{c} and \textit{d}.
      \item[Hint:] Use that $v_j$ and $v_k$ are independent (private) to write \textit{i}'s expected payoff in eq.:
    \end{itemize}
    \vspace{-10pt}
    \begin{align*}
      \mathbb{E}[u_i(b_i,v_i)]
      &=\mathbb{P}(i\ wins|b_i)(v_i-b_i)\\
      &=\mathbb{P}\left(b_i>b_j(v_j),b_i>b_k(v_k)\right)(v_i-b_i)\\
      &=\mathbb{P}(b_i>cv_j+d,b_i>cv_k+d)(v_i-b_i),&&\text{using }(*)\\
      &=\mathbb{P}\left(\frac{b_i-d}{c}>v_j,\frac{b_i-d}{c}>v_k\right)(v_i-b_i)\\
      &=\mathbb{P}\left(\frac{b_i-d}{c}>v_j\right)\times\mathbb{P}\left(\frac{b_i-d}{c}>v_j\right)(v_i-b_i)\\
      &=\mathbb{P}\left(\frac{b_i-d}{c}>v_j\right)\times\mathbb{P}\left(\frac{b_i-d}{c}>v_j\right)(v_i-b_i)\\
      &=\left(\frac{b_i-d-c}{2c}\right)^2(v_i-b_i),&&\text{using ex. (3.a)}
    \end{align*}
    \vspace{-8pt}
    \begin{align*}
      FOC:\quad   0&=\frac{1}{2c}[2(b_i-d-c)(v_i-b_i)-(b_i-d-c)^2]\\
                  0&=2(v_i-b_i)-(b_i-d-c),&&\text{assuming }b_i-d-c\neq0\\
      b_i^{**}(v_i)&=\frac{2}{3}v_i+\frac{1}{3}(c+d)
    \end{align*}
    \vfill\null
\end{frame}
\begin{frame}{PS8, Ex. 4.a: First-price sealed bid auctions with three bidders}
    \begin{itemize}
      \item[(a)] For three bidders, show that there is a symmetric Bayesian Nash Equilibrium in linear strategies: $b_i(v_i) = cv_i + d\ (*)$. Find \textit{c} and \textit{d}.
      \item[Hint:] Use that $v_j$ and $v_k$ are independent (private) to write \textit{i}'s expected payoff in eq.:
    \end{itemize}
    \vspace{-10pt}
    \begin{align*}
      \mathbb{E}[u_i(b_i,v_i)]
      &=\mathbb{P}(i\ wins|b_i)(v_i-b_i)\\
      &=\mathbb{P}\left(b_i>b_j(v_j),b_i>b_k(v_k)\right)(v_i-b_i)\\
      &=\mathbb{P}(b_i>cv_j+d,b_i>cv_k+d)(v_i-b_i),&&\text{using }(*)\\
      &=\mathbb{P}\left(\frac{b_i-d}{c}>v_j,\frac{b_i-d}{c}>v_k\right)(v_i-b_i)\\
      &=\mathbb{P}\left(\frac{b_i-d}{c}>v_j\right)\times\mathbb{P}\left(\frac{b_i-d}{c}>v_j\right)(v_i-b_i)\\
      &=\mathbb{P}\left(\frac{b_i-d}{c}>v_j\right)\times\mathbb{P}\left(\frac{b_i-d}{c}>v_j\right)(v_i-b_i)\\
      &=\left(\frac{b_i-d-c}{2c}\right)^2(v_i-b_i),&&\text{using ex. (3.a)}
    \end{align*}
    \vspace{-8pt}
    \begin{align*}
      FOC:\quad   0&=\frac{1}{2c}[2(b_i-d-c)(v_i-b_i)-(b_i-d-c)^2]\\
                  0&=2(v_i-b_i)-(b_i-d-c),&&\text{assuming }b_i-d-c\neq0\\
      b_i^{**}(v_i)&=\underbrace{\frac{2}{3}}_{c^{*}=\frac{2}{3}}v_i+\underbrace{\frac{1}{3}(c+d)}_{d^{*}=\frac{1}{3}\left(\frac{2}{3}+d^{*}\right)\Rightarrow d^{*}=\frac{1}{3}}=\underline{\underline{\frac{2}{3}v_i+\frac{1}{3}}}&&Q.E.D.
    \end{align*}
    \vfill\null
\end{frame}


\begin{frame}{PS8, Ex. 4.b: First-price sealed bid auctions with three bidders}
    \begin{itemize}
      \item[(b)] Do you expect seller to earn a higher or a lower revenue than in the previous auction? What is causing this effect?
    \end{itemize}
    \vspace{-8pt}
    \begin{multicols}{2}
    \vfill\null\columnbreak
    BNE found:
    \begin{itemize}
      \item[(3.a)] $b_i^{*}(v_i)=\frac{1}{2}v_i+\frac{1}{2}$ for $i\in1,2,3$
      \item[(4.a)] $b_i^{**}(v_i)=\frac{2}{3}v_i+\frac{1}{3}$ for $i\in1,2,3$
    \end{itemize}
    \vfill\null
    \end{multicols}
\end{frame}
\begin{frame}{PS8, Ex. 4.b: First-price sealed bid auctions with three bidders}
    \begin{itemize}
      \item[(b)] Do you expect seller to earn a higher or a lower revenue than in the previous auction? What is causing this effect?
    \end{itemize}
    \vspace{-8pt}
    \begin{multicols}{2}
    Intuitively, \intuition{more bidders decreases the chance of winning, which should lead to less bid shading $\left(\frac{2}{3}>\frac{1}{2}\right)$ and therefore a \textit{higher} revenue for the seller.}
    \vfill\null\columnbreak
    BNE found:
    \begin{itemize}
      \item[(3.a)] $b_i^{*}(v_i)=\frac{1}{2}v_i+\frac{1}{2}$ for $i\in1,2,3$
      \item[(4.a)] $b_i^{**}(v_i)=\frac{2}{3}v_i+\frac{1}{3}$ for $i\in1,2,3$
    \end{itemize}
    \vfill\null
    \end{multicols}
\end{frame}
\begin{frame}{PS8, Ex. 4.b: First-price sealed bid auctions with three bidders}
    \begin{itemize}
      \item[(b)] Do you expect seller to earn a higher or a lower revenue than in the previous auction? What is causing this effect?
    \end{itemize}
    \vspace{-8pt}
    \begin{multicols}{2}
    Intuitively, \intuition{more bidders decreases the chance of winning, which should lead to less bid shading $\left(\frac{2}{3}>\frac{1}{2}\right)$ and therefore a \textit{higher} revenue for the seller.}\\\medskip
    Analytically, \intuition{we can confirm this:}
    \begin{align*}
      b_i^{**}&>b_i^{*}\Leftrightarrow\\
      \frac{2}{3}v_i+\frac{1}{3}&>\frac{1}{2}v_i+\frac{1}{2}\Leftrightarrow\\
      \frac{1}{6}v_i&>\frac{1}{6}\Leftrightarrow\\
                 v_i&>1
    \end{align*}
    \intuition{I.e. except for the rare case where all players have the valuation $v=1$, the seller's revenue  is strictly higher with three players than with two players.}
    \vfill\null\columnbreak
    BNE found:
    \begin{itemize}
      \item[(3.a)] $b_i^{*}(v_i)=\frac{1}{2}v_i+\frac{1}{2}$ for $i\in1,2,3$
      \item[(4.a)] $b_i^{**}(v_i)=\frac{2}{3}v_i+\frac{1}{3}$ for $i\in1,2,3$
    \end{itemize}
    \vfill\null
    \end{multicols}
\end{frame}


\begin{frame}{PS8, Ex. 4.c: First-price sealed bid auctions with three bidders}
    \begin{itemize}
      \item[(c)] (More difficult). Calculate the revenue to the seller.
    \end{itemize}
    \vspace{-8pt}
    \begin{multicols}{2}
      \vfill\null\columnbreak
      Results so far:
      \vspace{-6pt}
      \begin{itemize}
        \item[($*$)] $b_i(v_i) = cv_i+d$
        \item[($\dagger$)] $\mathbb{P}(i\ wins|v_i)=\left(\frac{b_i-d-c}{2c}\right)^2$ from (4.a)
        \item[(3.a)] $c^*=d^*=\frac{1}{2}$
        \item[(4.a)] $c^*=\frac{2}{3},\ d^*=\frac{1}{2}$
      \end{itemize}
      \vfill\null
    \end{multicols}
    \vfill\null
\end{frame}
\begin{frame}{PS8, Ex. 4.c: First-price sealed bid auctions with three bidders}
    \begin{itemize}
      \item[(c)] (More difficult). Calculate the revenue to the seller.
    \end{itemize}
    \vspace{-8pt}
    \begin{multicols}{2}
      \begin{itemize}
        \item[\nth{1} step:] \textbf{Calculate the expected payment of bidder \textit{i} with valuation $v_i$.}
      \end{itemize}
      \vfill\null\columnbreak
      Results so far:
      \vspace{-6pt}
      \begin{itemize}
        \item[($*$)] $b_i(v_i) = cv_i+d$
        \item[($\dagger$)] $\mathbb{P}(i\ wins|v_i)=\left(\frac{b_i-d-c}{2c}\right)^2$ from (4.a)
        \item[(3.a)] $c^*=d^*=\frac{1}{2}$
        \item[(4.a)] $c^*=\frac{2}{3},\ d^*=\frac{1}{2}$
      \end{itemize}
      \vfill\null
    \end{multicols}
    \vfill\null
\end{frame}
\begin{frame}{PS8, Ex. 4.c: First-price sealed bid auctions with three bidders}
    \begin{itemize}
      \item[(c)] (More difficult). Calculate the revenue to the seller.
    \end{itemize}
    \vspace{-8pt}
    \begin{multicols}{2}
      \begin{itemize}
        \item[\nth{1} step:] \textbf{Calculate the expected payment of bidder \textit{i} with valuation $v_i$:}
      \end{itemize}
      \vspace{-14pt}
      \begin{align*}
        m_i(v_i)&=\mathbb{P}(i\ wins|v_i)b_i(v_i)
      \end{align*}
      \vfill\null\columnbreak
      Results so far:
      \vspace{-6pt}
      \begin{itemize}
        \item[($*$)] $b_i(v_i) = cv_i+d$
        \item[($\dagger$)] $\mathbb{P}(i\ wins|v_i)=\left(\frac{b_i-d-c}{2c}\right)^2$ from (4.a)
        \item[(3.a)] $c^*=d^*=\frac{1}{2}$
        \item[(4.a)] $c^*=\frac{2}{3},\ d^*=\frac{1}{2}$
      \end{itemize}
      \vfill\null
    \end{multicols}
    \vfill\null
\end{frame}
\begin{frame}{PS8, Ex. 4.c: First-price sealed bid auctions with three bidders}
    \begin{itemize}
      \item[(c)] (More difficult). Calculate the revenue to the seller.
    \end{itemize}
    \vspace{-8pt}
    \begin{multicols}{2}
      \begin{itemize}
        \item[\nth{1} step:] Calculate the expected payment of bidder \textit{i} with valuation $v_i$:
      \end{itemize}
      \vspace{-4pt}
      \begin{align*}
        m_i(v_i)&=\mathbb{P}(i\ wins|v_i)b_i(v_i)\\
                &=\left(\frac{b_i-d-c}{2c}\right)^2b_i(v_i),\text{ using }(\dagger)\\
                &=\left(\frac{cv_i-c}{2c}\right)^2(cv_i+d),\text{ using }(*)\\
                &=\left(\frac{v_i-1}{2}\right)^2\left(\frac{2}{3}v_i+\frac{1}{3}\right),\text{ using }(4.a)\\
                &=\left(\frac{2v_i^3-3v_i^2+1}{12}\right)
      \end{align*}
      \vfill\null\columnbreak
      Results so far:
      \vspace{-6pt}
      \begin{itemize}
        \item[($*$)] $b_i(v_i) = cv_i+d$
        \item[($\dagger$)] $\mathbb{P}(i\ wins|v_i)=\left(\frac{b_i-d-c}{2c}\right)^2$ from (4.a)
        \item[(3.a)] $c^*=d^*=\frac{1}{2}$
        \item[(4.a)] $c^*=\frac{2}{3},\ d^*=\frac{1}{2}$
      \end{itemize}
      \vfill\null
    \end{multicols}
    \vfill\null
\end{frame}
\begin{frame}{PS8, Ex. 4.c: First-price sealed bid auctions with three bidders}
    \begin{itemize}
      \item[(c)] (More difficult). Calculate the revenue to the seller.
    \end{itemize}
    \vspace{-8pt}
    \begin{multicols}{2}
      \begin{itemize}
        \item[\nth{1} step:] Calculate the expected payment of bidder \textit{i} with valuation $v_i$:
      \end{itemize}
      \vspace{-10pt}
      \begin{align*}
        m_i(v_i)&=\mathbb{P}(i\ wins|v_i)b_i(v_i)\\
                &=\left(\frac{b_i-d-c}{2c}\right)^2b_i(v_i),\text{ using }(\dagger)\\
                &=\left(\frac{cv_i-c}{2c}\right)^2(cv_i+d),\text{ using }(*)\\
                &=\left(\frac{v_i-1}{2}\right)^2\left(\frac{2}{3}v_i+\frac{1}{3}\right),\text{ using }(4.a)\\
                &=\left(\frac{2v_i^3-3v_i^2+1}{12}\right)
      \end{align*}
      \vspace{-10pt}
      \begin{itemize}
        \item[\nth{2} step:] \textbf{Find the ex-ante expected payment by integrating $m_i(v_i)$ using the PDF.}
      \end{itemize}
      \vfill\null\columnbreak
      Results so far:
      \vspace{-6pt}
      \begin{itemize}
        \item[($*$)] $b_i(v_i) = cv_i+d$
        \item[($\dagger$)] $\mathbb{P}(i\ wins|v_i)=\left(\frac{b_i-d-c}{2c}\right)^2$ from (4.a)
        \item[(3.a)] $c^*=d^*=\frac{1}{2}$
        \item[(4.a)] $c^*=\frac{2}{3},\ d^*=\frac{1}{2}$
      \end{itemize}
      \vfill\null
    \end{multicols}
    \vfill\null
\end{frame}
\begin{frame}{PS8, Ex. 4.c: First-price sealed bid auctions with three bidders}
    \begin{itemize}
      \item[(c)] (More difficult). Calculate the revenue to the seller.
    \end{itemize}
    \vspace{-8pt}
    \begin{multicols}{2}
      \begin{itemize}
        \item[\nth{1} step:] Calculate the expected payment of bidder \textit{i} with valuation $v_i$:
      \end{itemize}
      \vspace{-10pt}
      \begin{align*}
        m_i(v_i)&=\mathbb{P}(i\ wins|v_i)b_i(v_i)\\
                &=\left(\frac{b_i-d-c}{2c}\right)^2b_i(v_i),\text{ using }(\dagger)\\
                &=\left(\frac{cv_i-c}{2c}\right)^2(cv_i+d),\text{ using }(*)\\
                &=\left(\frac{v_i-1}{2}\right)^2\left(\frac{2}{3}v_i+\frac{1}{3}\right),\text{ using }(4.a)\\
                &=\left(\frac{2v_i^3-3v_i^2+1}{12}\right)
      \end{align*}
      \vspace{-10pt}
      \begin{itemize}
        \item[\nth{2} step:] \textbf{Find the ex-ante expected payment by integrating $m_i(v_i)$ using the PDF.}
      \end{itemize}
      \vfill\null\columnbreak
      Results so far:
      \vspace{-6pt}
      \begin{itemize}
        \item[($*$)] $b_i(v_i) = cv_i+d$
        \item[($\dagger$)] $\mathbb{P}(i\ wins|v_i)=\left(\frac{b_i-d-c}{2c}\right)^2$ from (4.a)
        \item[(3.a)] $c^*=d^*=\frac{1}{2}$
        \item[(4.a)] $c^*=\frac{2}{3},\ d^*=\frac{1}{2}$
        \item[\nth{2}:] Ex-ante payment of bidder \textit{i}:
      \end{itemize}
      \vspace{-12pt}
      \begin{align*}
        \mathbb{E}[m_i(v_i)]&=\textstyle\int_1^3m_i(v_i)f_i(v_i)dv_i
      \end{align*}
      \vfill\null
    \end{multicols}
    \vfill\null
\end{frame}
\begin{frame}{PS8, Ex. 4.c: First-price sealed bid auctions with three bidders}
    \begin{itemize}
      \item[(c)] (More difficult). Calculate the revenue to the seller.
    \end{itemize}
    \vspace{-8pt}
    \begin{multicols}{2}
      \begin{itemize}
        \item[\nth{1} step:] Calculate the expected payment of bidder \textit{i} with valuation $v_i$:
      \end{itemize}
      \vspace{-10pt}
      \begin{align*}
        m_i(v_i)&=\mathbb{P}(i\ wins|v_i)b_i(v_i)\\
                &=\left(\frac{b_i-d-c}{2c}\right)^2b_i(v_i),\text{ using }(\dagger)\\
                &=\left(\frac{cv_i-c}{2c}\right)^2(cv_i+d),\text{ using }(*)\\
                &=\left(\frac{v_i-1}{2}\right)^2\left(\frac{2}{3}v_i+\frac{1}{3}\right),\text{ using }(4.a)\\
                &=\left(\frac{2v_i^3-3v_i^2+1}{12}\right)
      \end{align*}
      \vspace{-10pt}
      \begin{itemize}
        \item[\nth{2} step:] Find the ex-ante expected payment by integrating $m_i(v_i)$ using the PDF.
      \end{itemize}
      \vspace{-4pt}
      \textbf{Why is the ex-ante expected payment lower than in exercise 3.b?}
      \vfill\null\columnbreak
      Results so far:
      \vspace{-6pt}
      \begin{itemize}
        \item[($*$)] $b_i(v_i) = cv_i+d$
        \item[($\dagger$)] $\mathbb{P}(i\ wins|v_i)=\left(\frac{b_i-d-c}{2c}\right)^2$ from (4.a)
        \item[(3.a)] $c^*=d^*=\frac{1}{2}$
        \item[(4.a)] $c^*=\frac{2}{3},\ d^*=\frac{1}{2}$
        \item[\nth{2}:] Ex-ante payment of bidder \textit{i}:
      \end{itemize}
      \vspace{-12pt}
      \begin{align*}
        \mathbb{E}[m_i(v_i)]&=\textstyle\int_1^3m_i(v_i)f_i(v_i)dv_i\\
                            &=\textstyle\int_1^3\left(\frac{2v_i^3-3v_i^2+1}{12}\right)\cdot\frac{1}{3-1}dv_i\\
                            &=\frac{1}{24}\left[\frac{2}{4}v_i^4-\frac{3}{3}v_i^3+v_i\right]_1^3\\
                            &=\frac{1}{24}\left(\frac{33}{2}-\frac{1}{2}\right)=\frac{2}{3}<\frac{5}{6}
      \end{align*}
      \vfill\null
    \end{multicols}
    \vfill\null
\end{frame}
\begin{frame}{PS8, Ex. 4.c: First-price sealed bid auctions with three bidders}
    \begin{itemize}
      \item[(c)] (More difficult). Calculate the revenue to the seller.
    \end{itemize}
    \vspace{-8pt}
    \begin{multicols}{2}
      \begin{itemize}
        \item[\nth{1} step:] Calculate the expected payment of bidder \textit{i} with valuation $v_i$:
      \end{itemize}
      \vspace{-10pt}
      \begin{align*}
        m_i(v_i)&=\mathbb{P}(i\ wins|v_i)b_i(v_i)\\
                &=\left(\frac{b_i-d-c}{2c}\right)^2b_i(v_i),\text{ using }(\dagger)\\
                &=\left(\frac{cv_i-c}{2c}\right)^2(cv_i+d),\text{ using }(*)\\
                &=\left(\frac{v_i-1}{2}\right)^2\left(\frac{2}{3}v_i+\frac{1}{3}\right),\text{ using }(4.a)\\
                &=\left(\frac{2v_i^3-3v_i^2+1}{12}\right)
      \end{align*}
      \vspace{-10pt}
      \begin{itemize}
        \item[\nth{2} step:] Find the ex-ante expected payment by integrating $m_i(v_i)$ using the PDF.
      \end{itemize}
      \vspace{-4pt}
      \intuition{Though the bids are higher, the expected payment from each bidder is lower due to a lower probability of winning.}
      \vfill\null\columnbreak
      Results so far:
      \vspace{-6pt}
      \begin{itemize}
        \item[($*$)] $b_i(v_i) = cv_i+d$
        \item[($\dagger$)] $\mathbb{P}(i\ wins|v_i)=\left(\frac{b_i-d-c}{2c}\right)^2$ from (4.a)
        \item[(3.a)] $c^*=d^*=\frac{1}{2}$
        \item[(4.a)] $c^*=\frac{2}{3},\ d^*=\frac{1}{2}$
        \item[\nth{2}:] Ex-ante payment of bidder \textit{i}:
      \end{itemize}
      \vspace{-12pt}
      \begin{align*}
        \mathbb{E}[m_i(v_i)]&=\textstyle\int_1^3m_i(v_i)f_i(v_i)dv_i\\
                            &=\textstyle\int_1^3\left(\frac{2v_i^3-3v_i^2+1}{12}\right)\cdot\frac{1}{3-1}dv_i\\
                            &=\frac{1}{24}\left[\frac{2}{4}v_i^4-\frac{3}{3}v_i^3+v_i\right]_1^3\\
                            &=\frac{1}{24}\left(\frac{33}{2}-\frac{1}{2}\right)=\frac{2}{3}<\frac{5}{6}
      \end{align*}
      \vfill\null
    \end{multicols}
    \vfill\null
\end{frame}
\begin{frame}{PS8, Ex. 4.c: First-price sealed bid auctions with three bidders}
    \begin{itemize}
      \item[(c)] (More difficult). Calculate the revenue to the seller.
    \end{itemize}
    \vspace{-8pt}
    \begin{multicols}{2}
      \begin{itemize}
        \item[\nth{1} step:] Calculate the expected payment of bidder \textit{i} with valuation $v_i$:
      \end{itemize}
      \vspace{-10pt}
      \begin{align*}
        m_i(v_i)&=\mathbb{P}(i\ wins|v_i)b_i(v_i)\\
                &=\left(\frac{b_i-d-c}{2c}\right)^2b_i(v_i),\text{ using }(\dagger)\\
                &=\left(\frac{cv_i-c}{2c}\right)^2(cv_i+d),\text{ using }(*)\\
                &=\left(\frac{v_i-1}{2}\right)^2\left(\frac{2}{3}v_i+\frac{1}{3}\right),\text{ using }(4.a)\\
                &=\left(\frac{2v_i^3-3v_i^2+1}{12}\right)
      \end{align*}
      \vspace{-10pt}
      \begin{itemize}
        \item[\nth{2} step:] Find the ex-ante expected payment by integrating $m_i(v_i)$ using the PDF.
      \end{itemize}
      \vspace{-4pt}
      \intuition{Though the bids are higher, the expected payment from each bidder is lower due to a lower probability of winning.}
      \vspace{-4pt}
      \begin{itemize}
        \item[\nth{3} step:] \textbf{Calculate the seller's revenue and compare to exercise (3.b).}
      \end{itemize}
      \vfill\null\columnbreak
      Results so far:
      \vspace{-6pt}
      \begin{itemize}
        \item[($*$)] $b_i(v_i) = cv_i+d$
        \item[($\dagger$)] $\mathbb{P}(i\ wins|v_i)=\left(\frac{b_i-d-c}{2c}\right)^2$ from (4.a)
        \item[(3.a)] $c^*=d^*=\frac{1}{2}$
        \item[(4.a)] $c^*=\frac{2}{3},\ d^*=\frac{1}{2}$
        \item[\nth{2}:] Ex-ante payment of bidder \textit{i}:
      \end{itemize}
      \vspace{-12pt}
      \begin{align*}
        \mathbb{E}[m_i(v_i)]&=\textstyle\int_1^3m_i(v_i)f_i(v_i)dv_i\\
                            &=\textstyle\int_1^3\left(\frac{2v_i^3-3v_i^2+1}{12}\right)\cdot\frac{1}{3-1}dv_i\\
                            &=\frac{1}{24}\left[\frac{2}{4}v_i^4-\frac{3}{3}v_i^3+v_i\right]_1^3\\
                            &=\frac{1}{24}\left(\frac{33}{2}-\frac{1}{2}\right)=\frac{2}{3}<\frac{5}{6}
      \end{align*}
      \vfill\null
    \end{multicols}
    \vfill\null
\end{frame}
\begin{frame}{PS8, Ex. 4.c: First-price sealed bid auctions with three bidders}
    \begin{itemize}
      \item[(c)] (More difficult). Calculate the revenue to the seller.
    \end{itemize}
    \vspace{-8pt}
    \begin{multicols}{2}
      \begin{itemize}
        \item[\nth{1} step:] Calculate the expected payment of bidder \textit{i} with valuation $v_i$:
      \end{itemize}
      \vspace{-14pt}
      \begin{align*}
        m_i(v_i)&=\mathbb{P}(i\ wins|v_i)b_i(v_i)\\
                &=\left(\frac{b_i-d-c}{2c}\right)^2b_i(v_i),\text{ using }(\dagger)\\
                &=\left(\frac{cv_i-c}{2c}\right)^2(cv_i+d),\text{ using }(*)\\
                &=\left(\frac{v_i-1}{2}\right)^2\left(\frac{2}{3}v_i+\frac{1}{3}\right),\text{ using }(4.a)\\
                &=\left(\frac{2v_i^3-3v_i^2+1}{12}\right)
      \end{align*}
      \vspace{-14pt}
      \begin{itemize}
        \item[\nth{2} step:] Find the ex-ante expected payment by integrating $m_i(v_i)$ using the PDF.
      \end{itemize}
      \vspace{-4pt}
      \intuition{Though the bids are higher, the expected payment from each bidder is lower due to a lower probability of winning.}
      \vspace{-4pt}
      \begin{itemize}
        \item[\nth{3} step:] Calculate the seller's revenue and compare to exercise (3.b).
      \end{itemize}
      \vfill\null\columnbreak
      Results so far:
      \vspace{-6pt}
      \begin{itemize}
        \item[($*$)] $b_i(v_i) = cv_i+d$
        \item[($\dagger$)] $\mathbb{P}(i\ wins|v_i)=\left(\frac{b_i-d-c}{2c}\right)^2$ from (4.a)
        \item[(3.a)] $c^*=d^*=\frac{1}{2}$
        \item[(4.a)] $c^*=\frac{2}{3},\ d^*=\frac{1}{2}$
        \item[\nth{2}:] Ex-ante payment of bidder \textit{i}:
      \end{itemize}
      \vspace{-12pt}
      \begin{align*}
        \mathbb{E}[m_i(v_i)]&=\textstyle\int_1^3m_i(v_i)f_i(v_i)dv_i\\
                            &=\textstyle\int_1^3\left(\frac{2v_i^3-3v_i^2+1}{12}\right)\cdot\frac{1}{3-1}dv_i\\
                            &=\frac{1}{24}\left[\frac{2}{4}v_i^4-\frac{3}{3}v_i^3+v_i\right]_1^3\\
                            &=\frac{1}{24}\left(\frac{33}{2}-\frac{1}{2}\right)=\frac{2}{3}<\frac{5}{6}
      \end{align*}
      \vspace{-16pt}
      \begin{itemize}
        \item[\nth{3}:] $Revenue=3\cdot\mathbb{E}[m_i(v_i)]=2>\frac{5}{3}$
      \end{itemize}
      \vspace{-6pt}
      \textbf{Why is seller’s revenue higher than in exercise 3.b?}
      \vfill\null
    \end{multicols}
    \vfill\null
\end{frame}
\begin{frame}{PS8, Ex. 4.c: First-price sealed bid auctions with three bidders}
    \begin{itemize}
      \item[(c)] (More difficult). Calculate the revenue to the seller.
    \end{itemize}
    \vspace{-8pt}
    \begin{multicols}{2}
      \begin{itemize}
        \item[\nth{1} step:] Calculate the expected payment of bidder \textit{i} with valuation $v_i$:
      \end{itemize}
      \vspace{-14pt}
      \begin{align*}
        m_i(v_i)&=\mathbb{P}(i\ wins|v_i)b_i(v_i)\\
                &=\left(\frac{b_i-d-c}{2c}\right)^2b_i(v_i),\text{ using }(\dagger)\\
                &=\left(\frac{cv_i-c}{2c}\right)^2(cv_i+d),\text{ using }(*)\\
                &=\left(\frac{v_i-1}{2}\right)^2\left(\frac{2}{3}v_i+\frac{1}{3}\right),\text{ using }(4.a)\\
                &=\left(\frac{2v_i^3-3v_i^2+1}{12}\right)
      \end{align*}
      \vspace{-14pt}
      \begin{itemize}
        \item[\nth{2} step:] Find the ex-ante expected payment by integrating $m_i(v_i)$ using the PDF.
      \end{itemize}
      \vspace{-4pt}
      \intuition{Though the bids are higher, the expected payment from each bidder is lower due to a lower probability of winning.}
      \vspace{-4pt}
      \begin{itemize}
        \item[\nth{3} step:] Calculate the seller's revenue and compare to exercise (3.b).
      \end{itemize}
      \vfill\null\columnbreak
      Results so far:
      \vspace{-6pt}
      \begin{itemize}
        \item[($*$)] $b_i(v_i) = cv_i+d$
        \item[($\dagger$)] $\mathbb{P}(i\ wins|v_i)=\left(\frac{b_i-d-c}{2c}\right)^2$ from (4.a)
        \item[(3.a)] $c^*=d^*=\frac{1}{2}$
        \item[(4.a)] $c^*=\frac{2}{3},\ d^*=\frac{1}{2}$
        \item[\nth{2}:] Ex-ante payment of bidder \textit{i}:
      \end{itemize}
      \vspace{-12pt}
      \begin{align*}
        \mathbb{E}[m_i(v_i)]&=\textstyle\int_1^3m_i(v_i)f_i(v_i)dv_i\\
                            &=\textstyle\int_1^3\left(\frac{2v_i^3-3v_i^2+1}{12}\right)\cdot\frac{1}{3-1}dv_i\\
                            &=\frac{1}{24}\left[\frac{2}{4}v_i^4-\frac{3}{3}v_i^3+v_i\right]_1^3\\
                            &=\frac{1}{24}\left(\frac{33}{2}-\frac{1}{2}\right)=\frac{2}{3}<\frac{5}{6}
      \end{align*}
      \vspace{-16pt}
      \begin{itemize}
        \item[\nth{3}:] $Revenue=3\cdot\mathbb{E}[m_i(v_i)]=2>\frac{5}{3}$
      \end{itemize}
      \vspace{-6pt}
      \intuition{The seller can expect higher revenue as more players increases competition and the chance of one having high valuation.}
      \vfill\null
    \end{multicols}
    \vfill\null
\end{frame}
