\section{PS8, Ex. 4: A simple principal-agent model of corruption}

\begin{frame}{PS8, Ex. 4: A simple principal-agent model of corruption}
    Suppose two lobbyists, $i = 1, 2$, are trying to persuade a policymaker to implement their preferred policy by making a costly effort $e_i\in[0, 1]$. The policymaker can only implement one of the policies, and will implement the policy of the lobbyist who makes the most effort (you can also think of the policymaker as being corrupt, and the effort being a bribe.) The point is, that the lobbyist has to make the effort \textit{before} he learns if his policy is implemented.\\\medskip
    The value to $i$ of having his preferred policy implemented is $v_i$, where $v_i\sim U(0, 1)$ independently (private values). The lobbyists know their own valuation, but not that of the other lobbyist.
    \begin{itemize}
      \item[(a)] Rewrite this as an auction. What is the difference to the auctions we have seen so far?
      \item[(b)] Check that there is a symmetric Bayesian Nash Equilibrium of the type $b_i(v_i) = cv_i^2$, and find \textit{c}.
    \end{itemize}
\end{frame}

\begin{frame}{PS8, Ex. 4.a: A simple principal-agent model of corruption}
    Suppose two lobbyists, $i = 1, 2$, are trying to persuade a policymaker to implement their preferred policy by making a costly effort $e_i\in[0, 1]$. The policymaker can only implement one of the policies, and will implement the policy of the lobbyist who makes the most effort (you can also think of the policymaker as being corrupt, and the effort being a bribe.) The point is, that the lobbyist has to make the effort \textit{before} he learns if his policy is implemented.\\\medskip
    The value to $i$ of having his preferred policy implemented is $v_i$, where $v_i\sim U(0, 1)$ independently (private values). The lobbyists know their own valuation, but not that of the other lobbyist.
    \begin{itemize}
      \item[(a)] Rewrite this as an auction. What is the difference to the auctions we have seen so far?
    \end{itemize}
    \vspace{-8pt}
    \begin{multicols}{2}
      \vfill\null\columnbreak
      \vfill\null
    \end{multicols}
\end{frame}

\begin{frame}{PS8, Ex. 4.b: A simple principal-agent model of corruption}
    Suppose two lobbyists, $i = 1, 2$, are trying to persuade a policymaker to implement their preferred policy by making a costly effort $e_i\in[0, 1]$. The policymaker can only implement one of the policies, and will implement the policy of the lobbyist who makes the most effort (you can also think of the policymaker as being corrupt, and the effort being a bribe.) The point is, that the lobbyist has to make the effort \textit{before} he learns if his policy is implemented.\\\medskip
    The value to $i$ of having his preferred policy implemented is $v_i$, where $v_i\sim U(0, 1)$ independently (private values). The lobbyists know their own valuation, but not that of the other lobbyist.
    \begin{itemize}
      \item[(b)] Check that there is a symmetric Bayesian Nash Equilibrium of the type $b_i(v_i) = cv_i^2$, and find \textit{c}.
    \end{itemize}
    \vspace{-8pt}
    \begin{multicols}{2}
      \vfill\null\columnbreak
      \vfill\null
    \end{multicols}
\end{frame}




\section{PS8, Ex. 5: Extensive form games (Perfect Bayesian Equilibria)}

\begin{frame}{PS8, Ex. 5.a: Extensive form games (Perfect Bayesian Equilibria)}
    \begin{multicols}{2}
      \vfill\null\columnbreak
      \vfill\null
    \end{multicols}
\end{frame}

\begin{frame}{PS8, Ex. 5.b: Extensive form games (Perfect Bayesian Equilibria)}
    \begin{multicols}{2}
      \vfill\null\columnbreak
      \vfill\null
    \end{multicols}
\end{frame}




\section{PS8, Ex. 6: Extensive form game (Mixed-strategy Perfect Bayesian Equilibria)}

\begin{frame}{PS8, Ex. 6: Mixed-strategy Perfect Bayesian Equilibria (extensive form game)}
    \begin{multicols}{2}
      \vfill\null\columnbreak
      \vfill\null
    \end{multicols}
\end{frame}




\section{PS8, Ex. 7: Dissolving a partnership (Perfect Bayesian Equilibria)}

\begin{frame}{PS8, Ex. 7: Dissolving a partnership (Perfect Bayesian Equilibria)}
    Difficult. Exercise 4.10 in Gibbons (p. 250). Two partners must dissolve their partnership. Partner 1 currently owns share $s$ of the partnership, partner 2 owns share $1-s$. The partners agree to play the following game: partner 1 names a price, $p$, for the whole partnership, and partner 2 then chooses either to buy l's share for $ps$ or to sell his or her share to 1 for $p(1-s)$. Suppose it is common knowledge that the partners' valuations for owning the whole partnership are independently and uniformly distributed on $[0,1]$, but that each partner's valuation is private information. What is the perfect Bayesian equilibrium?
    \begin{multicols}{2}
      \vfill\null\columnbreak
      \vfill\null
    \end{multicols}
\end{frame}
