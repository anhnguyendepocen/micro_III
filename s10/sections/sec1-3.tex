\section{PS8, Ex. 1 (A): Asymmetric values (second-price sealed bid auction)}

\begin{frame}{PS8, Ex. 1 (A): Asymmetric values (second-price sealed bid auction)}
    Suppose there are two bidders who have private but asymmetric values. In particular, $v_1\sim U(0, 1)$ and $v_2\sim U(0, 2)$. Suppose the auction format is second-price sealed bid. When the values are private and symmetric, it is a weakly dominant strategy to bid one’s value. Is this still true when the values are asymmetric?
\end{frame}

\begin{frame}{PS8, Ex. 1 (A): Asymmetric values (second-price sealed bid auction)}
    Suppose there are two bidders who have private but asymmetric values. In particular, $v_1\sim U(0, 1)$ and $v_2\sim U(0, 2)$. Suppose the auction format is second-price sealed bid. When the values are private and symmetric, it is a weakly dominant strategy to bid one’s value. Is this still true when the values are asymmetric?
    \begin{multicols}{2}
      \vfill\null\columnbreak
      \vfill\null
    \end{multicols}
\end{frame}



\section{PS8, Ex. 2 (A): Crimea Through a Game-Theory }

\begin{frame}{PS8, Ex. 2 (A): }
    Read through the New York Times article \href{https://www.nytimes.com/2014/03/16/business/crimea-through-a-game-theory-lens.html}{Crime Through a Game-Theory Lens} by Tyler Cowen (co-author of the popular economics blog Marginal Revolution). Try to think about how you would set up models to describe the situations he writes about. \textit{(This exercise is just for reflection, no answer will be provided).}
\end{frame}



\section{PS8, Ex. 3 (A): The 'Lemons' model (asymmetric information)}

\begin{frame}{PS8, Ex. 3 (A): The 'Lemons' model (asymmetric information)}
    Consider the The 'Lemons' model of Akerlof. Suppose that used cars come in two types: high-quality “beauties” and low-quality “lemons”. Lemon-owners are willing to sell for \$800 but Beauty-owners will not sell for anything less than \$2000. Buyers will pay up to \$1200 for a lemon and up to \$2400 for a beauty.
    \begin{itemize}
      \item[(a)] Describe what would happen in the used-car market if buyers can distinguish between beauties and lemons.
      \item[(b)] What would happen if buyers cannot do so, and know that half of all used cars are lemons? Draw this as a dynamic game of incomplete information, where nature chooses the type of the car, the seller observes this and sets a price (any positive real number) and the buyer decides whether to buy or not.
      \item[(c)] Find a Perfect Bayesian Equilibrium of this model.
    \end{itemize}
\end{frame}

\begin{frame}{PS8, Ex. 3.a (A): The 'Lemons' model (asymmetric information)}
    Consider the The 'Lemons' model of Akerlof. Suppose that used cars come in two types: high-quality “beauties” and low-quality “lemons”. Lemon-owners are willing to sell for \$800 but Beauty-owners will not sell for anything less than \$2000. Buyers will pay up to \$1200 for a lemon and up to \$2400 for a beauty.
    \begin{itemize}
      \item[(a)] Describe what would happen in the used-car market if buyers can distinguish between beauties and lemons.
      \item[(b)] What would happen if buyers cannot do so, and know that half of all used cars are lemons? Draw this as a dynamic game of incomplete information, where nature chooses the type of the car, the seller observes this and sets a price (any positive real number) and the buyer decides whether to buy or not.
      \item[(c)] Find a Perfect Bayesian Equilibrium of this model.
    \end{itemize}
    \vspace{-8pt}
    \begin{multicols}{2}
      \vfill\null\columnbreak
      \vfill\null
    \end{multicols}
\end{frame}

\begin{frame}{PS8, Ex. 3.b (A): The 'Lemons' model (asymmetric information)}
    Consider the The 'Lemons' model of Akerlof. Suppose that used cars come in two types: high-quality “beauties” and low-quality “lemons”. Lemon-owners are willing to sell for \$800 but Beauty-owners will not sell for anything less than \$2000. Buyers will pay up to \$1200 for a lemon and up to \$2400 for a beauty.
    \begin{itemize}
      \item[(b)] What would happen if buyers cannot do so, and know that half of all used cars are lemons? Draw this as a dynamic game of incomplete information, where nature chooses the type of the car, the seller observes this and sets a price (any positive real number) and the buyer decides whether to buy or not.
    \end{itemize}
    \vspace{-8pt}
    \begin{multicols}{2}
      \vfill\null\columnbreak
      \vfill\null
    \end{multicols}
\end{frame}

\begin{frame}{PS8, Ex. 3.c (A): The 'Lemons' model (asymmetric information)}
    \begin{itemize}
      \item[(c)] Find a Perfect Bayesian Equilibrium of this model.
    \end{itemize}
    \vspace{-8pt}
    \begin{multicols}{2}
      \vfill\null\columnbreak
      \vfill\null
    \end{multicols}
\end{frame}
