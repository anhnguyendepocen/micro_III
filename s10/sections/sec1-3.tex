\section{PS8, Ex. 1 (A): Asymmetric values (second-price sealed bid auction)}

\begin{frame}{PS8, Ex. 1 (A): Asymmetric values (second-price sealed bid auction)}
    Suppose there are two bidders who have private but asymmetric values. In particular, $v_1\sim U(0, 1)$ and $v_2\sim U(0, 2)$. Suppose the auction format is second-price sealed bid. When the values are private and symmetric, it is a weakly dominant strategy to bid one’s value. Is this still true when the values are asymmetric?
\end{frame}

\begin{frame}{PS8, Ex. 1 (A): Asymmetric values (second-price sealed bid auction)}
    Suppose there are two bidders who have private but asymmetric values. In particular, $v_1\sim U(0, 1)$ and $v_2\sim U(0, 2)$. Suppose the auction format is second-price sealed bid. When the values are private and symmetric, it is a weakly dominant strategy to bid one’s value. Is this still true when the values are asymmetric?
    \begin{multicols}{2}
      \begin{itemize}
        \item[Step 1:] Recall the argument in PS9, Ex. 3.c.
      \end{itemize}
      \vfill\null\columnbreak
      \vfill\null
    \end{multicols}
\end{frame}
\begin{frame}{PS8, Ex. 1 (A): Asymmetric values (second-price sealed bid auction)}
    Suppose there are two bidders who have private but asymmetric values. In particular, $v_1\sim U(0, 1)$ and $v_2\sim U(0, 2)$. Suppose the auction format is second-price sealed bid. When the values are private and symmetric, it is a weakly dominant strategy to bid one’s value. Is this still true when the values are asymmetric?
    \begin{multicols}{2}
      \begin{itemize}
        \item[Step 1:] Recall the argument in PS9, Ex. 3.c.
        \begin{itemize}\normalsize
          \item[i.]   \textbf{Suppose player 2 bids his valuation: $b_2(v_2) = v_2$. Write down the expected payoffs to player 1 from bidding $b_1$.}
        \end{itemize}
      \end{itemize}
      \vfill\null\columnbreak
      \vfill\null
    \end{multicols}
\end{frame}
\begin{frame}{PS8, Ex. 1 (A): Asymmetric values (second-price sealed bid auction)}
    Suppose there are two bidders who have private but asymmetric values. In particular, $v_1\sim U(0, 1)$ and $v_2\sim U(0, 2)$. Suppose the auction format is second-price sealed bid. When the values are private and symmetric, it is a weakly dominant strategy to bid one’s value. Is this still true when the values are asymmetric?
    \begin{multicols}{2}
      \begin{itemize}
        \item[Step 1:] Recall the argument in PS9, Ex. 3.c.
        \begin{itemize}\normalsize
          \item[i.]   Suppose player 2 bids his valuation: $b_2(v_2) = v_2$. Write down the expected payoffs to player 1 from bidding $b_1$.
          \item[ii.]  \textbf{Using your previous answer, argue that there is a symmetric Bayesian Nash Equilibrium (BNE) in which both players bid their valuation.}
        \end{itemize}
      \end{itemize}
      \vfill\null\columnbreak
      \begin{itemize}
        \item[(i)] The expected payoffs of P1 given $b_2$:
      \end{itemize}
      \vspace{-12pt}
      \begin{align*}
        u_1(b_1,b_2)=\left\{\begin{array}{lcl}
          v_1-b_2     & \text{if} & b_1>b_2 \\
          (v_1-b_2)/2 & \text{if} & b_1=b_2 \\
          0           & \text{if} & b_1<b_2
        \end{array}\right.
      \end{align*}
      \vfill\null
    \end{multicols}
\end{frame}
\begin{frame}{PS8, Ex. 1 (A): Asymmetric values (second-price sealed bid auction)}
    Suppose there are two bidders who have private but asymmetric values. In particular, $v_1\sim U(0, 1)$ and $v_2\sim U(0, 2)$. Suppose the auction format is second-price sealed bid. When the values are private and symmetric, it is a weakly dominant strategy to bid one’s value. Is this still true when the values are asymmetric?
    \begin{multicols}{2}
      \begin{itemize}
        \item[Step 1:] Recall the argument in PS9, Ex. 3.c.
        \begin{itemize}\normalsize
          \item[i.]   Suppose player 2 bids his valuation: $b_2(v_2) = v_2$. Write down the expected payoffs to player 1 from bidding $b_1$.
          \item[ii.]  Using your previous answer, argue that there is a symmetric Bayesian Nash Equilibrium (BNE) in which both players bid their valuation.
        \end{itemize}
        \item[Step 2:] \textbf{How is this result affected by the distribution of the bidder's values?}
      \end{itemize}
      \vfill\null\columnbreak
      \begin{itemize}
        \item[(i)] The expected payoffs of P1 given $b_2$:
      \end{itemize}
      \vspace{-12pt}
      \begin{align*}
        u_1(b_1,b_2)=\left\{\begin{array}{lcl}
          v_1-b_2     & \text{if} & b_1>b_2 \\
          (v_1-b_2)/2 & \text{if} & b_1=b_2 \\
          0           & \text{if} & b_1<b_2
        \end{array}\right.
      \end{align*}
      \vspace{-18pt}
      \begin{itemize}
        \item[(ii)] P1 wins: Payoff is independent of $b_1$ unless $b_1<b_2$, in which case P1 no longer wins, thus, gets zero payoff.
        \item[] P1 looses: Payoff is independent of $b_1$ unless $b_1>b_2$, in which case P1 wins instead but bids more than her evaluation and gets negative payoff.
        \item[] i.e. there is no incentive to deviate from $BNE=(b_1^*,b_2^*)=\{(v_1,v_2)\}$.
      \end{itemize}
      \vfill\null
    \end{multicols}
\end{frame}
\begin{frame}{PS8, Ex. 1 (A): Asymmetric values (second-price sealed bid auction)}
    Suppose there are two bidders who have private but asymmetric values. In particular, $v_1\sim U(0, 1)$ and $v_2\sim U(0, 2)$. Suppose the auction format is second-price sealed bid. When the values are private and symmetric, it is a weakly dominant strategy to bid one’s value. Is this still true when the values are asymmetric?
    \begin{multicols}{2}
      \begin{itemize}
        \item[Step 1:] Recall the argument in PS9, Ex. 3.c.
        \begin{itemize}\normalsize
          \item[i.]   Suppose player 2 bids his valuation: $b_2(v_2) = v_2$. Write down the expected payoffs to player 1 from bidding $b_1$.
          \item[ii.]  Using your previous answer, argue that there is a symmetric Bayesian Nash Equilibrium (BNE) in which both players bid their valuation.
        \end{itemize}
        \item[Step 2:] How is this result affected by the distribution of the bidder's values?
      \end{itemize}
      \vfill\null\columnbreak
      \begin{itemize}
        \item[(i)] The expected payoffs of P1 given $b_2$:
      \end{itemize}
      \vspace{-12pt}
      \begin{align*}
        u_1(b_1,b_2)=\left\{\begin{array}{lcl}
          v_1-b_2     & \text{if} & b_1>b_2 \\
          (v_1-b_2)/2 & \text{if} & b_1=b_2 \\
          0           & \text{if} & b_1<b_2
        \end{array}\right.
      \end{align*}
      \vspace{-18pt}
      \begin{itemize}
        \item[(ii)] P1 wins: Payoff is independent of $b_1$ unless $b_1<b_2$, in which case P1 no longer wins, thus, gets zero payoff.
        \item[] P1 looses: Payoff is independent of $b_1$ unless $b_1>b_2$, in which case P1 wins instead but bids more than her evaluation and gets negative payoff.
        \item[] i.e. there is no incentive to deviate from $BNE=(b_1^*,b_2^*)=\{(v_1,v_2)\}$.
        \item[2:] The result is independent of the distributions, thus it's still a best-response to bid one's value.
      \end{itemize}
      \vfill\null
    \end{multicols}
\end{frame}




\section{PS8, Ex. 2 (A): Crimea Through a Game-Theory }

\begin{frame}{PS8, Ex. 2 (A): }
    Read through the New York Times article \href{https://www.nytimes.com/2014/03/16/business/crimea-through-a-game-theory-lens.html}{Crimea Through a Game-Theory Lens} by Tyler Cowen (co-author of the popular economics blog Marginal Revolution). Try to think about how you would set up models to describe the situations he writes about. \textit{(This exercise is just for reflection, no answer will be provided).}
\end{frame}



\section{PS8, Ex. 3 (A): The 'Lemons' model (asymmetric information)}

\begin{frame}{PS8, Ex. 3 (A): The 'Lemons' model (asymmetric information)}
    Consider the The 'Lemons' model of Akerlof. Suppose that used cars come in two types: high-quality “beauties” and low-quality “lemons”. Lemon-owners are willing to sell for \$800 but Beauty-owners will not sell for anything less than \$2000. Buyers will pay up to \$1200 for a lemon and up to \$2400 for a beauty.
    \begin{itemize}
      \item[(a)] Describe what would happen in the used-car market if buyers can distinguish between beauties and lemons.
      \item[(b)] What would happen if buyers cannot do so, and know that half of all used cars are lemons? Draw this as a dynamic game of incomplete information, where nature chooses the type of the car, the seller observes this and sets a price (any positive real number) and the buyer decides whether to buy or not.
      \item[(c)] Find a Perfect Bayesian Equilibrium of this model.
    \end{itemize}
    \textit{"In US English, a lemon is a vehicle (often new) that turns out to have several manufacturing defects affecting its safety, value or utility."} (Source: \href{https://en.wikipedia.org/wiki/Lemon_(automobile)}{Wikipedia})
\end{frame}

\begin{frame}{PS8, Ex. 3.a (A): The 'Lemons' model (asymmetric information)}
    Consider the The 'Lemons' model of Akerlof. Suppose that used cars come in two types: high-quality “beauties” and low-quality “lemons”. Lemon-owners are willing to sell for \$800 but Beauty-owners will not sell for anything less than \$2000. Buyers will pay up to \$1200 for a lemon and up to \$2400 for a beauty.
    \begin{itemize}
      \item[(a)] Describe what would happen in the used-car market if buyers can distinguish between beauties and lemons.
    \end{itemize}
    If buyers can distinguish between beauties and lemons they would be traded on two separate markets with prices within $[800,1200]$ and $[2000,2400]$ respectively.
\end{frame}

\begin{frame}{PS8, Ex. 3.b (A): The 'Lemons' model (asymmetric information)}
    Consider the The 'Lemons' model of Akerlof. Suppose that used cars come in two types: high-quality “beauties” and low-quality “lemons”. Lemon-owners are willing to sell for \$800 but Beauty-owners will not sell for anything less than \$2000. Buyers will pay up to \$1200 for a lemon and up to \$2400 for a beauty.
    \vspace{-4pt}
    \begin{itemize}
      \item[(b)] What would happen if buyers cannot do so, and know that half of all used cars are lemons? Draw this as a dynamic game of incomplete information, where nature chooses the type of the car, the seller observes this and sets a price (any positive real number) and the buyer decides whether to buy or not.
    \end{itemize}
    \vspace{-8pt}
    \begin{figure}[!h]
      \center
      \def\svgwidth{.8\columnwidth}
      \import{figures/}{3b.pdf_tex}
    \end{figure}
\end{frame}


\begin{frame}{PS8, Ex. 3.c (A): The 'Lemons' model (asymmetric information)}
    \begin{figure}[!h]
      \center
      \def\svgwidth{.8\columnwidth}
      \import{figures/}{3b.pdf_tex}
    \end{figure}
    \vspace{-8pt}
    \begin{itemize}
      \item[(c)] Find a Perfect Bayesian Equilibrium of this model.
    \end{itemize}
    \vfill\null
\end{frame}
\begin{frame}{PS8, Ex. 3.c (A): The 'Lemons' model (asymmetric information)}
    \begin{figure}[!h]
      \center
      \def\svgwidth{.8\columnwidth}
      \import{figures/}{3b.pdf_tex}
    \end{figure}
    \vspace{-8pt}
    \begin{itemize}
      \item[(c)] Find a Perfect Bayesian Equilibrium of this model.
    \end{itemize}
    \vspace{-12pt}
    \begin{multicols}{2}
      \begin{itemize}
        \item[Step 1:] \textbf{Write up buyer's expectation to the car's value given her beliefs regarding \textit{p}}
      \end{itemize}
      \vfill\null\columnbreak
      \vfill\null
    \end{multicols}
\end{frame}
\begin{frame}{PS8, Ex. 3.c (A): The 'Lemons' model (asymmetric information)}
    \begin{figure}[!h]
      \center
      \def\svgwidth{.8\columnwidth}
      \import{figures/}{3b.pdf_tex}
    \end{figure}
    \vspace{-8pt}
    \begin{itemize}
      \item[(c)] Find a Perfect Bayesian Equilibrium of this model.
    \end{itemize}
    \vspace{-12pt}
    \begin{multicols}{2}
      \begin{itemize}
        \item[Step 1:] Write up buyer's expectation to the car's value given her beliefs regarding \textit{p}.
      \end{itemize}
      \vfill\null\columnbreak
      \begin{enumerate}
        \item $E[V]=\frac{1}{2}1200+\frac{1}{2}2400=1800$
      \end{enumerate}
      \vfill\null
    \end{multicols}
\end{frame}
\begin{frame}{PS8, Ex. 3.c (A): The 'Lemons' model (asymmetric information)}
    \begin{figure}[!h]
      \center
      \def\svgwidth{.8\columnwidth}
      \import{figures/}{3b.pdf_tex}
    \end{figure}
    \vspace{-8pt}
    \begin{itemize}
      \item[(c)] Find a Perfect Bayesian Equilibrium of this model.
    \end{itemize}
    \vspace{-12pt}
    \begin{multicols}{2}
      \begin{itemize}
        \item[Step 1:] Write up buyer's expectation to the car's value given her beliefs regarding \textit{p}.
        \item[Step 2:] \textbf{As both the seller and the buyer know this expectation, what will the outcome be?}
      \end{itemize}
      \vfill\null\columnbreak
      \begin{enumerate}
        \item $E[V]=\frac{1}{2}1200+\frac{1}{2}2400=1800$
      \end{enumerate}
      \vfill\null
    \end{multicols}
\end{frame}
\begin{frame}{PS8, Ex. 3.c (A): The 'Lemons' model (asymmetric information)}
    \begin{figure}[!h]
      \center
      \def\svgwidth{.8\columnwidth}
      \import{figures/}{3b.pdf_tex}
    \end{figure}
    \vspace{-8pt}
    \begin{itemize}
      \item[(c)] Find a Perfect Bayesian Equilibrium of this model.
    \end{itemize}
    \vspace{-12pt}
    \begin{multicols}{2}
      \begin{itemize}
        \item[Step 1:] Write up buyer's expectation to the car's value given her beliefs regarding \textit{p}.
        \item[Step 2:] As both the seller and the buyer know this expectation, what will the outcome be?
      \end{itemize}
      \vfill\null\columnbreak
      \begin{enumerate}
        \item $E[V]=\frac{1}{2}1200+\frac{1}{2}2400=1800$
        \item The seller will not sell \textit{beauties} for a price below 2000. The buyer anticipates this, thus, there will only be a market for \textit{lemons} being sold for $price\in[800,1200]$ as 1200 is the highest amount that the buyer is willing to pay for a \textit{lemon}.
      \end{enumerate}
      \vfill\null
    \end{multicols}
\end{frame}
