\section{PS11, Ex. 6: Spence’s education signaling model (pooling and separating PBE)}

\begin{frame}{PS11, Ex. 6: Spence’s education signaling model (PBE)}
    Consider the following version of Spence’s education signaling model, where a firm is hiring a worker. Workers is characterized by their type $\theta$, which measures their ability. There are two worker types: $\theta\in\{\theta_L,\theta_H\}$. Nature chooses the worker’s type, with $p_H=\mathbb{P}[\theta=\theta_H]$ and $p_H=\mathbb{P}[\theta=\theta_H]=1-p_H$.\\\smallskip
    The worker observes his own type, but the firm does not. The worker can choose his level of education: $e\in\mathbb{R}^{+}$. The cost to him of acquiring this education is $c_\theta(e)=e/\theta$. Education is observed by the firm, who then forms beliefs about the workers type: $\mu(\theta|e)$. We assume that the marginal productivity of a worker is equal to his ability and that the company is in competition such it pays the marginal productivity: $w(e)=\mathbb{E}[\theta|e]$. Thus, the payoff to a worker conditional on his type and education is $u_\theta(e)=w(e)-c_\theta(e)$. Suppose for this exercise that $\theta_H=3$ and $\theta_L=1$.\vspace{-4pt}
    \begin{itemize}
      \item[(a)] Find a separating pure strategy Perfect Bayesian Equilibrium.
      \item[(b)] Find a pooling pure strategy Perfect Bayesian Equilibrium.
    \end{itemize}
    \vfill\null
\end{frame}
\begin{frame}{PS11, Ex. 6: Spence’s education signaling model (PBE)}
    Consider the following version of Spence’s education signaling model, where a firm is hiring a worker. Workers are characterized by their type $\theta$, which measures their ability. There are two worker types: $\theta\in\{\theta_L,\theta_H\}$. Nature chooses the worker’s type, with $p_H=\mathbb{P}[\theta=\theta_H]$ and $p_H=\mathbb{P}[\theta=\theta_H]=1-p_H$.\\\vspace{2pt}
    The worker observes his own type, but the firm does not. The worker can choose his level of education: $e\in\mathbb{R}^{+}$. The cost to him of acquiring this education is $c_\theta(e)=e/\theta$. Education is observed by the firm, who then forms beliefs about the workers type: $\mu(\theta|e)$. We assume that the marginal productivity of a worker is equal to his ability and that the company is in competition such it pays the marginal productivity: $w(e)=\mathbb{E}[\theta|e]$. Thus, the payoff to a worker conditional on his type and education is $u_\theta(e)=w(e)-c_\theta(e)$. Suppose for this exercise that $\theta_H=3$ and $\theta_L=1$.\vspace{-6pt}
    \begin{multicols}{2}
      \begin{itemize}
        \item[(a)] Find a separating pure strategy PBE.
        \item[(b)] Find a pooling pure strategy PBE.
      \end{itemize}\vspace{-2pt}
      \textbf{Given the firm expects a worker to be type $\bm{\theta_L}$ and pays $\bm{\theta_L=1}$, find the utility maximizing education level for each type:}\vspace{-4pt}
      \begin{itemize}
        \item[Type $\theta_L$:] $\max\limits_{e_L}u_{\theta_L}(e_L)$\\
        \item[Type $\theta_H$:] $\max\limits_{e_L}u_{\theta_H}(e_L)$
      \end{itemize}
      \vfill\null\columnbreak
      \vfill\null
    \end{multicols}
\end{frame}
\begin{frame}{PS11, Ex. 6: Spence’s education signaling model (PBE)}
    Consider the following version of Spence’s education signaling model, where a firm is hiring a worker. Workers are characterized by their type $\theta$, which measures their ability. There are two worker types: $\theta\in\{\theta_L,\theta_H\}$. Nature chooses the worker’s type, with $p_H=\mathbb{P}[\theta=\theta_H]$ and $p_H=\mathbb{P}[\theta=\theta_H]=1-p_H$.\\\vspace{2pt}
    The worker observes his own type, but the firm does not. The worker can choose his level of education: $e\in\mathbb{R}^{+}$. The cost to him of acquiring this education is $c_\theta(e)=e/\theta$. Education is observed by the firm, who then forms beliefs about the workers type: $\mu(\theta|e)$. We assume that the marginal productivity of a worker is equal to his ability and that the company is in competition such it pays the marginal productivity: $w(e)=\mathbb{E}[\theta|e]$. Thus, the payoff to a worker conditional on his type and education is $u_\theta(e)=w(e)-c_\theta(e)$. Suppose for this exercise that $\theta_H=3$ and $\theta_L=1$.\vspace{-6pt}
    \begin{multicols}{2}
      \begin{itemize}
        \item[(a)] Find a separating pure strategy PBE.
        \item[(b)] Find a pooling pure strategy PBE.
      \end{itemize}\vspace{-6pt}
      Given the firm expects a worker to be type $\theta_L$ and pays $\theta_L=1$, find the utility maximizing education level for each type:\vspace{-14pt}
      \begin{itemize}
        \item[Type $\theta_L$:] $\max\limits_{e_L}u_{\theta_L}(e_L)=\max\limits_{e_L}1-e_L\Rightarrow e_L=0$\\
        \item[Type $\theta_H$:] $\max\limits_{e_L}u_{\theta_H}(e_L)=\max\limits_{e_L}1-\frac{e_L}{3}\Rightarrow e_L=0$
      \end{itemize}\vspace{-4pt}
      \textbf{Write up the firm's profit from \textit{h} (hiring) or \textit{n} (not hiring) for each type sending a signal of either $\bm{e_H}$ or $\bm{e_L}$.}
      \vfill\null\columnbreak
      \vfill\null
    \end{multicols}
\end{frame}
\begin{frame}{PS11, Ex. 6: Spence’s education signaling model (PBE)}
    Consider the following version of Spence’s education signaling model, where a firm is hiring a worker. Workers are characterized by their type $\theta$, which measures their ability. There are two worker types: $\theta\in\{\theta_L,\theta_H\}$. Nature chooses the worker’s type, with $p_H=\mathbb{P}[\theta=\theta_H]$ and $p_H=\mathbb{P}[\theta=\theta_H]=1-p_H$.\\
    The worker observes his own type, but the firm does not. The worker can choose his level of education: $e\in\mathbb{R}^{+}$. The cost to him of acquiring this education is $c_\theta(e)=e/\theta$. Education is observed by the firm, who then forms beliefs about the workers type: $\mu(\theta|e)$. We assume that the marginal productivity of a worker is equal to his ability and that the company is in competition such it pays the marginal productivity: $w(e)=\mathbb{E}[\theta|e]$. Thus, the payoff to a worker conditional on his type and education is $u_\theta(e)=w(e)-c_\theta(e)$. Suppose for this exercise that $\theta_H=3$ and $\theta_L=1$.\vspace{-8pt}
    \begin{multicols}{2}
      \begin{itemize}
        \item[(a)] Find a separating pure strategy PBE.
        \item[(b)] Find a pooling pure strategy PBE.
      \end{itemize}\vspace{-6pt}
      Given the firm expects a worker to be type $\theta_L$ and pays $\theta_L=1$, find the utility maximizing education level for each type:\vspace{-16pt}
      \begin{itemize}
        \item[Type $\theta_L$:] $\max\limits_{e_L}u_{\theta_L}(e_L)=\max\limits_{e_L}1-e_L\Rightarrow e_L=0$\\
        \item[Type $\theta_H$:] $\max\limits_{e_L}u_{\theta_H}(e_L)=\max\limits_{e_L}1-\frac{e_L}{3}\Rightarrow e_L=0$
      \end{itemize}\vspace{-8pt}
      \begin{align*}
        \pi_{F}=\left\{\begin{array}{rl}
           -2 & \text{for }h|e_H(\theta_L) \\
            0 & \text{for }n\vee h|e_H(\theta_H)\vee h|e_L(\theta_L) \\
            2 & \text{for }h|e_L(\theta_H)
        \end{array}\right.
      \end{align*}
      \vfill\null\columnbreak
      \textbf{Draw the extensive form of this signaling game where either type of worker sends the signal of taking an education of level $\bm{e_L}=0$ or $\bm{e_H>0}$.\\
      Observing the signal, the firm \textit{F} forms their beliefs and choose \textit{h} (hire and pay the wage according to the education level) or \textit{n} (not hire).}
      \vfill\null
    \end{multicols}
\end{frame}
\begin{frame}{PS11, Ex. 6: Spence’s education signaling model (PBE)}
    Consider the following version of Spence’s education signaling model, where a firm is hiring a worker. Workers are characterized by their type $\theta$, which measures their ability. There are two worker types: $\theta\in\{\theta_L,\theta_H\}$. Nature chooses the worker’s type, with $p_H=\mathbb{P}[\theta=\theta_H]$ and $p_H=\mathbb{P}[\theta=\theta_H]=1-p_H$.\\
    The worker observes his own type, but the firm does not. The worker can choose his level of education: $e\in\mathbb{R}^{+}$. The cost to him of acquiring this education is $c_\theta(e)=e/\theta$. Education is observed by the firm, who then forms beliefs about the workers type: $\mu(\theta|e)$. We assume that the marginal productivity of a worker is equal to his ability and that the company is in competition such it pays the marginal productivity: $w(e)=\mathbb{E}[\theta|e]$. Thus, the payoff to a worker conditional on his type and education is $u_\theta(e)=w(e)-c_\theta(e)$. Suppose for this exercise that $\theta_H=3$ and $\theta_L=1$.\vspace{-8pt}
    \begin{multicols}{2}
      \begin{itemize}
        \item[(a)] Find a separating pure strategy PBE.
        \item[(b)] Find a pooling pure strategy PBE.
      \end{itemize}\vspace{-6pt}
      Given the firm expects a worker to be type $\theta_L$ and pays $\theta_L=1$, find the utility maximizing education level for each type:\vspace{-18pt}
      \begin{itemize}
        \item[Type $\theta_L$:] $\max\limits_{e_L}u_{\theta_L}(e_L)=\max\limits_{e_L}1-e_L\Rightarrow e_L=0$\\
        \item[Type $\theta_H$:] $\max\limits_{e_L}u_{\theta_H}(e_L)=\max\limits_{e_L}1-\frac{e_L}{3}\Rightarrow e_L=0$
      \end{itemize}\vspace{-12pt}
      \begin{align*}
        \pi_{F}=\left\{\begin{array}{rl}
           -2 & \text{for }h|e_H(\theta_L) \\
            0 & \text{for }n\vee h|e_H(\theta_H)\vee h|e_L(\theta_L) \\
            2 & \text{for }h|e_L(\theta_H)
        \end{array}\right.
      \end{align*}
      \vfill\null\columnbreak
      Draw the extensive form:\vspace{-14pt}
      \begin{figure}[!h]
        \center\def\svgwidth{1.1\columnwidth}
        \import{figures/}{Spence.pdf_tex}
      \end{figure}\vspace{-4pt}
      \textbf{Now, solve question (a) and (b).}
      \vfill\null
    \end{multicols}
\end{frame}


\subsection{PS11, Ex. 6.a: Spence’s education signaling model (separating PBE)}

\begin{frame}{PS11, Ex. 6.a: Spence’s education signaling model (PBE)}
    \begin{multicols}{2}
      \begin{itemize}
        \item[(a)] Find a separating pure strategy PBE.
        \item[Step 1:] \textbf{Looking at the game tree, which separating PBE are not viable?}
      \end{itemize}\vspace{-6pt}
      \vfill\null\columnbreak
      \begin{figure}[!h]
        \center
        \def\svgwidth{1.1\columnwidth}
        \import{figures/}{Spence.pdf_tex}
      \end{figure}\vspace{-4pt}
      \vfill\null
    \end{multicols}
\end{frame}
\begin{frame}{PS11, Ex. 6.a: Spence’s education signaling model (PBE)}
    \begin{multicols}{2}
      \begin{itemize}
        \item[(a)] Find a separating pure strategy PBE.
        \item[Step 1:] Looking at the game tree, which separating PBE are not viable?
      \end{itemize}\vspace{-6pt}
      \vfill\null\columnbreak
      \begin{figure}[!h]
        \center\def\svgwidth{1.1\columnwidth}
        \import{figures/}{Spence_separating_no-PBE.pdf_tex}
      \end{figure}\vspace{-6pt}
      \begin{enumerate}
        \item $(e_L,e_H)$ is not viable, as $F$ would not hire $\theta_L$ despite his high education level $e_H$. Nor is any other PBE where either type of worker is not hired as he would then want to deviate.
      \end{enumerate}
      \vfill\null
    \end{multicols}
\end{frame}
\begin{frame}{PS11, Ex. 6.a: Spence’s education signaling model (PBE)}
    \begin{multicols}{2}
      \begin{itemize}
        \item[(a)] Find a separating pure strategy PBE.
        \item[Step 1:] Looking at the game tree, which separating PBE are not viable?
        \item[Step 2:] \textbf{Instead, go over SR3, SR2R and SR2S for the PBE candidate $\bm{((e_H,e_L),(h,h))}$.}
      \end{itemize}\vspace{-6pt}
      \vfill\null\columnbreak
      \begin{figure}[!h]
        \center\def\svgwidth{1.1\columnwidth}
        \import{figures/}{Spence_separating.pdf_tex}
      \end{figure}\vspace{-6pt}
      \begin{enumerate}
        \item $(e_L,e_H)$ is not viable, as $F$ would not hire $\theta_L$ despite his high education level $e_H$. Nor is any other PBE where either type of worker is not hired as he would then want to deviate.
      \end{enumerate}
      \vfill\null
    \end{multicols}
\end{frame}
\begin{frame}{PS11, Ex. 6.a: Spence’s education signaling model (PBE)}
    \begin{multicols}{2}
      \begin{itemize}
        \item[(a)] Find a separating pure strategy PBE.
        \item[Step 1:] Looking at the game tree, which separating PBE are not viable?
        \item[Step 2:] Go over SR3, SR2R and SR2S for the PBE candidate $((e_H,e_L),(h,h))$:
        \item[SR3:] F: Beliefs given worker's strategy:
      \end{itemize}\vspace{-10pt}
      \begin{align*}
        \mu(\theta_H|e_L)=p=0\text{ and }\mu(\theta_H|e_H)=q=1
      \end{align*}\vspace{-20pt}
      \begin{itemize}
        \item[SR2R:] F: Is indifferent between $h$ and $n$.
        \item[SR2S:] Type $\theta_H$ will not deviate when
      \end{itemize}\vspace{-12pt}
      \begin{align*}
        u_{\theta_H}(e_H,h)&\geq u_{\theta_H}(e_L,h)\\
        3-\frac{e_H}{3}&\geq1\\
        e_H&\leq6
      \end{align*}\vspace{-20pt}
      \begin{itemize}
        \item[] Type $\theta_L$ will not deviate when
      \end{itemize}\vspace{-12pt}
      \begin{align*}
        u_{\theta_L}(e_L,h)&\geq u_{\theta_L}(e_H,h)\\
        1&\geq3-e_H\\
        e_H&\geq2
      \end{align*}
      \vfill\null\columnbreak
      \begin{figure}[!h]
        \center\def\svgwidth{1.1\columnwidth}
        \import{figures/}{Spence_separating.pdf_tex}
      \end{figure}\vspace{-6pt}
      \begin{enumerate}
        \item $(e_L,e_H)$ is not viable, as $F$ would not hire $\theta_L$ despite his high education level $e_H$. Nor is any other PBE where either type of worker is not hired as he would then want to deviate.
        \item No deviation for $e_H\in[2,6]$. It's optimal for $\theta_H$ to choose $e_H=2$ as it's sufficient for credibly signaling his type. Above 2, worker's marginal effect of education is negative.
      \end{enumerate}\vspace{-12pt}
      \begin{align*}
        PBE=\{(e_H=2,e_L=0),(h,h),p=0,q=1\}
      \end{align*}
      \vfill\null
    \end{multicols}
\end{frame}


\subsection{PS11, Ex. 6.b: Spence’s education signaling model (pooling PBE)}

\begin{frame}{PS11, Ex. 6.b: Spence’s education signaling model (PBE)}
    \begin{multicols}{2}
      \begin{itemize}
        \item[(b)] Find a pooling pure strategy PBE.
        \item[Step 1:] \textbf{Looking at the game tree, which pooling PBE is not viable?}
      \end{itemize}\vspace{-6pt}
      \vfill\null\columnbreak
      \begin{figure}[!h]
        \center
        \def\svgwidth{1.1\columnwidth}
        \import{figures/}{Spence.pdf_tex}
      \end{figure}\vspace{-4pt}
      \vfill\null
    \end{multicols}
\end{frame}
\begin{frame}{PS11, Ex. 6.b: Spence’s education signaling model (PBE)}
    \begin{multicols}{2}
      \begin{itemize}
        \item[(b)] Find a pooling pure strategy PBE.
        \item[Step 1:] Looking at the game tree, which pooling PBE is not viable?
      \end{itemize}\vspace{-6pt}
      \vfill\null\columnbreak
      \begin{figure}[!h]
        \center\def\svgwidth{1.1\columnwidth}
        \import{figures/}{Spence_pooling_no-PBE.pdf_tex}
      \end{figure}\vspace{-6pt}
      \begin{enumerate}
        \item $(e_H,e_H)$ is not viable, as $F$ would not hire unless the probability of type $\theta_L$ is $p_L=0$.
      \end{enumerate}
      \vfill\null
    \end{multicols}
\end{frame}
\begin{frame}{PS11, Ex. 6.b: Spence’s education signaling model (PBE)}
    \begin{multicols}{2}
      \begin{itemize}
        \item[(b)] Find a pooling pure strategy PBE.
        \item[Step 1:] Looking at the game tree, which pooling PBE is not viable?
        \item[Step 2:] \textbf{Instead, go over SR3, SR2R and SR2S for the PBE candidate $\bm{((e_L,e_L),(h,n))}$.}
      \end{itemize}\vspace{-6pt}
      \vfill\null\columnbreak
      \begin{figure}[!h]
        \center\def\svgwidth{1.1\columnwidth}
        \import{figures/}{Spence_pooling.pdf_tex}
      \end{figure}\vspace{-6pt}
      \begin{enumerate}
        \item $(e_H,e_H)$ is not viable, as $F$ would not hire unless the probability of type $\theta_L$ is $p_L=0$.
      \end{enumerate}
      \vfill\null
    \end{multicols}
\end{frame}
\begin{frame}{PS11, Ex. 6.b: Spence’s education signaling model (PBE)}
    \begin{multicols}{2}
      \begin{itemize}
        \item[(b)] Find a pooling pure strategy PBE.
        \item[Step 1:] Looking at the game tree, which pooling PBE is not viable?
        \item[Step 2:] Go over SR3, SR2R and SR2S for the PBE candidate $((e_L,e_L),(h,n))$:
        \item[SR3:] F: Beliefs given worker's strategy:
      \end{itemize}\vspace{-10pt}
      \begin{align*}
        \mu(\theta_H|e_L)=p_H\text{ and }\mu(\theta_H|e_H)=q\in[0,1]
      \end{align*}\vspace{-20pt}
      \begin{itemize}
        \item[SR2R:] F: $(h,n)$ is strictly dominant except for probability $p_H=0$ and belief $q=1$ where it's weakly dominant.
        \item[SR2S:] Type $\theta_H$ will not deviate as
      \end{itemize}\vspace{-12pt}
      \begin{align*}
        u_{\theta_H}(e_L,h)=1>-\frac{e_H}{3}=u_{\theta_H}(e_H,n)
      \end{align*}\vspace{-20pt}
      \begin{itemize}
        \item[] Type $\theta_L$ will not deviate when
      \end{itemize}\vspace{-12pt}
      \begin{align*}
        u_{\theta_L}(e_L,h)=1>-e_H=u_{\theta_L}(e_H,n)
      \end{align*}\vspace{-20pt}
      \begin{itemize}
        \item[Step 3:] \textbf{Explain: Which 2 assumptions make it possible to have an equilibrium where both high-ability and low-ability workers take zero education?}
      \end{itemize}
      \vfill\null\columnbreak
      \begin{figure}[!h]
        \center\def\svgwidth{1.1\columnwidth}
        \import{figures/}{Spence_pooling.pdf_tex}
      \end{figure}\vspace{-6pt}
      \begin{enumerate}
        \item $(e_H,e_H)$ is not viable, as $F$ would not hire unless the probability of type $\theta_L$ is $p_L=0$.
        \item No deviation, thus, we have a PBE:
      \end{enumerate}\vspace{-12pt}
      \begin{align*}
        \{(e_L=0,e_L=0),(h,n),p=p_H,q\in[0,1]\}
      \end{align*}
      \vfill\null
    \end{multicols}
\end{frame}
\begin{frame}{PS11, Ex. 6.b: Spence’s education signaling model (PBE)}
    \begin{multicols}{2}
      \begin{itemize}
        \item[(b)] Find a pooling pure strategy PBE.
        \item[Step 1:] Looking at the game tree, which pooling PBE is not viable?
        \item[Step 2:] Go over SR3, SR2R and SR2S for the PBE candidate $((e_L,e_L),(h,n))$:
        \item[SR3:] F: Beliefs given worker's strategy:
      \end{itemize}\vspace{-10pt}
      \begin{align*}
        \mu(\theta_H|e_L)=p_H\text{ and }\mu(\theta_H|e_H)=q\in[0,1]
      \end{align*}\vspace{-20pt}
      \begin{itemize}
        \item[SR2R:] F: $(h,n)$ is strictly dominant except for probability $p_H=0$ and belief $q=1$ where it's weakly dominant.
        \item[SR2S:] Type $\theta_H$ will not deviate as
      \end{itemize}\vspace{-12pt}
      \begin{align*}
        u_{\theta_H}(e_L,h)=1>-\frac{e_H}{3}=u_{\theta_H}(e_H,n)
      \end{align*}\vspace{-20pt}
      \begin{itemize}
        \item[] Type $\theta_L$ will not deviate when
      \end{itemize}\vspace{-12pt}
      \begin{align*}
        u_{\theta_L}(e_L,h)=1>-e_H=u_{\theta_L}(e_H,n)
      \end{align*}\vspace{-20pt}
      \begin{itemize}
        \item[Step 3:] Explain: Which 2 assumptions make it possible to have an eq. where both high-ability and low-ability workers take zero education?
      \end{itemize}
      \vfill\null\columnbreak
      \begin{figure}[!h]
        \center\def\svgwidth{1.1\columnwidth}
        \import{figures/}{Spence_pooling.pdf_tex}
      \end{figure}\vspace{-6pt}
      \begin{enumerate}
        \item $(e_H,e_H)$ is not viable, as $F$ would not hire unless the probability of type $\theta_L$ is $p_L=0$.
        \item \vspace{-2pt} No deviation, thus, we have a PBE:
      \end{enumerate}\vspace{-14pt}
      \begin{align*}
        \{(e_L=0,e_L=0),(h,n),p=p_H,q\in[0,1]\}
      \end{align*}\vspace{-24pt}
      \begin{itemize}
        \item[3.i] Education is unproductive; it only serves as a signal of one's ability.
        \item[3.ii] \vspace{-4pt} Under competition, $F$ pays marginal productivity and is indifferent between $n$, $h(e_L)|\theta_L$, and $h(e_H)|\theta_H$. Thus, $F$ has no reason to run the risk of overpaying $\theta_L$ imitating $\theta_H$.
      \end{itemize}
      \vfill\null
    \end{multicols}
\end{frame}
